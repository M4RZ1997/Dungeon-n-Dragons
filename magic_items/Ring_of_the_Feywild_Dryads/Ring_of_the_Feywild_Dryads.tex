\documentclass[letterpaper,openany,oneside,twocolumn]{book}

\newcommand{\PATH}{../../}

\usepackage{fontspec}
\usepackage[justified]{\PATH dndtemplate/dnd}
\usepackage{ifthen}
\usepackage{pstricks}

\usepackage{intcalc}

\usepackage[UKenglish]{babel}
\usepackage{\PATH dndtemplate}

\usepackage{\PATH magic_items/stylesheets/magic_items_commands}

\setlength\oddsidemargin{\dimexpr(\paperwidth-\textwidth)/2 - 1in\relax}
\setlength\evensidemargin{\oddsidemargin}

\begin{document}
\chapter*{Ring of the Feywild Dryads}
\itemDescriptionAndImage{Ring, uncommon (requires attunement)}{Ring_of_Meld_into_Trees.png}{11.25cm}
\section*{Appearance}
The "Ring of the Feywild Dryads" appears as a masterfully crafted piece of ancient magic and natural wonder. The band, formed from what seems to be twisted silver, mimics the intricate intertwining of tree branches, encircling the finger with a grace that speaks of the forest's deep and mystic ties. This silver has been textured to resemble bark, enhancing its bond with the trees it represents. At the heart of the ring, a large, captivating emerald sits prominently, its deep green hue glowing softly as if lit from within, symbolizing the life force of the forest and the essence of tree magic. Delicate Elvish runes, suggestive of enchantment and age-old secrets, are engraved along the band, hinting at the ring's ancient origins and its potent ability to connect the wearer with the natural world. This ring is not merely an accessory but a powerful artifact that embodies the spirit of the forest and the magic of melding with trees.
\vfill\eject
\section*{History}
The "Ring of the Feywild Dryads" is not merely a magical item; it is a relic of deep history and symbiotic power, born from the ancient alliance between the mortal realms and the feywilds. Crafted in an age long forgotten, under the boughs of the Eternal Forest in the heart of the feywilds, this ring was a gift from the dryads to the first of the Elven druids who walked the path between worlds.

The dryads, spirits of the trees, and guardians of the forest's deepest magics, sought to imbue this ring with the essence of their being and the forests they protect. Using silver that was mined from the sacred mountains surrounding the Eternal Forest and blessed by the light of the full moon, the ring was forged in the shape of intertwining branches, a symbol of life's complexity and interconnection. The centerpiece, a radiant emerald, was chosen for its resemblance to the heart of the forest - a gem that had captured the very essence of the feywild's verdant splendor.

Upon its creation, the Ring of the Feywild Dryads was bestowed upon an elf, chosen for their unwavering dedication to protecting the natural world. This elf, a druid of unparalleled connection to the earth's rhythms and cycles, became the first of many guardians who would wield the ring's power to merge with the trees, learning their secrets, and protecting the forests from those who would do them harm.

Over the centuries, the ring passed from one guardian to another, always finding its way to those who had proven themselves protectors of nature's sanctity. Each bearer left their mark on the ring, contributing to its legacy and the stories it holds within. Legends say that the ring still carries the whispers of the dryads and the echoes of the Eternal Forest, guiding its wearer not only to meld into trees but to understand the very language of the forest.

Today, the "Ring of the Feywild Dryads" remains a bridge between the mortal realms and the enchantments of the feywilds, a testament to the enduring alliance between the elves and the dryads. It is a reminder of the deep bonds that can form when different beings unite in a common cause, protecting the natural world and preserving its wonders for generations to come.
\section*{Magic}
While attuned to the ring, its wearer can cast the "Meld into Trees" spell once per long rest.
\end{document}