\chapter*{Ingranagg}
{\vspace*{-3em}\entryfont\footnotesize Level 5, 2 Special Facilities}

\begin{tikzpicture}[remember picture, overlay]%
	% Background
	\node[xshift=-0.5cm, yshift=-0.5cm, anchor=south west] at (current page.south west) {\includegraphics[width=0.75\paperwidth]{%
		images/Ingranagg_background%
	}};%
	
	% S.C.R.I.B.E
	\node[xshift=0.75cm, yshift=0.5cm, anchor=north east] at (current page.north east) {\includegraphics[width=0.25\paperwidth]{%
		images/Arcane_Study_Hireling%
	}};%
	
	% Workshop Hirelings
	\node[xshift=0.5cm, yshift=-0.5cm, anchor=south east] at (current page.south east) {\includegraphics[width=0.5\paperwidth]{%
		images/Workshop_Hirelings%
	}};%
\end{tikzpicture}%

{\entryfont\noindent Rising from the ordered streets of \textbf{[redacted]}, the Bastion is less a building than a vast, living mechanism. Walls of burnished brass interlock with turning cogs, great pistons drive hidden machinery, and lattices of copper piping breathe quiet hisses of steam. Windows gleam like polished lenses, while doorways shift and seal with the precision of clockwork. Every surface hums with subtle movement, as though the structure itself is always thinking, always working - an engine of stone and metal, standing in harmony with the eternal order of Mechanus.}

\section*{Special Facilities}
\subsection*{\inTextCircled[10]{3}{10} Arcane Study}
{\textit{Roomy, 1 Hireling}}

{\entryfont \paragraph*{Craft: Arcane Focus} You commission the facility's hireling to craft an Arcane Focus. Ther work takes 7 days and costs no money. The Arcane Focus remains in your Bastion until you claim it.}

{\entryfont \paragraph*{Craft: Book} You commission the facility's hireling to craft a blank book. The work takes 7 days and costs you 10 GP. The book remains in your Bastion until you claim it.}

{\entryfont \paragraph*{Arcane Study Charm} After spending a Long Rest in your Bastion, you gain a magical Charm that lasts for 7 days or until you use it. The Charm allows you to cast Identify without expending a spell slot or using Material components. You can't gain this  Charm again while you still have it.}

{\entryfont This is the domain of S.C.R.I.B.E, the meticulous little automaton designed by A.R.T.I.F.I.C.E.R-$\beta$ to maintain the Bastion's growing archive of lore.}
\subsubsection*{S.C.R.I.B.E}
{\entryfont Standing no taller than a halfling child, S.C.R.I.B.E is a clockwork marvel of articulated brass limbs, gemstone optics, and rune-etched plates. At its core rests a \textbf{Prismatic Codex-Engine}, a rare crystalline cube that refracts not light, but knowledge itself - gleaming with shifting runes and layered truths, like a living tesseract of thought. This singular artifact is what grants S.C.R.I.B.E its uncanny ability to sift, organize, and cross-reference vast bodies of lore with tireless precision.

Calm, fastidious, and unfailingly polite, S.C.R.I.B.E speaks in a measured, parchment-dry tone, forever eager to assist with research or the annotation of magical findings. It is loyal to its master and the Bastion, but curiosity often pulls it down rabbit-holes of obscure scholarship.

S.C.R.I.B.E is a peculiar little construct, and its eccentricities are as notable as its scholarship. It has an unshakable compulsion to catalogue every magical item it encounters, carefully drafting notes\\and filing references as if the world itself\\were one great index waiting to be\\completed. When spoken to, it often\\replies with cross-referenced citations, quoting passages from obscure scrolls or long-forgotten treatises, even if they have little to do with the matter at hand. Stranger still is its fascination with the discarded tools of writing - broken quills, half-spilled ink pots, and blotched scraps of parchment are gathered and preserved as though they were relics of profound significance, each stored away in a private archive it guards with surprising reverence.}

\subsection*{\inTextCircled[10]{4}{10} Workshop}
{\textit{Roomy, 3 Hirelings}}

{\entryfont \paragraph*{Chosen Artisan's Tools} Carpenter's Tools, Glassblower's Tools, Leatherworker's Tools, Tinker's Tools, Weaver's Tools, Woodcarver's Tools}

{\entryfont \paragraph*{Craft: Adventuring Gear} The facility hirelings craft anything that can be made with the tools you chose when you added the Workshop to your Bastion, using the rules in the Player's Handbook.}

{\entryfont \paragraph*{Source of Inspiration} After spending an entire Short Rest in your Workshop, you gain Heroic Inspiration. You can't gain this benefit again until you finish a Long Rest.}

{\entryfont \paragraph*{Enlarging the facility} You can enlarge your Workshop to a Vast facility by spending 2'000 GP. If you do so, the Workshop gains two additional hirelings and three additional Artisan's Tools.}
\subsubsection*{Crucibelle - Glass \& Tinkering}
{\entryfont A delicate construct of bronze and glass tubing, Crucibelle manipulates flame and sand with jewelers' precision. She speaks in a faint chiming voice, like clinking glass, and often "breathes" mist when excited.}
\subsubsection*{Splinter - Woodworking}
{\entryfont A sturdy, saw-limbed automaton of oak and iron, Splinter is built for carpentry and joinery. It hums softly while working, leaving behind meticulous wood-shavings it insists on sweeping into neat spirals.}
\subsubsection*{Stitch - Textile \& Leatherworks}
{\entryfont Draped in thread-spools and needle-arms, Stitch is a nimble textile-worker of brass joints and soft leather. It hums lullabies as it sews, and grows oddly attached to scraps of cloth it refuses to discard.}
\vfill\eject
\section*{Basic Facilities}
\subsection*{\inTextCircled[10]{1}{10} Parlor}
{\entryfont The parlor stretches out as a grand, high-ceilinged hall, its walls paneled in dark wood framed by gleaming brass supports, each one etched with turning cogs that quietly whir as if the room itself breathes. Tables of polished iron and oak are arranged throughout, their surfaces inlaid with intricate gearwork patterns that occasionally shift and realign, never quite the same twice. A wide fireplace dominates one wall, its mantle crowned by slowly rotating vents that cycle warmth throughout the chamber, filling it with a steady glow and the faint hiss of pressure valves.}

{\entryfont At the far end stands a compact but well-stocked bar, its shelves rotating on a vertical gear-track to present bottles with mechanical precision. Not far from it rests a piano of wrought iron and polished brass, its keys of bone-white ivory gleaming under lamplight, the entire instrument faintly humming with tensioned springs, ready to play at the lightest touch. The air here is warm and welcoming despite the omnipresent rhythm of ticking gears, making the parlor a space where invention and comfort find rare harmony.}
\subsection*{\inTextCircled[10]{2}{10} Bedroom}
{\entryfont In stark contrast to the grandeur of the parlor, the bedroom is a sparse and utilitarian chamber. The walls are bare metal plates with only the faint rhythm of hidden gears for ornament. A simple bed of iron frame and stiff mattress stands against one wall, accompanied by a single writing desk bolted directly into the floor. A narrow cabinet provides just enough storage, its drawers sliding open with the soft click of clockwork. The room is functional, precise, and utterly without excess - designed for rest and nothing more.}

\begin{tikzpicture}[remember picture, overlay]%
	% Background
	\node[yshift=-1cm, anchor=south] at (current page.south) {\includegraphics[width=\paperwidth]{%
		images/Ingranagg_Floorplan%
	}};%
	
	% Identification Numbers
	\begin{scope}[scale=0.25, xshift=-4cm, yshift=-30cm]
		\pgfmathsetmacro{\IndicatorSize}{12*\scaleFactor}%
		
		\indicatorNumberField[10]{(37,5)}{\IndicatorSize}
		\indicatorNumberField[10]{(25,25)}{\IndicatorSize}
		\indicatorNumberField[10]{(65,5)}{\IndicatorSize}
		\indicatorNumberField[10]{(15,10)}{\IndicatorSize}
	\end{scope}
\end{tikzpicture}%
\vfill\eject
\section*{Fall of a Bastion}
{\entryfont A player character can lose their Bastion in the following ways:}\\
{\entryfont\paragraph*{Divestiture} A character can give up their Bastion anytime, releasing the Bastion's hirelings and abandoning the location. The divested Bastion is quickly vacated, is eventually looted, and might even be burned to the ground.}

{\entryfont\paragraph*{Neglect} If a character issues no order to their Bastion for a number of consecutive Bastion turns equal to the character's level (typically because the character's dead or otherwise out of commission), the hirelings abandon the Bastion and the site is eventually looted. If the character returns later, they can start a new Bastion, perhaps building it amid the ruins of the old one.}

{\entryfont\paragraph*{Ruination} Drawing the Ruin card from the Deck of Many Things instantly deprives a character of their Bastion. When such an event occurs, the player can decide what terrible fate befalls the Bastion. The Bastion might be sacked by enemies or destroyed by an earthquake, for example.}\\\\
{\entryfont Regardless of how the Bastion falls, the player can work with the DM to establish a new Bastion and determine how it comes into being. Use the Special Facility Acquisition table to determine how many special facilities come with it. The new Bastion also starts with two basic facilities (one Cramped and one Roomy) of the player's choice.}