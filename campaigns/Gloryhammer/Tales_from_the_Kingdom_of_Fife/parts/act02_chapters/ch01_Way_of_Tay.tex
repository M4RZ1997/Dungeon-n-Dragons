\chapter*{Way of Tay}\stepcounter{chapter}\phantomsection\addcontentsline{toc}{chapter}{Way of Tay}

\DndDropCapLine{S}{\entryfont etting out along the paved road most travelled, the party makes their way toward the distant citadel of Crail. The path follows the rugged eastern coastline of Caledonia, cutting through windswept highlands, brackish moors, and sparse coastal woods. Early in their journey, grim sights greet them - bodies left where they fell beside the road, likely survivors of the siege who were not as fortunate as others.}

{\entryfont The journey takes approximately two days. The party will need to rest at least once, most likely in the mist-choked moors near the estuary of the river Eden.

If \prettyref{or:HastyDeparture} is in effect, the adventurers must forage for food and water, or find other means to sustain themselves during travel.

Throughout their journey, the party will face a series of challenges. One encounter should take place during the first day of travel, and another on the second. These can be selected by the DM, determined randomly, or influenced by the party's choices - for example, encountering a spriggan while foraging in a nearby grove. During the night the party will experience a fixed event tied to an ancient Caledonian legend. This encounter is narrative in nature and intended to build atmosphere and mystery rather than present a direct threat.}

\section*{Random Encounter}\phantomsection\addcontentsline{toc}{section}{Random Encounter}
\subsection*{Travelling Merchant (Day 1)}
\begin{DndReadAloud}
	A ragged figure stumbles onto the path ahead, clothing torn and caked in mud, his face pale and streaked with dried blood. He raises a trembling hand as you approach. \textit{"My cart... it crashed down the ravine... I barely made it out"}, he gasps, gesturing weakly toward a steep drop nearby. \textit{"Please, my father's sword and crown are down there... family heirlooms. If you retrieve them, you can keep whatever else you find. But be warned... there are things in the marsh - froglike creatures, savage and territorial. I was lucky to get away."}
\end{DndReadAloud}

{\noindent\entryfont To reach the crash site, the party must climb down a 30-foot ravine. This requires a DC 15 Strength (Athletics) check. If the party does not attempt a stealthy descent, they draw the attention of a Grung warband currently looting the cart. The group consists of 4 \hyperref[monster:Grung]{\LinkFont{Grung}}, 2 \hyperref[monster:GrungWildling]{\LinkFont{Grung Wildlings}}, and 1 \hyperref[monster:GrungEliteWarrior]{\LinkFont{Grung Elite Warrior}}. The Grung will scatter and flee once the Elite Warrior is slain or if the party successfully ambushes them.

After the encounter, each party member may search the wreckage of the cart:
\begin{itemize}
	\renewcommand\labelitemi{\textbf{\textbullet}}
	\item \textbf{DC 10 Perception / Investigation Check}\\Heirloom Sword and Crown
	\item \textbf{DC 15 Perception / Investigation Check}\\1 Ration (Apples) for each success
	\item \textbf{DC 20 Perception / Investigation Check}\\2 Potions of Healing and 20 GP
	\item \textbf{NAT20 Perception / Investigation Check}\\Potion of Lesser Restoration
\end{itemize}}

\subsection*{Firbolg Forest (Day 1)}
{\entryfont As the party moves through the coastal woods in search for food, they are approached by a large, moss-covered figure slowly steps into view - a Firbolg with bark-like skin, a tangled beard, and wide, gentle eyes. His presence is calm and uncertain, and expression carries a quiet, worried sorrow.}

\begin{DndReadAloud}
	A strange, thick-bodied and tall creature leaps into your path, eyeing you with hopeful urgency.

	\textit{"Smallfolk! Help Firbolg!"} he rumbles, his voice low and anxious. \textit{"Firbolg lost his Shrubbies. Sing beautiful. Firbolg sad."}

	He steps closer, glancing behind him before continuing:

	\textit{"Shrubbies gone! Four!... No more songs in woods. If brave Smallfolk help... Firbolg give... Nature Wonder! Yes! Also show good food place. Many berries. Fat squirrels!"}
\end{DndReadAloud}

{\noindent\entryfont If the party agrees to help, the Firbolg brightens immediately, clapping with joy. He begins to describe his four lost companions - sentient, singing plants he affectionately calls his "Shrubbies".}

\begin{DndQuestHook}[width=0.5\textwidth - 4pt]{{\large Mis}SING Shrubbies}
	\DndQuestHookBasics[
		location = {Way of Tay, South of Dundee},
		quest-giver = {Firbolg},
		objective = {Find the 4 Awakened Shrubs in the woods.},
	]
	
	{%
		\noindent\entryfont Before searching for the Awakened Shrubs the party can deduce where the Firbolg last saw them before losing track of them (\textbf{DC 13 Insight Check}), thereby reducing every subsequent DC by 2. Ideas how to find the Shrubbies are stated below - but the players can also come up with ideas of their own:
		\begin{itemize}
			\renewcommand\labelitemi{\textbf{\textbullet}}
			\item \textbf{DC 17 Survival - \textit{"Little Leafy Tracks"}}\\Tracking the small, erratic prints of the Shrubbies through mud, fallen leaves, or soft moss. Some are partially disguised as natural disturbances.
			\item \textbf{DC 13 Perception - \textit{"Hear the Hum"}}\\The Awakened Shrubs emit faint, melodic hums when alone. A character who listens carefully can hear one humming in the distance, helping the party home in on its location.
			\item \textbf{DC 15 Arcana - \textit{"Magical Signatures"}}\\A character senses faint traces of druidic magic lingering in the area and uses that arcane residue to triangulate a Shrubby's location.
			\item \textbf{DC 17 Nature - \textit{"Follow the Roots"}}\\A character uses their knowledge of local flora to identify signs of movement among plants that may suggest the Shrubbies passed through, such as unusual vine trails or bent branches.
			\item \textbf{DC 13 Performance - \textit{"Sing Along"}}\\One Shrubby responds to song. A character can sing or play an instrument to draw it out of hiding, encouraging it to sing back and reveal itself.
			\item \textbf{DC 15 Animal Handling - \textit{"Ask the Woodland Creatures"}}\\The character attempts to communicate or bribe a small woodland animal, like a squirrel or jay, to point toward one of the Shrubbies' hiding spots.
		\end{itemize}
	}%
	
	\DndQuestRewards{Upon successfully retrieving the Awakened Shrubs, the Firbolg will reward the party with the following items:}
	{%
		{5 Dried Leeches}{}%
		{Conditional Magic Item}{%
			Depending on which classes your players are, you can give them a suitable uncommon item from the following list:
			\begin{itemize}
				\item If a party member is a Druid or Ranger:\\
				\textbf{Nature's Mantle}
				\item If a party member is a Bard:\\
				\textbf{Rhythm Maker's Drum}
				\item Otherwise:\\
				\textbf{Gloves of Thievery}
			\end{itemize}
		}%
		{Plentiful Gathering Grounds}{The Firbolg eagerly leads the party to a hidden grove teeming with wild edibles. Each adventurer is able to gather enough to create 2 rations of fresh, nourishing food.}
	}%
\end{DndQuestHook}

{\noindent\entryfont As an additional thank-you once the Shrubbies are found, the Firbolg insists they sing for the party. The four leafy performers sway, rustle, and begin their enthusiastic serenade. However, the melody is far from pleasant. Each adventurer must make a \textbf{DC 12 Constitution Saving Throw} or suffer 1d6 psychic damage from the Shrubbies' truly dreadful performance.}

\section*{Soul Snatching Feline}\phantomsection\addcontentsline{toc}{section}{Soul Snatching Feline}
{\entryfont In the middle of the night, while the players are fast asleep, any player with a \textbf{Passive Perception of 15 or higher} is awoken by an eerie howling sound. The Cait-Sìth is not intended as a combat encounter but as a narrative and atmospheric event, emphasizing the mystical and eerie tone of the moors. Read aloud or paraphrase:}
\begin{DndReadAloud}
	As the night deepens and the mist thickens over the estuary of the River Eden, an otherworldly wail slices through the stillness, its haunting tones reverberating across the moors. Turning towards the sound, you glimpse a faint green light drifting through the shallow marsh, a brighter white glow hovering beneath it. The lights move with a ghostly grace, pausing in place as the mournful cry echoes once more. Then, as if guided by some unseen purpose, the lights drift onward, vanishing briefly into the mist before reappearing at the next stop.
\end{DndReadAloud}

{\noindent\entryfont A successful \textbf{DC 18 Perception Check} reveals that the lights belong to a dark, wolf-sized shape skulking through the marsh. Its outline is blurred and ill-defined, as if the mist itself clings to its form, distorting its presence.

The party may attempt to follow the creature and investigate the spots where it pauses. At each stop, they will uncover bodies - likely survivors of the recent siege who have succumbed to their injuries. Despite their efforts, the party will find it impossible to catch up to the creature. It moves with unnatural speed and grace, occasionally seeming to blink out of existence entirely, vanishing into the mist only to reappear further along its path.

A \textbf{DC 15 Religion Check} (or \textbf{History Check} if the character has a Caledonian background) allows a character to recall the local legend of the Cait-Sìth, a spectral feline said to roam the night, collecting souls.}

\begin{DndComment}[color=DmgCoral]{Real World Legend of the Cait-Sìth}
	The Cait-Sìth, \textit{Fairy Cat}, is a spectral creature rooted in Celtic mythology. Described as a large, black feline with piercing green eyes and a distinctive white spot upon its chest, it is said to prowl ancient barrows and linger among the cold, mossy grounds of forgotten cemeteries. There, it watches with silent patience, guarding sacred places and observing as souls slip from the bodies of the recently deceased.

	Legends tell that if the Cait-Sìth reaches a corpse before it has been properly prepared, it will claim the soul, dragging it into the otherworld. To prevent such a fate, wakes were held in vigil, where friends and family would gather around the deceased, filling the air with noise, laughter, and distraction to deter the Cait-Sìth from drawing near.

	The Cait-Sìth endures as a symbol of ritual and community, a spectral reminder of the importance of honouring the dead and guarding their passage to the afterlife.
\end{DndComment}

{\noindent\entryfont Near their camp, the party can choose to perform proper burial rites for one of the bodies. During this process, the Cait-Sìth will silently observe from atop a nearby boulder, its green eyes faintly glowing in the dark. To successfully complete the rite, the party must succeed in both a \textbf{DC 13 Religion Check} and a \textbf{DC 13 Medicine Check}.

If successful, the Cait-Sìth vanishes without a trace, but the party will find a \textbf{Soul Coin} resting on the boulder where it sat.

If unsuccessful, the ethereal form of the Cait-Sìth will descend, snatching the soul of the fallen and fades away into the mist. Each player must make a \textbf{DC 17 Constitution Saving Throw}. On a failure, the player does not recover a level of exhaustion during this Long Rest.}

\section*{Haunted Wreckage}\phantomsection\addcontentsline{toc}{section}{Haunted Wreckage}
{\entryfont On the following morning, as the party continues their journey near the southern edge of the River Eden delta, they spot the remains of a large trade vessel half-buried in the brackish mud. Its hull leans at a precarious angle, split and weatherworn by time and tide. The scene is quiet, but carries a heavy, eerie presence.}

\begin{DndReadAloud}
	The wind flows through the torn sails and crooked masts, producing a mournful howling sound that echoes across the marsh. But beneath that hollow moan, you could swear you hear something else - a voice, faint and whispering, calling you closer. It seems to beckon from within the wreck, promising treasure for those brave enough to claim it.

	A large hole gapes in the stern of the ship, its jagged edges framed by broken timbers. The hollow darkness beyond invites you in.
\end{DndReadAloud}

{\entryfont\noindent A successful DC 19 Perception check reveals fleeting glimpses of small, quick figures darting across the upper deck - barely more than shadows. Throughout the wreck, the party can hear bursts of giggling and faint laughter, made all the more unsettling by the acoustics of the broken ship and the rhythmic creaking of wood. The sounds echo and blend with the howling wind, creating an eerie, almost supernatural atmosphere.

Unbeknownst to the party, these figures are a group of juvenile Grung who have made a hideaway within the wreck.}

\subsection*{Croaking Nuisances}
{\entryfont A small group of juvenile Grung have claimed the River Maiden as their personal playground. They leap across beams, splash through flooded decks, and whisper ghost stories to each other in the dark. As a simple but effective alarm, they've rigged a "Cluttering Clutter Trap" at the entrance hole - an improvised line of shells, potsherds, and bones strung between timbers. If disturbed, it produces a loud clatter that alerts the Grung to any intruders entering their domain.}
\begin{DndTrap}[width=0.5\textwidth - 4pt]{Clattering Clutter Trap}
	\DndTrapBasics[
		save_dc			= 15,
		skill			= {Wisdom (Perception)},
		%pos_reaction	= {Brace},
		%neg_reaction	= {Jump},
		%damage			= {-},
	]
	{\entryfont Rusted cookware, broken bottles, and shells strung together fall loudly when triggered. The Grung children become alerted and gain a +2 bonus when attempting to steal the party's gold and clutter.}
\end{DndTrap}

{\noindent\entryfont The juvenile Grung that inhabit the wreck of the River Maiden are more pranksters than predators. To deter intruders, they've rigged a variety of makeshift traps and nuisances throughout the ship. While most are not truly dangerous, they serve to slow, frustrate, or confuse those exploring the wreck, often with comedic or inconvenient results.

Some of these traps, especially those that restrain or trap a creature (such as the Moss Net), are strategically positioned near hidden crawlspaces and tunnels. From these concealed spots, one or more young Grung may emerge and attempt to pilfer items from distracted adventurers.

When a creature is restrained, prone, or otherwise vulnerable near one of these trap sites, a juvenile Grung can make a DC 10 Sleight of Hand check to steal a small item (a pouch, a coin purse, a ring, etc.) or a minor amount of gold. The exact item stolen is left to DM discretion, and may depend on the degree of success - rolling significantly above the DC may allow a more valuable or specific item to be taken.

All stolen goods are carried away to the Grung's hideout located on the main deck of the River Maiden. This hideout may be discovered later in the encounter and can serve as a reward cache for perceptive or persistent players.}

{\entryfont\paragraph*{Makeshift Traps} Below is a list of makeshift traps. Each includes a potential advantageous and/or detrimental reaction to be used with the custom "Click"-Ruling.}

\begin{DndTrap}[width=0.5\textwidth - 4pt]{Moss Net Drop}
	\DndTrapBasics[
		save_dc			= 12,
		skill			= {Dexterity Saving Throw},
		%pos_reaction	= {Brace},
		neg_reaction	= {Jump},
		%damage			= {-},
	]
	{\entryfont\noindent \textbf{Grung Steal Possible}\\A simple net made of vines and moss drops from above. All creatures within a 10-foot-by-10-foot area that failed the saving throw are restrained by the net. A creature can use its action to make a \textbf{DC 10 Strength Check} to try to free itself or another creature in the net. Dealing 5 Slashing damage to the net (AC 10, HP 20) destroys a 5-foot-square of it, freeing any creature trapped in that section.}
\end{DndTrap}
\begin{DndTrap}[width=0.5\textwidth - 4pt]{Greased Planks}
	\DndTrapBasics[
		save_dc			= 12,
		skill			= {Dexterity Saving Throw},
		pos_reaction	= {Brace},
		%neg_reaction	= {Jump},
		damage			= {\DndDice{1d4} Bludgeoning},
	]
	{\entryfont\noindent A section of the deck has been slicked with fish oil and swamp gunk. Any creature failing the saving throw slips and falls prone.}
\end{DndTrap}
\begin{DndTrap}[width=0.5\textwidth - 4pt]{Sticky Sap Patch}
	\DndTrapBasics[
		save_dc			= 11,
		skill			= {Strength Saving Throw},
		%pos_reaction	= {Brace},
		neg_reaction	= {Duck/Brace},
		%damage			= {-},
	]
	{\entryfont\noindent \textbf{Grung Steal Possible}\\A section of the floor has been smeared with thick, tacky sap. When stepped on, it clings to boots or gear, anchoring the creature in place. Escaping risks losing or tearing equipment.}
\end{DndTrap}
\begin{DndTrap}[width=0.5\textwidth - 4pt]{Springy Log Trap}
	\DndTrapBasics[
		save_dc			= 12,
		skill			= {Dexterity Saving Throw},
		pos_reaction	= {Duck},
		neg_reaction	= {Brace},
		%damage			= {-},
	]
	{\entryfont\noindent A creature failing the saving throw is blinded for 1 Round by flying muck. For 1 Minute the creature has Disadvantage on Wisdom (Perception) Checks that rely on sight.}
\end{DndTrap}
\begin{DndTrap}[width=0.5\textwidth - 4pt]{Trip Vine Pebble Shower}
	\DndTrapBasics[
		save_dc			= 11,
		skill			= {Dexterity Saving Throw},
		pos_reaction	= {Jump},
		%neg_reaction	= {Duck/Brace},
		damage			= {1 Bludgeoning},
	]
	{\entryfont\noindent All creatures within a 5-foot radius have to make the saving throw. Each creature failing the save takes 1 Bludgeoning damage and lose concentration on any spell.}
\end{DndTrap}
\begin{DndTrap}[width=0.5\textwidth - 4pt]{Tied-Together Buckets}
	\DndTrapBasics[
		save_dc			= 12,
		skill			= {Dexterity Saving Throw},
		pos_reaction	= {Freeze/Step Back},
		%neg_reaction	= {Duck/Brace},
		%damage			= {-},
	]
	{\entryfont\noindent Two old pails filled with water or swamp goo swing inward when a character walks through a narrow passage. Each creature in a 15-foot cone must make the save. Each creature that failed the check has Disadvantage on all Charisma-Checks for 10 minutes and must make a \textbf{DC 10 Constitution Saving Throw}, taking \DndDice{1d4} poison damage, or half as much if successful.}
\end{DndTrap}

\subsection*{Traversing the Wreckage}
{\entryfont\noindent Exploring the River Maiden is dangerous - the ship is old, splintered, and partially sunken. Boards groan underfoot, rusted nails jut from shattered beams, and sections of the deck and hull have collapsed entirely. As the party moves through the vessel, they will encounter several natural hazards and obstacles.

\noindent Below are a few examples of traversal challenges; DMs are encouraged to invent their own, provided they keep the danger moderate. Saving throw or skill check DCs for environmental hazards should not exceed 14, and the maximum damage dealt on a failed check should be no more than 2d10.}

{\entryfont \paragraph*{Environmental Hazards} These natural hazards reflect the unpredictable dangers of traversing a wrecked and decaying vessel. They aren't traps in the traditional sense, but they still carry risk - often resulting in minor injury or disorientation if not carefully avoided.
\begin{DndTrap}[width=0.5\textwidth - 4pt]{Flooded Passage}
	\DndTrapBasics[
		save_dc = 12,
		skill = {Wisdom (Perception)},
		damage = {\DndDice{1d10} Piercing},
	]
	{\entryfont\noindent Parts of the lower hold are submerged in cold, murky water. Jagged splinters, rusted nails, and broken crates lurk beneath the surface.}
\end{DndTrap}
\begin{DndTrap}[width=0.5\textwidth - 4pt]{Rotten Hull}
	\DndTrapBasics[
		save_dc = 12,
		skill = {Dexterity Saving Throw},
		damage = {\DndDice{1d8} Bludgeoning},
	]
	{\entryfont\noindent Some of the ship's outer walls and support beams are structurally compromised and may collapse inward.}
\end{DndTrap}
\begin{DndTrap}[width=0.5\textwidth - 4pt]{Slippery Slope}
	\DndTrapBasics[
		save_dc = 13,
		skill = {Dexterity (Acrobatics)},
	]
	{\entryfont\noindent Several interior slopes and gangways are coated in algae and moisture. On a failure, they slide uncontrollably and fall prone, potentially colliding with sharp wreckage for \DndDice{1d6} damage.}
\end{DndTrap}
\begin{DndTrap}[width=0.5\textwidth - 4pt]{Swinging Beam}
	\DndTrapBasics[
		save_dc = 14,
		skill = {Dexterity Saving Throw},
		damage = {\DndDice{2d6} Bludgeoning},
	]
	{\entryfont\noindent An unstable section of ceiling creaks ominously. A sudden gust or shift in weight causes a thick wooden beam to fall from above.}
\end{DndTrap}
}

{\entryfont \paragraph*{Obstacles} These obstacles reflect the collapsed and unstable nature of the wreck. While not inherently dangerous, they can block progress or isolate party members if not navigated carefully.
\subparagraph*{Collapsed Stairwell} The stairs connecting the decks are broken and missing multiple steps. Characters must either climb up the broken framework using \textbf{Athletics (DC 13)} or \textbf{Acrobatics (DC 12) Check}, or find another way to get to the upper deck.
\subparagraph*{Dislodged Beam} A massive support beam has shifted and blocks a narrow hallway. It takes effort to move it safely aside. A character must succeed on a \textbf{DC 16 Strength (Athletics) Check}. Clever solutions involving teamwork or tools may reduce the DC.
\subparagraph*{Loose Rigging} Tangled ropes and nets hang from the ceiling and trail across the floor. They must be carefully navigated or cleared. A \textbf{DC 12 Dexterity Check} allows a character to slip through or cut them free. Failure may result in becoming restrained until freed.
\subparagraph*{Fractured Decking} Large, jagged holes have opened in the ship's deck. A successful \textbf{DC 13 Dexterity (Acrobatics) Check} clears the gap. Alternatively, a successful \textbf{DC 12 Investigation Check} might reveal a more stable path.
}

\subsection*{All On Deck}
{\entryfont Upon reaching the River Maiden's main deck, the party is met with a brief moment of calm, broken only by the wind whispering through the rigging and the rhythmic creak of the ship's worn timbers. With a successful \textbf{DC 16 Wisdom (Perception) Check}, a character can spot a small opening hidden beneath torn canvas and driftwood - just large enough for a Small or smaller creature to crawl through.

This tunnel leads to the juvenile Grung's hideout. If entered, the Grung will immediately scatter in a panicked flurry, abandoning their wooden playground and vanishing into the nooks of the wreck. Inside the hideout, the party will find any stolen items the Grung have taken, along with additional trinkets (up to 10 GP).

Nearby stands a reinforced door leading to the captain's quarters. It can be unlocked with either a \textbf{DC 17 Dexterity (Sleight of Hand) Check}, the key found in the Grung's hideout, or brute-forced open with a \textbf{DC 25 Strength (Athletics) Check}.}

\subsection*{Encounter: Captain Yarrface}
{\entryfont Beyond the creaking door lies a dim and dust-choked chamber, once regal but now rotted and sunken with age. The walls are hung with shredded naval banners, and the scent of salt and decay clings thick in the air. At the far end of the room sits an old, iron-banded chest, streaked with sea salt and green with mildew.

When a player attempts to open the chest, a sudden chill sweeps through the room. The ghostly figure of \hyperref[monster:CaptainYarrface]{\LinkFont{Captain Yarrface}} materializes from the air. With a shriek that echoes through the wreck, he attempts to possess the intruder.

At the same time, two \hyperref[monster:Shadow]{\LinkFont{Shadows}} rise from the corners of the room, their forms flickering like black sails in a storm.

\paragraph*{Combat} Captain Yarrface opens combat with Possession if possible, targeting the character who touched the chest. When Yarrface is reduced below 10 HP, he becomes ethereal and flees, vowing to one day reclaim his treasure.

\paragraph*{Reward} If the party successfully defeats the Shadows and drives off Yarrface, they are free to open the chest. Inside they can find the following loot:
\begin{itemize}
	\item 50 GP
	\item Breathing Bubble 
	\item Medal of the Conch
	\item Shield of the Turtoise (cursed)
\end{itemize}
}

\section*{Crimson Paw Bandits}\phantomsection\addcontentsline{toc}{section}{Crimson Paw Bandits}
{\entryfont The forest between the River Eden and Crail becomes increasingly twisted and difficult to navigate. As the players move deeper, they encounter multiple forks and ambiguous clearings, forcing them to make directional choices. Each fork can lead to either progress toward the Bandit Nest, a dead end, or a hazard/trap.

If the players have completed the River Maiden wreckage encounter earlier, the bandits will have since abandoned the camp and surrounding forest. Only faint traces of their presence remain - trampled brush, smouldering fire pits, and a bloody corpse half-covered in leaves. In the abandoned camp the party can still discover the \hyperref[resource:TatteredDirective]{\LinkFont{Tattered Directive}}.}

\subsection*{\inTextCircled[10]{1}{10} Starting Point}
\begin{DndReadAloud}
	As the trail narrows and the canopy thickens overhead, the air grows damp and heavy. As you follow the path through the ever-thickening woods, you spot something lying in the mud - small, pale, and matted with blood. Upon closer inspection, it's a severed rabbit's foot, soaked crimson and still fresh. No other remains are nearby. A hush settles over the forest as twisted trees loom ahead, their tangled roots and gnarled limbs forming a natural maze. The path splits in three directions - left, right, and straight ahead - each one vanishing into dense, shifting foliage.
\end{DndReadAloud}

\begin{tikzpicture}[remember picture, overlay]%
	\node at (0,0) {};
	\node[yshift=-0.1cm, anchor=south] at (current page.south) {\includegraphics[width=\paperwidth]{%
		images/Maps/Crimson_Paw_72DPI%
	}};%
	
	\begin{scope}[scale=0.25, xshift=-4cm, yshift=-56cm]
		\pgfmathsetmacro{\IndicatorSize}{7*\scaleFactor}%
		
		\indicatorNumberField[10]{(78,26)}{\IndicatorSize}
		\indicatorNumberField[10]{(67,49)}{\IndicatorSize}
		\indicatorNumberField[10]{(68.5,9)}{\IndicatorSize}
		\indicatorNumberField[10]{(65,5)}{\IndicatorSize}
		\indicatorNumberField[10]{(48,37)}{\IndicatorSize}
		\indicatorNumberField[10]{(37,5)}{\IndicatorSize}
		\indicatorNumberField[10]{(25.5,41)}{\IndicatorSize}
		\indicatorNumberField[10]{(6,27)}{\IndicatorSize}
	\end{scope}
\end{tikzpicture}%

\vfill\eject
\begingroup
	\DndSetThemeColor[PhbLightGreen]
	\begin{DndComment}{Rabbit's Foot}
		\textit{Wondrous Item, common (requires attunement)}\\
		Whenever you roll a 1 on an Ability Check, Saving Throw, or Attack Roll, you can re-roll the d20 and use the new result. Once this feature is used, the foot is no longer magical.
	\end{DndComment}
\endgroup
\subsection*{\inTextCircled[10]{2}{10} Hunter's Dread}
{\entryfont Strung between two trees like a crude wind chime hangs a cluster of small animal bones - birds, squirrels, and perhaps something larger. They sway gently in the breeze, clattering softly together. The arrangement is deliberate, but primitive.

Anyone who lays eyes on this grotesque construction must succeed on a \textbf{DC 13 Constitution Saving Throw}. On a failure, they have Disadvantage on Wisdom Saving Throws against being Frightened for the next hour.}
\subsection*{\inTextCircled[10]{3}{10} Hidden Treasure}
{\entryfont Tucked deep within a dense cluster of ferns and tangled roots, a moss-covered treasure chest lies half-buried in the soil. Time and weather have worn its iron fittings, but the sturdy lock remains intact. It can be opened with a \textbf{DC 14 Dexterity (Sleight of Hand) Check} using thieves' tools.
\begin{itemize}
	\item 2 Potion of Healing
	\item Quartz Gemstone (50 GP)
	\item Garnet Gemstone (100 GP)
\end{itemize}
}
\begin{tikzpicture}[remember picture, overlay]%
	\node[xshift=-0.25cm, yshift=-0.5cm, anchor=north east] at (current page.north east) {\includegraphics[width=2cm]{%
		images/Magic_Items/Rabbit_s_Foot%
	}};%
\end{tikzpicture}
\vfill\eject
\subsection*{\inTextCircled[10]{4}{10}/\inTextCircled[10]{5}{10} Carts of Teleportation}
{\entryfont Along two separate forest paths, the party encounters derelict merchant carts completely blocking the narrow way forward. Each is broken and overgrown - one wheel sunken into the mud, the other tilted against a leaning tree. Vines creep across their surfaces, but a faint shimmer distorts the air around them like a mirage. If a creature touches either cart, they are instantly teleported to the location of the other, vanishing in a flash of faint light. There's no obvious connection between the two, and no magical aura unless specifically detected. From the party's perspective, it may seem like they've been inexplicably spun in circles - unless they're paying very close attention.}
\subsection*{\inTextCircled[10]{6}{10} Destroyed Shrine}
{\entryfont In a small glade, half-buried in roots and brush, lie the crumbled remains of a once-sacred shrine. Vines creep over shattered stonework and scorched wooden carvings, and the faint smell of smoke still clings to the air.

If the party investigates the shrine directly, they trigger an ambush by two figures hidden in the treeline: a \hyperref[monster:BanditArcher]{\LinkFont{Bandit Archer}} and a \hyperref[monster:BanditDruid]{\LinkFont{Bandit Druid}}, determined to keep interlopers from interfering.
\begin{itemize}
	\item A successful \textbf{DC 14 Intelligence (History) Check} reveals that the shrine was destroyed relatively recently—within the past few weeks.
	\item A \textbf{DC 16 Intelligence (Religion) Check} identifies the shrine's purpose: it was once dedicated to Solonor Thelandira, elven god of hunting, archery, and wilderness.
\end{itemize}
If the party repairs the shrine - through a combined \textbf{DC 15 Religion} and/or appropriate \textbf{Tool Check} (Woodcarver's tools, mason's tools, or creative magic use) - they receive a small divine favour.}
\begingroup
	\DndSetThemeColor[PhbMauve]
	\begin{DndComment}{Favour of Solonor Thelandira}
		Once in the next 24 hours, each character may re-roll one Ranged Attack Roll or Wisdom (Perception or Survival) Check, choosing the better result. This boon manifests as a brief whisper of wind or a glimmer of golden light guiding their aim.
	\end{DndComment}
\endgroup
\subsection*{\inTextCircled[10]{7}{10} Dim Clearing}
{\entryfont The path opens into a narrow forest clearing, but the towering treetops above interlock so densely that they form a natural canopy, casting most of the ground in shadow. Only a few faint shafts of light pierce through the branches, forming pale, flickering circles on the blood-stained earth.

Near the edge of the glade, a young man in simple armour sits with his back pressed against a tree. He clutches a longsword in trembling hands, staring wide-eyed into a darker patch of the clearing where the underbrush seems unnaturally still. He does not acknowledge the party's approach, breathing in ragged gasps, frozen in place by some primal fear.

As soon as anyone speaks or reaches for him, a massive \hyperref[monster:ShadowMastiff]{\LinkFont{Alpha Shadow Mastiff}} lunges from the gloom with terrifying speed. The trainee has no time to react - he is torn down in an instant. The creature lifts its black-eyed gaze toward the party, a deep growl rumbling from its shadowed form, before it attacks.}
\subsection*{\inTextCircled[10]{8}{10} Bandit Camp}
{\entryfont
Hidden in the heart of the forest labyrinth lies the bandits' camp - a crude encampment of weathered tents, makeshift barricades, and scavenged crates. In the center of the camp stands a large, rusted cage containing a \hyperref[monster:OwlbearCub]{\LinkFont{Owlbear Cub}}, pacing anxiously. Surrounding the cage are four Snare Traps (DC 13), one on each side - hastily set and indiscriminate, just as likely to catch an unlucky bandit as a trespasser.

A \hyperref[monster:BanditBerserker]{\LinkFont{Bandit Berserker}}, already half drunk on bloodlust, and a calculating \hyperref[monster:BanditCaptain]{\LinkFont{Bandit Captain}}, calmly survey the perimeter. If the Shadow Mastiff's howl or the Bandit Druid's Animal Messenger from earlier encounters has triggered an alert, the bandits are already prepared and hot on the party's heels, awaiting their arrival at the camp.

Any surviving enemies not yet encountered - such as the \hyperref[monster:BanditArcher]{\LinkFont{Bandit Archer}}, \hyperref[monster:BanditDruid]{\LinkFont{Bandit Druid}}, or \hyperref[monster:ShadowMastiff]{\LinkFont{Shadow Mastiff}} - will arrive at the start of Turn 2, entering the fray as reinforcements.

During the battle, the owlbear cub can be freed. Once free, it panics and bolts into the surrounding woods, remaining hidden until the fight ends.

After the fight concludes the party can find the following items scattered around the camp:
\begin{itemize}
	\item Figurine of Wondrous Power (Silver Raven)
	\item Horn of Silent Alarms
	\item Pipe of Smoke Monsters
	\item Bottle of Boundless Coffee
\end{itemize}}
\begingroup
	\DndSetThemeColor[PhbLightGreen]
	\begin{DndComment}{Figurine of Wondrous Power (Silver Raven)}
		\textit{Wondrous Item, uncommon}\\
		This silver statuette of a raven can become a Raven for up to 12 hours. Once it has been used, it can't be used again until 2 days have passed. While in raven form, the figurine grants you the ability to cast Animal Messenger on it.
	\end{DndComment}
\endgroup
{\entryfont\noindent While searching the camp for loot the party will also find a \textbf{Tattered Directive}.}
\subsubsection*{Tattered Directive}\phantomsection\label{resource:TatteredDirective}
\begin{tikzpicture}
	\WrittenNote[]{\linewidth}{6}{note_parchment_background_horizontal.png}{%
		\hfill\\\\Squirrels and rats are not enough. They lack strength - and fear. Find me something more intimidating. Something fit for combat, not scavenging.\\\\
		- K
	}%
%	\handwrittenNote{\linewidth}{6}{note_parchment_background_horizontal.png}{%
%		\hfill\\\\Squirrels and rats are not enough. They lack strength - and fear. Find me something more intimidating. Something fit for combat, not scavenging.\\\\
%		- K
%	}%
\end{tikzpicture}
\hfill\\
{\entryfont\noindent After the battle, the cub cautiously re-emerges. If approached gently, it may be befriended with a successful \textbf{DC 17 Wisdom (Animal Handling) Check} or through other creative, peaceful means. If successful, the creature shows a quiet loyalty - carefully following the party during their travel to Crail, but keeping a considerable distance. It remains watchful from the tree line, never approaching too closely, yet always nearby.}

{\centering\contourlength{0.05em}\Large\contour{black}{\textcolor{titlegold}{\textbf{\textsc{Level-Up}}}}\\}