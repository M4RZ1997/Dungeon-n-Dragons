\chapter*{Wedding Ceremony}\stepcounter{chapter}\phantomsection\addcontentsline{toc}{chapter}{Wedding Ceremony}
\DndDropCapLine{W}{\entryfont ith the Grand Joust concluded and the dust of the arena settled, the realm turns its eyes toward the Wedding Ceremony. Nobles and warriors, emissaries and jesters, all gather beneath the soaring banners of the Kingdom of Fife, bearing witness to the sacred union that will shape the future of the land.}

{\entryfont The air is filled with the soft melodies of minstrels, the quiet murmur of conversation, and the distant toll of ceremonial bells. The feast is prepared, the finest ales poured, and every detail has been arranged to honour this momentous occasion. As the ceremony begins, the weight of history settles upon those in attendance - this is more than a union of two souls; it is a bond that will shape the future of the realm.}

\subsection*{Lost Bet}
{\entryfont Should a player have lost their bet with Ser Proletius their fate is now inescapable. The grandmaster, ever the showman, seizes the moment, throwing a jubilant arm around the unlucky soul and pulling them toward the center of the stage.}
\begin{DndReadAloud}
	\textit{"A debt is a debt, my friend!"} Proletius declares, his voice booming over the gathered guests. \textit{"And tonight, we shall hear your voice in all its glory!"}

	With a grand gesture, he signals the minstrels, who immediately begin playing the first triumphant notes of \textbf{The Anthem of Crail}. The crowd erupts in cheers, awaiting your performance - whether you sing with the heart of a true Knight of Crail, fumble awkwardly through forgotten lyrics, or attempt some last-ditch trickery to escape your fate.
\end{DndReadAloud}

\section*{Dwarven Keg of Chaos}\phantomsection\addcontentsline{toc}{section}{Dwarven Keg of Chaos}
{\entryfont A relic of mirth and mystery, the Dwarven Keg of Arcane Chaos is a legendary wedding tradition among the dwarves of the Mines of Methven, nestled deep within the western mountains of the Kingdom of Fife. Said to have been crafted by the Brewlords of Old, the keg is not bound by mortal hands - instead, it is infused with the raw, unpredictable essence of arcane fermentation.

The Methven dwarves believe that drinking from the keg is a blessing of fortune and folly alike - a way for fate to weave its hand into the revelry of a grand occasion. No two drinks are ever the same, and those who partake may find themselves endowed with temporary gifts, burdened by absurd curses, or simply confused as to why they now possess a perfectly baked meat pie.

It is tradition among the dwarves that the keg be brought forth at grand unions and feasts of great joy, for a wedding without chaos is a wedding doomed to boredom. Legends claim the first Grandmaster of Crail himself drank from its frothy depths and spent an entire evening levitating uncontrollably while composing a ballad about cheese wheels - a song still sung in dwarven halls to this day.}
\begin{tikzpicture}[remember picture, overlay]%
	\node[xshift=0.25cm, yshift=-0.25cm, anchor=south east] at (current page.south east) {\includegraphics[width=.5\paperwidth]{%
		images/Others/Pocket_Meat_Pie%
	}};%
\end{tikzpicture}%
\begin{DndTable}[header=Magical Effects]{cX}
	1d10	& Effect \\
	1 		& \textbf{Beard of Glory}\newline You grow a magnificent beard, regardless of gender. For the next hour you have advantage on Charisma (Persuasion) rolls.\\
	2		& \textbf{Tongue of the Brewlord}\newline For the next 5 minutes you can only speak, read and understand Dwarven. (Primordial if you already can speak and understand Dwarven)\\
	3		& \textbf{The Floor is Lava!}\newline You think that the whole floor is lava. You have to jump from object to object to move through the room. If you touch the floor in any way, you take \DndDice{1d4} Psychic damage. This effect persists for 1 minute or until you take damage.\\
	4		& \textbf{Mithral Stomach}\newline For 1 hour you are immune to poison - even alcoholic poisoning - and can eat anything.\\
	5		& \textbf{Jolly Jig}\newline A random dwarven drinking song fills your mind. You must make a DC 12 Wisdom Saving Throw or dance and sing for the next minute.\\
	6		& \textbf{Echoing Belch}\newline Your next burp is so loud that it can be heard 300 feet away. Small or tiny creatures that can hear the burp must succeed on a DC 15 Wisdom Saving Throw or are frightened for 1 minute.\\
	7		& \textbf{Dwarven Gourmet}\newline For the next 10 minutes you have developed an absolute love for the dwarven cuisine. You may demand for ale-soaked mushrooms, stone-bread, or lava-boiled snails for the duration of this effect.\\
	8		& \textbf{Mysterious Pocket Snack}\newline You find a still warm meat pie (\DndDice{2d4 + 2} Temporary Hit Points when eaten) in your pocket. How did it get there? It smells delicious.\\
	9		& \textbf{Bard's Curse}\newline For the next 5 minutes whenever you try to speak, you instead sing your words in a dramatic ballad-like fashion.\\
	10		& \textbf{Blessing of the Brewlords}\newline A faint golden glow surrounds you. For the next hour you have advantage on Charisma Checks when dealing with dwarves. Also, dwarves will offer you free drinks.
\end{DndTable}

\vfill\eject

\section*{Seer's Confection}\phantomsection\addcontentsline{toc}{section}{Seer's Confection}
{\entryfont At the heart of the wedding feast, standing upon a pedestal of ornate silver and enchanted stone, rests a cake of mysterious origin and whispered legend. Though no one can say exactly where it came from, it is known to appear only at the most significant unions in history - always present, yet never explained. Some believe it to be the work of the Cairngorm Wizards, the enigmatic spellcasters of the western reaches, while others insist it is a creation of fate itself, woven from threads of time and possibility.

The nobility refer to it in hushed tones as \textit{"The Seer's Confection"}, a name that carries both awe and caution. The legends claim that those who partake will experience visions of their destiny, glimpsing possible futures - some glorious, some tragic, some utterly incomprehensible. However, fate does not reveal itself lightly, and the cake's magic is not without risk.}
\begin{DndTable}{cX}
	1d100	& Vision/Effect \\
	1 		& You fall into a \textit{Destiny Coma}. When the player is woken up he speaks the words of \textbf{"Anstruther's Dark Prophecy"}, but cannot recollect the words or the reason why afterwards.\\
	2-20	& \textit{Destiny Coma}\\
	21-40	& \textit{Hammer of Glory Quest}\\
	41-75	& \textit{In the far corner of the Braided Unicorn Tavern, half-hidden beneath a worn, dust-covered rug, you glimpse the outline of a trapdoor - its edges marked by age and secrecy. A faint draft of cold, earthy air whispers from its seams, hinting at a tunnel descending into darkness.}\\
	76-90	& \textit{Pool/Prison} of Liquid Ice\\
	91-99	& You have a vision of three items: a Battlehammer, an Amulet, and a Knife.\\
	100		& You speak the words of \textbf{"Anstruther's Dark Prophecy"}.
\end{DndTable}
\subsection*{The Destiny Coma}
{\entryfont For some, the sheer magnitude of the destinies they witness is too much to bear. Their minds become lost in the flood of possibility, their bodies collapsing into unconsciousness as they struggle to grasp what they have seen. This state, known as the Destiny Coma, has struck down kings, knights, and scholars alike. While most awaken quickly with aid, some never return at all, their minds forever lost in the depths of fate's tapestry.

Should one succumb to the Destiny Coma, they experience the following effect:}
\begingroup
	\DndSetThemeColor[PhbMauve]
	\begin{DndComment}{Destiny Coma}
		\textit{You are overwhelmed by the many destinies you see and fall unconscious, taking \DndDice{1d6} Psychic damage. You can only be woken up by a healing spell/potion - a Goodberry is sufficient - or you are hit after 1 minute of being unconscious, taking at least 1 bludgeoning damage.}
	\end{DndComment}
\endgroup

\vfill\eject

\section*{Floating Goblets}\phantomsection\addcontentsline{toc}{section}{Floating Goblets}
{\entryfont Tucked into a shadowed corner of the bustling grand plaza, a small table draws a lively crowd. Laughter, cheers, and playful jeers rise above the din, as people gather around what appears to be a simple game - yet there's a spark of excitement, and perhaps mischief, in the air.

The "stall" is hosted by the charismatic Lady Belissa and the quick-tongued Sir Alrik the Swift, both commanding the crowd with practised flair and contagious energy. However, beneath their convincing appearances, these two are in fact the notorious Breeza and Arlen in masterful disguise.

Only a successful DC 30 Wisdom (Insight) Check reveals subtle clues - a familiar piece of jewelry worn by "Belissa", or a fleeting speech quirk in "Alrik's" banter - hinting at their true identities beneath the disguises. To everyone else, they remain just another pair of colourful revellers adding their own brand of excitement to the festivities.}

\subsection*{The Hobgoblet Shuffleboard}
{\entryfont Players each take turns sliding a goblet down the polished table, aiming for marked scoring zones at the far end. The goal is to accumulate the highest total score over three rounds.}

\begin{DndTable}[header=Shuffleboard Scoring]{lXX}
\textbf{Attack Roll}		& \textbf{Location}		& \textbf{Points}		\\
less than 8				& Too Short				& 0						\\
8+						& First Zone				& 1						\\
12+						& Second Zone			& 2						\\
16+						& Third Zone				& 3						\\
NAT20					& Right on the Edge		& 3	- chance to cheat	\\
20+						& Too Long				& 0						\\
\end{DndTable}

{\entryfont \paragraph*{Cheating} The hosts still cannot move away from their cheating ways and again created a sophisticated cheating mechanism to turn the odds in their favour. The table can be tilted by Belissa to ensure Alrik's goblet either moves further into a higher scoring zone or not falling from the table and stopping right at the end.}

\vfill\clearpage

\section*{Encounters}\phantomsection\addcontentsline{toc}{section}{Encounters}
{\entryfont As the ceremony turns to celebration, the wedding feast offers a rare chance to speak with key figures of great importance. Conversations held tonight may reveal hidden truths, forge alliances, or stir tensions yet unseen.}

\subsection*{Prince Angus McFife}
{\entryfont On this day, Prince Angus McFife is the happiest man in the world. His love for Iona McDougall is evident in every word he speaks, every glance he steals in her direction. He radiates joy, pride, and an unwavering belief in the future, knowing that their union will bring peace and prosperity to the kingdom.

Angus is genuinely pleased to see so many people from all corners of Fife gathered for the festivities. Unlike many nobles, he is deeply interested in the lives of others, eagerly asking the party about their backgrounds, their adventures, and what brought them to this moment.}
\subsubsection*{Stance on the Highland Mysteries}
{\entryfont Angus does not dismiss the danger, knowing that people - real people - have died or gone missing, even if the stories themselves are exaggerated or wrapped in legend. What frustrates him most is his father's reluctance to act, as King Dundax \RoyalRoman{XIII} sees the tales as nothing more than superstitious nonsense.

Angus knows arguing with his father is futile, but that does not mean he does nothing. If he cannot fight the cause directly, he will fight for and support the families left behind.}
\subsection*{King Dundax \RoyalRoman{XIII}}
{\entryfont The esteemed ruler of the Kingdom of Fife, King Dundax \RoyalRoman{XIII}, welcomes conversation with a regal yet measured presence. He speaks at length about the history of the kingdom, its alliances, and the broader landscape of power across the land. His focus remains on matters of diplomacy, legacy, and the future of Fife, offering insights into the realm's political state and its standing among neighbouring territories.}
\subsubsection*{Stance on the Highland Mysteries}
{\entryfont When the topic of strange sightings in the Highlands arises he reacts with mild amusement and scepticism. Dismissing them as superstitious nonsense, he considers such tales to be nothing more than the fancies of fearful peasants or exaggerations from overzealous hunters. He expresses no real concern, seeing no reason to divert attention from more pressing political matters.}
\subsection*{Ewan MacRae of Dunkeld}
{\entryfont If a party member faced Ewan MacRae in the Grand Joust, the stalwart protector of Dunkeld greets them with genuine admiration, praising their skill and courage in battle. He remarks that it's rare to meet someone who can truly hold their own, offering an invitation to share a drink in good company.

Ewan proves to be a welcoming and honourable man, eager to speak of his beloved city of Dunkeld and the responsibility he bears in protecting its people. He takes great pride in his duty, ensuring that the city remains safe from both external threats and the dangers of the wilds beyond its borders.}
\subsubsection*{Stance on the Highland Mysteries}
{\entryfont When asked about the mysteries of the Highlands, Ewan's expression darkens for a moment as he falls into deep thought. In a quieter voice, he admits that the hunters of Dunkeld have spoken to him about strange occurrences in the mountains and forests that lie beyond the kingdom's heartlands. Their stories are too detailed, too consistent, to be dismissed as mere folklore.

Yet, Ewan is no fool - he knows that openly speaking about such matters would invite scepticism and ridicule. Instead, he keeps his concerns private, though he remains curious and watchful, eager to uncover the truth behind whatever truly lurks in the Highlands.}
\subsection*{Princess Iona McDougall}
{\entryfont \paragraph*{Bobo was rescued} If the party successfully rescued Bobo, Iona McDougall seeks them out personally, her usual composed demeanour replaced by genuine gratitude. She expresses her deep appreciation for their actions and rewards them with \textbf{a magical item}, ensuring they know how much their deed means - not just to Prince Angus McFife, but to her as well.}
{\entryfont \paragraph*{Bobo is still missing}
If the party failed to rescue Bobo, her tone is far more solemn. She informs them that while Angus McFife does not outwardly show his grief, she can see the weight of his sorrow beneath the surface. The loss has affected him deeply, and though he will never speak of it, his heart is heavy.}
\subsubsection*{Stance on the Highland Mysteries}
{\entryfont Iona is concerned not for the strange sightings themselves, but for the people who have vanished - hunters, adventurers, and those who set out never to return. She worries for the families left behind, the unexplained absences, and what may be lurking in the Highlands. However, as for the stories of "rabid" unicorns and other legends, she remains largely indifferent, dismissing them as embellishments on a real and more troubling reality.}
\subsection*{Grandmaster Ser Proletius}
{\entryfont A conversation with Ser Proletius is never a quiet affair. The Grandmaster of the Knights of Crail speaks with unshakable pride about his order, boasting of their glorious deeds, unwavering honour, and unmatched skill in battle. He never misses an opportunity to praise the Knights of Crail, often weaving grand tales of past victories - sometimes exaggerated, sometimes entirely true. Despite his loud and boastful nature, he is not without humour, readily laughing at a well-placed jest and responding in kind with his own repertoire of knightly jokes and tales.}
\subsubsection*{Stance on the Highland Mysteries}
{\entryfont If there is a threat to the realm, Ser Proletius swears he would hunt it down and strike it down himself, should he ever uncover the truth. However, when it comes to the tales of "rabid" unicorns and other Highland legends, he finds himself torn between scepticism and caution. While he doesn't dismiss the rumours outright, he is equally unwilling to waste time chasing after ghost stories. If a true danger exists, he believes it will reveal itself soon enough - and when it does, he will be the one to put an end to it.}