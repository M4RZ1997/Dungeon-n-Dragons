\chapter*{Wedding Fair}\stepcounter{chapter}\phantomsection\addcontentsline{toc}{chapter}{Wedding Fair}
\DndDropCapLine{B}{\entryfont ustling with life, Dundee's wedding fairgrounds overflow with color and sound, drawing crowds into a joyous celebration that spills into nearby streets. Bright banners ripple in the breeze, lanterns glow warmly, and the tantalizing scents of roasted meats, honeyed shortbread, and mulled cider drift through the air. At the center stands a grand oak tree, its branches adorned with ribbons and lights, casting shade over the lively gathering. Roaming the grounds is the festival's famed dragon, a marvel of craftsmanship with snapping jaws, fluttering wings, and bursts of theatrical fire that delight festival-goers. Stalls brim with Highland treats, fine tartans, and handmade goods, while performers and musicians keep the festivities alive with juggling, dancing, and lively tunes.}

\begin{DndReadAloud}
	You find yourself immersed in the sights and sounds of the wedding fair. Music from fiddles and bagpipes fills the air, weaving with laughter and the chatter of excited festival-goers. Children race between stalls, clutching candied apples and pointing in delight at jugglers tossing flaming pins. The tantalizing aroma of roasted meats and sweet shortbread tempts you at every turn.

	Nearby, a massive oak tree draped in ribbons and lanterns serves as the heart of the fair. The crowd's attention shifts as the festival's dragon - a towering, lifelike marvel of wood and metal - stirs. Its wings ripple and its jaws snap with dramatic flair eliciting gasps and cheers from the crowd.

	As the sun dips below the horizon, lanterns and torches bathe the fairgrounds in a golden glow. Laughter and awe ripple through the throng as the magic of the festival fully takes hold, drawing you deeper into its spellbinding atmosphere.
\end{DndReadAloud}

\section*{Dragon's Gold}\phantomsection\addcontentsline{toc}{section}{Dragon's Gold}
\textbf{Entry Fee:} 1 GP\\
{\entryfont The Dragon's Gold is the crown jewel of the festival, a dazzling and immersive event that draws crowds of eager participants and onlookers to the central plaza. The centrepiece of this spectacle is the magnificent dragon itself, a towering construct of painted paper, wood, and metal. Adorned with shimmering scales in vibrant reds, greens, and golds, the dragon moves with startling realism. Clever mechanisms allow its angular head to snap and jerk, its wooden wings to flutter with a show of might, and its mouth to "breathe fire" in the form of colourful ribbons and occasional bursts of magical smoke. The creature prowls menacingly near its sprawling treasure hoard, exuding an air of majesty and danger that captivates the entire square.}

\subsection*{Mechanics}
{\entryfont The Dragon's Gold game takes place in a 50-foot radius playing area, covered in thick layers of hay to conceal scattered treasures. At the heart of the area lies the 15-foot radius Inner Hoard, where the most coveted prizes are clustered. However, this inner zone is guarded by the festival dragon, which relentlessly defends its treasures.}

\begin{DndMonster}[width=0.5\textwidth]{Festival Dragon}
	\DndMonsterType{Large Construct, neutral good}
	
	% If you want to use commas in the key values, enclose the values in braces.
	\DndMonsterBasics[
		armor-class = {16},
		hit-points  = {\DndDice{9d10 + 36}},
		speed       = {20 ft.},
	]
	
	\renewcommand{\AbilityScoreSpacer}{~}
	\DndMonsterAbilityScores[
		str = 7,
		dex = 7,
		con = 18,
		int = 8,
		wis = 18,
		cha = 20,
	]
	
	\DndMonsterDetails[
		saving-throws = {CON +8, CHA +9},
		skills = {Insight +8, Intimidation +9, Perception +8},
		%damage-vulnerabilities = {cold},
		%damage-resistances = {bludgeoning, piercing, and slashing from nonmagical attacks},
		%damage-immunities = {poison},
		senses = {Passive Perception 18},
		%condition-immunities = {poisoned},
		languages = {-},
		challenge = -,
		proficiency-bonus = +4,
	]
	
	\DndMonsterSection{Traits}
    \DndMonsterAction{Operative Mastery}
    The Festival Dragon must be operated by a group of at least 6 people. If the Festival Dragon is operated by a properly trained performing team, its AC is increased by 2 and gains a bonus of +4 to each melee attack and damage roll.
    
	\DndMonsterAction{Non-Lethal Attacks}
	Each attack the Festival Dragon makes is considered non-lethal
	
	\DndMonsterAction{Friendly Nature}
    The Festival Dragon is usually friendly towards children and only chases them away, unless they get too greedy.
	
	\DndMonsterSection{Actions}
	\DndMonsterAction{Multiattack}
	The Festival Dragon makes one Claw Attack and one Bite or Tail Attack.
	
	\DndMonsterAttack[
		name=Claw,
		distance=melee, % valid options are in the set {both,melee,ranged},
		%type=weapon, %valid options are in the set {weapon,spell}
		mod=-2,
		reach=5,
		%range=20/60,
		targets=one target,
		dmg=\DndDice{1d4 - 2},
		dmg-type=slashing,
		%plus-dmg=\DndDice{1d6},
		%plus-dmg-type=poison,
		%or-dmg=,
		%or-dmg-when=,
		extra={ (minimum of 1 damage)},
	]
	
	\DndMonsterAttack[
		name=Bite,
		distance=melee, % valid options are in the set {both,melee,ranged},
		%type=weapon, %valid options are in the set {weapon,spell}
		mod=-2,
		reach=5,
		%range=20/60,
		targets=one target,
		dmg=\DndDice{1d6 - 2},
		dmg-type=piercing,
		%plus-dmg=\DndDice{1d6},
		%plus-dmg-type=poison,
		%or-dmg=,
		%or-dmg-when=,
		extra={ (minimum of 1 damage)},
	]
	
	\DndMonsterAttack[
		name=Tail,
		distance=melee, % valid options are in the set {both,melee,ranged},
		%type=weapon, %valid options are in the set {weapon,spell}
		mod=-2,
		reach=20,
		%range=20/60,
		targets=one target,
		dmg=\DndDice{1d6 - 2},
		dmg-type=bludgeoning,
		%plus-dmg=\DndDice{1d6},
		%plus-dmg-type=poison,
		%or-dmg=,
		%or-dmg-when=,
		extra={ (minimum of 1 damage)},
	]
	
	\DndMonsterAction{Breath Weapon (Recharge 5 Rounds)}
	Each creature in a 30ft. long, 5 ft. wide line must make a DC16 Dexterity Saving Throw or are instantly out of the Dragon's Gold game.
	
	\DndMonsterSection{Reactions}
	\DndMonsterAction{Automated Construct}
	The Festival Dragon can make one attack of opportunity with each of its front limbs, mouth, and tail.
\end{DndMonster}

{\entryfont\noindent Participants enter the game unarmed and unarmoured, with no equipment allowed. The use of magic is strictly prohibited, and contestants may not harm or attack one another under any circumstances, preserving the festive nature of the game.

Each round lasts 2 minutes (20 rounds in total). Contestants have this time to search for and claim as much treasure as they can manage.

To locate treasures, contestants may perform one of the following actions:
\begin{itemize}
	\renewcommand\labelitemi{\textbf{\textbullet}}
	\item \textbf{Spot Check (Perception):} Roll against a DC20 Check to locate a treasure anywhere in the play area (DC15 if searching within the Inner Hoard).
	\item \textbf{Search Check (Perception):} Roll against a DC10 Check to locate a treasure in a specific 5-foot square. Searching in this way within the dragon's reach provokes an Attack of Opportunity from the dragon.
\end{itemize}
If a contestant succeeds on either check, they can roll on the Treasure Table to determine the item found.

Contestants may continue to search and gather treasure for as long as they wish, carrying as many items as they can manage. However, treasures are forfeited if a contestant is struck by the dragon's attack and "killed". To remain in the spirit of the game, "dead" contestants are expected to leave the game when they are struck and may not interact with the hoard further, though they can remain in place as casualties for the remainder of the game.}

{\entryfont\noindent Contestants may leave the treasure hoard at any time, taking their gathered loot with them. Once a contestant leaves the playing area, they cannot re-enter. Leaving the area ends their participation in the game.\\\\
The balance of greed and caution defines the game: contestants must weigh the temptation of more treasures against the risk of provoking the dragon's wrath. The most daring and resourceful participants will walk away with riches, but those who overreach may lose everything!}

\begin{DndTable}[header=Treasure Table]{lXX}
\textbf{d100}	& \textbf{Outer Hoard (value)}							& \textbf{Inner Hoard (value)}	\\
01-10			& A small mound of 3d6 cp								& A mound of 5d6 sp				\\
11-20			& A shell bracelet (2 sp)								& Costume jewellery (1 gp)		\\
21-30			& A bag filled with coloured stones and marbles (5sp)	& A finely article of clothing, tied with ribbon (5 gp) \\
31-40			& A wreath of flowers and ribbons (2 sp)				& A large box that holds many rich candies (2gp) \\
41-50			& An embroidered cloth with a picture (1 gp)			& Bronze jewellery set with semi-precious stones (5gp) \\
51-60			& A wooden token good for a free meal at a local restaurant (5sp)	& A dagger or short-sword in sheath (10 gp) \\
61-70			& A toy sword painted silver (2 sp)						& A metal flask filled with spirits (2 gp) \\
71-80			& A gold coin (1 gp)									& A pretty doll or toy (5 gp) \\
81-90			& Some type of art supply, like inks, pens, fancy wood, or paint (3 gp)	& A common item made of silver or gold, like a needle (10 gp) \\
91-100			& Roll on the Inner Hoard table at +10					& Fancy jewellery or clothing (50 gp)
\end{DndTable}

\begin{DndOptionalRule}{Attack the Dragon}
	The Festival Dragon can be attacked during the Dragon's Gold Game with weapons and weapon-like items that are found during the game. However, it is not allowed to use any kind of magic to damage the construct.
\end{DndOptionalRule}

\section*{Tree Game}\phantomsection\addcontentsline{toc}{section}{Tree Game}
\textbf{Entry Fee:} 1 GP\\
{\entryfont The Tree Game is a lively and challenging test of accuracy and skill, where contestants attempt to claim prizes nestled among the high branches of a towering tree. Using arrows, handaxes, or throwing knives, competitors aim for trinkets, baubles, and even valuable treasures carefully lodged within the tree's limbs. Each well-placed strike sends a prize tumbling gracefully to the ground, greeted by the crowd's cheers - or the envious gazes of onlookers hoping for their turn.}

\begin{DndReadAloud}
	Before you stands a majestic oak, its sprawling branches adorned with glittering prizes that catch the light like stars in the daylight. Trinkets dangle from ribbons and strings, while more valuable treasures are wedged tightly into crooks and knots high above. The lower branches sway gently with small, easy-to-reach prizes, while the loftiest rewards glint enticingly just out of easy range. Contestants gather at the base of the tree, readying their arrows, handaxes, or knives as the crowd murmurs in anticipation.

	"Step forward and take your shot!" calls the event's master of ceremonies. "What will you claim from the tree's bounty - luck, skill, or sheer determination?"

	The air hums with excitement as the first competitor lines up their throw, aiming to send fortune falling from the great tree's branches.
\end{DndReadAloud}

\subsection*{Mechanics}
{\entryfont In the Tree Game, contestants are tasked with dislodging prizes tied to the branches of the grand festival tree using borrowed weapons provided by the game organizers. Each participant may choose either a shortbow with 5 arrows or two handaxes, both of which must be returned at the end of the game. Over the course of five rounds, contestants attempt to target and knock down the dangling prizes, each taking a single shot or throw per round.

Using the shortbow requires precise aim, as contestants must hit the harder AC (standard-AC +2) listed in the Prize Table to sever the ribbons holding the prizes in place. Conversely, handaxes rely on brute force, allowing contestants to attack against the standard AC of the target. Regardless of the weapon used, any prize that is successfully dislodged and falls to the ground is immediately claimed by the contestant.}

\begin{DndTable}[header=Prize Table]{llX}
\textbf{Height}	& \textbf{Effective AC}			& \textbf{Prize}												\\
10-24 ft		& 11 (13)						& Dart, Bucket, Piton, Signal Whistle							\\
25-39 ft		& 13 (15)						& Ladder, Sling Bullets (40), Sling								\\
40-44 ft		& 15 (17)						& Javelin, Lamp, Blanket, Sealing wax							\\
45-49 ft		& 18 (20)						& Arrows (20), Crossbow Bolts (20), Hammer, Caltrops (bag of 20)\\
50 ft +			& 18 (20)						& Padded Armour, Handaxe, Pike, Healer's kit					\\
\end{DndTable}

\begin{DndOptionalRule}{Losing Thrown Item}
	Critical Failure (NAT 1) leads to the thrown item to be stuck in the tree as well, and needs to be retrieved/dislodged in a similar manner like the prizes. Otherwise it will be lost to the game.
	
	For each borrowed hand axe lost during the game the contestant has to pay 10GP (40GP for a lost shortbow).
\end{DndOptionalRule}