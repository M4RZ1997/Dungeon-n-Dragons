\chapter*{Wedding Tourney}\stepcounter{chapter}\phantomsection\addcontentsline{toc}{chapter}{Wedding Tourney}
\DndDropCapLine{T}{\entryfont he Wedding Tourney celebrating the union of Angus McFife I and Iona McDougall has brought an air of excitement and festivity to Dundee. The grand event unfolds with a series of thrilling competitions, drawing skilled participants and eager spectators from across the land. The day's revelry begins with an archery contest and progresses through feats of strength and skill, including the long throw and the challenging log chop. Competitors earn points in each event based on their finishing positions, with their accumulated scores determining the seeding for the climactic jousting competition. In this final contest of valour and chivalry, the ultimate winner of the tourney will be crowned, marking the joyous celebration's triumphant conclusion. The point-scoring\linebreak}

\vspace*{-4.6\fontdimen6\font}\hfill\\\begin{multicols}{2}
{\entryfont \noindent system for the Wedding Tourney is designed to reward consistent performance across the Archery, Long Throw, and Log Chopping competitions. Competitors earn points based on their\linebreak}
\vspace*{-2.2em}\begin{DndTable}[header=Point Scoring System]{lXr}
\textbf{Rank}	&& \textbf{Points}	\\
1st				&& 10				\\
2nd				&& 8				\\
3rd				&& 6				\\
4th				&& 4				\\
5th				&& 2				\\
6th				&& 1				\\
\end{DndTable}\\
\end{multicols}
\vspace*{-4.3\fontdimen6\font}\hfill\\

{\noindent\entryfont finishing positions in each event, with the highest total scores determining their seeding for the jousting competition.}

\section*{Bullseye's Glory}\phantomsection\addcontentsline{toc}{section}{Bullseye's Glory}
{\entryfont The archery competition marks the opening event of the Wedding Tourney, testing the precision and steady hands of each competitor. Standing 80 feet away from a stationary target, each archer must demonstrate their mastery over the shortbow, aiming true to score as many points as possible. Crowds gather to watch as the competitors take their shots one by one, the rhythmic twang of bowstrings ringing through the air as arrows fly toward the bullseye. This contest of skill requires calm focus and careful aim, setting the tone for the challenging events to follow.}

\subsection*{Mechanics}
{\entryfont In the archery competition, each contestant has four shots to score points based on their aim and accuracy. To make each shot, competitors roll an attack against the target, with their point total determined by the roll's outcome, as detailed in the provided Scoring Table. All competitors must use an unenchanted shortbow, provided by the tourney organizers, to ensure fair play; any archer found using enchanted equipment will be immediately disqualified.}
\begin{DndTable}[header=Archery Scoring]{lXX}
\textbf{Attack Roll}	& \textbf{Location}		& \textbf{Points}	\\
less than 8				& Miss					& 0					\\
8+						& Outer Ring			& 1					\\
11+						& Middle Ring			& 3					\\
14+						& Inner Ring			& 5					\\
17+						& Bullseye				& 10				\\
NAT20					& Split another Arrow	& 10 + \textit{Crowd Roar*}	\\
\end{DndTable}
{\footnotesize * \textit{Crowd Roar} gives a +2 to the next attack roll in this competition.}
\vfill\eject
\begin{tikzpicture}[remember picture, overlay]%
	\node[xshift=0.25cm, yshift=0.25cm, anchor=north east] at (current page.north east) {\includegraphics[width=.6\paperwidth]{%
		images/Landscape/Archery_Competition%
	}};%
\end{tikzpicture}%
\hfill\\ \\ \\
\begin{DndTable}[header=Archery Ranking]{lXl}
\textbf{Rank}	& \textbf{Name}					& \textbf{Points}	\\
1st				& Ewan MacRae of Dunkeld		& 28				\\
2nd				& Gavin Buchanan				& 19				\\
3rd				& Rory MacTavish				& 14				\\
4th				& Alasdair MacLeod				& 8					\\
5th				& Hamish "Halfwit" McGregor		& 1					\\
\end{DndTable}

\section*{Hurl of Might}\phantomsection\addcontentsline{toc}{section}{Hurl of Might}
{\entryfont The Long Throw Competition is a test of raw power, technique, and endurance, as competitors attempt to throw a series of progressively heavier and more unwieldy objects as far as possible. Each object presents a unique challenge, requiring not only physical strength but also skillful control to reach impressive distances. Spectators gather eagerly along the marked field, cheering as each competitor heaves their object through the air, striving for the farthest distance in each throw. This contest celebrates feats of might and mastery, with each participant vying to surpass their rivals in sheer throwing range.}

\subsection*{Mechanics}
{\entryfont In the Long Throw Competition, contestants must throw four distinct objects, each with its own weight and difficulty, as outlined in the provided table. Each competitor has two attempts with each object, and only the best throw for each object counts toward their total score. Each throw requires a DC 10 check to clear the first range increment, and every additional 4 points above the DC increases the throw's range by one increment, as defined in the table. The contestant can use specific skills (also noted in the table) to attempt each throw. If the object's weight exceeds twice the contestant's Strength Score in pounds, they must roll with disadvantage due to the strain of hefting the object. The contestant's gain points for each distance increment, with the competitor achieving the greatest total declared the victor of the Long Throw Competition.}

\begin{DndTable}[header=Thrown Objects]{lllX}
\textbf{Object}	& \textbf{Range}				& \textbf{Weight}		& \textbf{Skill(s)}					\\
Javelin			& 30 ft.						& 2 lbs					& STR								\\
Discus			& 20 ft.						& 5 lbs					& Athletics (STR), Acrobatics (DEX)	\\
Stone			& 10 ft.						& 18 lbs				& STR, DEX							\\
Halfling		& 5 ft.							& 30 lbs				& STR								\\
\end{DndTable}

\begin{DndTable}[header=Long Throw Ranking]{lXl}
\textbf{Rank}	& \textbf{Name}					& \textbf{Points}	\\
1st				& Ewan MacRae of Dunkeld		& 13				\\
2nd				& Alasdair MacLeod				& 10				\\
3rd				& Rory MacTavish				& 6					\\
4th				& Gavin Buchanan				& 4					\\
5th				& Hamish "Halfwit" McGregor		& 2					\\
\end{DndTable}

\section*{Timber Trial}\phantomsection\addcontentsline{toc}{section}{Timber Trial}
{\entryfont The Log Chopping competition is a thrilling display of strength, stamina, and quick decision-making, as contestants race against the clock to chop through as many logs as they can in a single minute. With logs standing at regular intervals, participants must strike quickly and move swiftly to cover the ground between each target, chopping one after another in a relentless rhythm. Spectators cheer on as wood chips fly and axes swing, creating an intense spectacle of determination and power. The choice of weapon might play a key role, as each contestant must decide between speed and force, adjusting their approach to maximize their tally. Only the most determined and strategic log chopper will claim victory in this demanding event.}

\subsection*{Mechanics}
{\entryfont In the Log Chopping competition, contestants have one minute (10 rounds) to destroy as many logs as possible. Each log has an AC of 15 and 18 HP, challenging competitors to balance force and accuracy with each swing. The logs are spaced 10 feet apart, requiring contestants to move to the next log after each one is destroyed. At the start of the competition, contestants select a weapon from the provided weapon rack, which holds 2 \textbf{Handaxes} (1d6 slashing, light), a \textbf{Battleaxe} (1d8 (1d10 versatile) slashing), and a \textbf{Greataxe} (1d12 slashing, heavy, two-handed). They may switch weapons during the contest, but doing so requires a full round to return to the rack and select a new weapon. The contestant who destroys the most logs within the allotted time is declared the winner, with any ties broken by the lowest HP remaining on the last partially damaged log.}

\begin{DndTable}[header=Log Chopping Ranking]{lXl}
\textbf{Rank}	& \textbf{Name}					& \textbf{Points}	\\
1st				& Hamish "Halfwit" McGregor		& 10*				\\
2nd				& Ewan MacRae of Dunkeld		& 5 (18 HP)			\\
3rd				& Alasdair MacLeod				& 3 (12 HP)			\\
4th				& Rory MacTavish				& 2	(6 HP)			\\
5th				& Gavin Buchanan				& 1 (12 HP)			\\
\end{DndTable}
{\footnotesize * Hamish has used an illegal Adamantine Axe for this challenge and will be disqualified after the competition}

\section*{The Grand Joust}\phantomsection\addcontentsline{toc}{section}{The Grand Joust}
\begin{DndReadAloud}
	Before you stretches the grand jousting arena, a testament to the kingdom's dedication to noble sportsmanship and spectacle. The tiltyard is a long, sandy stretch bordered by wooden rails, with pennants in vibrant colors fluttering from high poles. Rows of wooden stands rise on either side, packed with eager onlookers waving banners of the Kingdom of Fife. At the center of the stands, the royal box gleams with ornate carvings and golden accents, where the bride and groom sit with their court, overseeing the proceedings with regal anticipation. Squires and attendants scurry along the perimeter, preparing horses and lances while the competitors mount up for the first tilt. The air is electric with the sound of cheering, the smell of trampled earth, and the thrill of impending glory.
\end{DndReadAloud}

\subsection*{Mechanics}
{\entryfont Based on their performances in the earlier contests - the Archery, Long Throw, and Log Chop - the champions have been seeded: the first-ranked competitor will face the fourth, and the second will challenge the third. Only the victors of these matches will move on to the final joust to determine the champion of the tourney.}
\subsubsection*{Mount Familiarization}
{\entryfont Before engaging in the joust, each competitor must take time to familiarize themselves with their mount. This crucial moment allows them to build a bond of trust and control, ensuring their steed responds with precision during the high-stakes contest.}

{\entryfont Each participant must make a DC 15 Animal Handling Check to gauge how well they connect with their mount.
\begin{itemize}
	\renewcommand\labelitemi{\textbf{\textbullet}}
	\item \textbf{Success:} The rider gains a +1 bonus to their first Ride (Constitution) check during the joust.
	\item \textbf{Critical Success (Natural 20):} The rider gains a +2 bonus to their first Ride (Constitution) check and an additional +1 bonus to a potential subsequent Ride (Constitution) check.
	\item \textbf{Critical Failure (Natural 1):} The rider suffers a -2 penalty to their first Ride (Constitution) check and a -1 penalty to all subsequent Ride (Constitution) checks for the duration of the competition.
\end{itemize}}
\subsubsection*{The Joust}
{\entryfont After familiarizing themselves with their mounts, the contestants ride onto the tiltyard for the main event. The jousting competition is a dramatic test of precision, strength, and endurance. Riders charge toward one another, their tourney lances poised to strike, as the crowd cheers in anticipation of every thunderous impact.

Contestants use specially crafted tourney lances provided by the organizers. These lances deal 1d4 piercing (non-lethal) damage. If a rider deals more than 10 damage in a single round, the lance shatters spectacularly, earning the rider a +2 bonus to their next attack roll as the crowd cheers their impressive display of skill and power.

The use of magic during the joust is strictly forbidden and results in immediate disqualification.

Each pass between the two contestants is resolved through simultaneous strikes. Riders may choose one of the following attack strategies for their turn:

\begin{itemize}
	\renewcommand\labelitemi{\textbf{\textbullet}}
	\item \textbf{Aim at Helmet:}
	\begin{itemize}
		\item Attack Roll Penalty: -8 to attack roll.
		\item If the attack lands, the opposing rider must make a Ride (Constitution) Check with a DC of 15 + damage dealt to stay mounted.
	\end{itemize}
	\item \textbf{Crouch defensively:}
	\begin{itemize}
		\item Attack Roll Penalty: -4 to attack roll.
		\item The rider gains a +4 bonus to their Ride (Constitution) Check if struck.
	\end{itemize}
\end{itemize}

\noindent When struck by an opponent's lance, a rider must make a Ride (Constitution) Check (DC 5 + damage dealt) to remain in the saddle. Any rider reduced to 0 HP will automatically fail this check.

\noindent The unseated rider must succeed on a DC 15 Athletics or Acrobatics Check to land safely. On a failed check, the rider takes 1d6 bludgeoning damage from the fall.

If both riders remain mounted after a pass, the joust continues into another round. The DC for all subsequent Ride (Constitution) Checks increases by +2 after each round to represent the mounting tension and fatigue of the contest. The joust ends when only one rider remains mounted. That rider is declared the winner.

If both riders are unseated in the same round, the contest escalates into a melee duel to determine the victor. Both contestants are armed with blunt shortswords (1d4 bludgeoning, non-lethal) provided by the organizers. The duel continues as a standard one-on-one melee fight, with both riders retaining the hit points and damage accumulated during the joust. The first contestant to submit or fall unconscious loses the joust.}

\subsection*{"Party-cipation"}
{\entryfont The jousting finale may be centred on one member of the party, but the bonds of friendship and camaraderie run deep. This optional rule allows the rest of the party to actively contribute to their champion's performance, giving them a chance to support and influence the outcome in meaningful ways. Whether through cheering, strategizing, or even small magical gestures (within the festival's rules, of course), the party's involvement can turn the tide in a close contest. Following are some ideas that can be implemented:
\begin{itemize}
	\renewcommand\labelitemi{\textbf{\textbullet}}
	\item \textbf{Hype up the Crowd}\\
	The party members can use their turn to cheer, chant, or otherwise encourage their champion. Doing so requires a successful Charisma (Performance) Check DC 13 to hype up the crowd and giving the following benefits:
	\begin{itemize}
		\renewcommand\labelitemii{\textbf{\textbullet}}
		\item \textbf{Success}\\
		The champion gets a +2 bonus on either the next attack roll or the next Ride (Constitution) Check.
		\item \textbf{Critical Success}\\
		The champion also gains temporary hit points equal to the party members' highest Charisma modifier (minimum of 1).
		\item \textbf{Critical Failure}\\
		The next attack roll and a possible Ride (Constitution) Check are made with disadvantage.
	\end{itemize}
\end{itemize}
\vfill\eject
\begin{itemize}
	\renewcommand\labelitemi{\textbf{\textbullet}}
	\item \textbf{Crowd Manipulation}\\
	A particularly charismatic party may attempt to rally the crowd to their champion's favour. This requires a Charisma (Persuasion) or Charisma (Deception) Check (DC 20).
	\begin{itemize}
		\renewcommand\labelitemii{\textbf{\textbullet}}
		\item \textbf{Success}\\
		The crowd becomes fully invested in the champion's success, granting a +2 bonus to their attack rolls for the next two rounds.
		\item \textbf{Critical Success}\\
		The crowd's overwhelming support inspires the champion, giving them advantage on all checks and attack rolls for the next two rounds.
		\item \textbf{Failure}\\
		The crowd's disapproval leads to the champion being distracted, having disadvantage on the next attack and potential Ride (Constitution) check.
		\item \textbf{Critical Failure}\\
		The crowd immensely disapproves of the champion, giving them disadvantage on all checks and attack rolls for the next two rounds.
	\end{itemize}
\end{itemize}
}
\subsubsection*{Boon of the Crowd}
{\entryfont If the party influenced the match (through cheering or crowd manipulation), the champion gains the following boon:
\begin{itemize}
	\item The champion can re-roll a failed Ride (Constitution) Check or attack roll during the joust.
\end{itemize}
}

\section*{Tourney Prizes}\phantomsection\addcontentsline{toc}{section}{Tourney Prizes}
{\entryfont \paragraph*{Winner} The Winner of this prestigious tourney will be known throughout the Kingdom of Fife which will have some benefits but also drawbacks later in this campaign if a player is able to win the tournament. A player also gets a \textbf{Heroic Inspiration} and 200 GP.}
{\entryfont \paragraph*{2nd Place} The 2nd placed contestant gets 100 GP.}
{\entryfont \paragraph*{3rd / 4th Place} The 3rd and 4th placed contestants get 25 GP each.}

\begin{tikzpicture}[remember picture, overlay]%
	\node[yshift=-0.6cm, anchor=south] at (current page.south) {\includegraphics[width=\paperwidth]{%
		images/Landscape/Jousting_Tourney%
	}};%
\end{tikzpicture}%