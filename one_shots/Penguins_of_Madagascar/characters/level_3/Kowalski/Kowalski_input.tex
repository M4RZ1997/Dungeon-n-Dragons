% Headline
\CharacterName{Kowalski}

\Class{Artificer}
\Level{3}
\Background{Guild Artisan}
\PlayerName{}
\Race{SpecOp Penguin}
\Alignment{Lawful Neutral}
\XP{}

% Ability scores (correct scores, no modifiers are automatically applied)
% Modifiers, Saving Throws and Skills are calculated automatically
\StrengthScore{9}
\DexterityScore{15}
\ConstitutionScore{12}
\IntelligenceScore{18}
\WisdomScore{7}
\CharismaScore{5}

% Proficiencies (Proficient = 'P', Expertise = 'E', otherwise = '')
\StrengthProficiency{}
\DexterityProficiency{}
\ConstitutionProficiency{P}
\IntelligenceProficiency{P}
\WisdomProficiency{}
\CharismaProficiency{}

\AcrobaticsProficiency{}
\AnimalHandlingProficiency{}
\ArcanaProficiency{P}
\AthleticsProficiency{}
\DeceptionProficiency{}
\HistoryProficiency{}
\InsightProficiency{P}
\IntimidationProficiency{}
\InvestigationProficiency{P}
\MedicineProficiency{}
\NatureProficiency{}
\PerceptionProficiency{}
\PerformanceProficiency{}
\PersuasionProficiency{P}
\ReligionProficiency{}
\SleightOfHandProficiency{}
\StealthProficiency{P}
\SurvivalProficiency{}

\Inspiration{}
\Proficiency{+2}

% Armor Class is not automatically calculated
\ArmorClass{13}
\InitiativeModifier{0}
\Speed{20}
\MaxHitPointsRolled{16} % Without Constitution Bonus, is added automatically
\CurrentHitPoints{}
\TemporaryHitPoints{}
\HitDice{d8}
\HitDiceSpent{0}


\OtherProficienciesLanguages{
\textbf{Languages:}\\ Common, Dwarvish \\
\textbf{Armor:}\\ Light Armor, Medium Armor, Shields\\
\textbf{Weapons:}\\ Simple Weapons, Firearms \\
\textbf{Tools:}\\  Thieves' Tools, Tinker's Tools, Smith's Tools
}

\PersonalityTraits{
	Kowalski is able to explain things very thoroughly and has great persuasive power. However, he expresses himself very carefully to disguise mistakes, problems and shortcomings.
}

\Ideals{
	Kowalski wants to have a very large and incredibly powerful brain with which he can work out his inventions with ease.
}

\Bonds{
	Science is the ultimate power in the world.
}

\Flaws{
	Whenever Kowalski replicates a magic item or uses the Mending Spell the item has a chance to explode within the next hour.
}

\FeaturesTraits{
	\textbf{SpecOp Penguin Traits}
	\begin{itemize}
		\item Arctic Native
		\item Creature of the Sea
		\item Penguin Nimbleness
		\item Penguin Strike
	\end{itemize}
	\textbf{Guild Artisan}\\
	\textbf{Artificer Traits}
	\begin{itemize}
		\item Magical Tinkering
		\item Infuse Item
		\begin{itemize}
			\item Enhanced Arcane Focus
			\item Homunculus Servant
			\item Resistant Armor
			\item Replicate Magic Item (3)
		\end{itemize}
		\item Artillerist
		\begin{itemize}
			\item Artillerist Spells
			\item Eldritch Cannon
		\end{itemize}
		\item The Right Tool for the Job
	\end{itemize}
}

\CharacterAppearancePicture{\PATH one_shots/Penguins_of_Madagascar/characters/images/Kowalski.png}

% Magic

\SpellcastingClass{Artificer}
\SpellcastingAbility{INT} % STR, DEX, CON, INT, WIS, CHA
\SpellSaveDCModifier{0} % any modifier that isn't contained in "8 + Ability Modifier + Proficiency Bonus"

\CantripSlotA{Fire Bolt (V, S)}
\CantripSlotB{Mending (V, S, M)}

\FirstLevelSpellSlotsTotal{4}
\FirstLevelSpellSlotA{Shield (V, S)}
\FirstLevelSpellSlotB{Thunderwave (V, S)}
\FirstLevelSpellSlotC{Detect Magic (V, S)}
\FirstLevelSpellSlotD{Expeditious Retreat (V, S)}
\FirstLevelSpellSlotE{Identify (V, S, M)}

\SecondLevelSpellSlotsTotal{3}
\SecondLevelSpellSlotA{Scorching Ray (V, S)}
\SecondLevelSpellSlotB{Shatter (V, S, M)}
\SecondLevelSpellSlotC{Arcane Lock (V, S, M)}
\SecondLevelSpellSlotD{Continual Flame (V, S, M)}
\SecondLevelSpellSlotE{Darkvision (V, S, M)}
\SecondLevelSpellSlotF{Pyrotechnics (V, S, M)}

\renewcommand{\FeaturesAndSpells}{
\chapter*{Features, Magic Items and Spells}

\section*{SpecOp Penguin Traits}
\subsection*{Arctic Native}
You have resistance to cold damage. You are immune to the to the effects of both cold weather and Extreme Cold as described in the Dungeon Master’s Guide (page 110). Furthermore, you do not treat shallow water, snow, or ice as difficult terrain.
\subsection*{Creature of the Sea}
You can breathe air and water, and you have resistance to cold damage. Aquatic animals have an extraordinary affinity with your people. You can communicate simple ideas to any Beast that has a swimming speed. It can understand your words, though you have no special ability to understand it in return.
\subsection*{Penguin Nimbleness}
You can move through the space of any creature that is of a size larger than yours. Your base walking speed is 20 feet but you can use your bonus action to "belly slide"; increasing your walking speed by 20 feet. While "belly sliding" you cannot use the Attack Action and also cannot cast spells or use objects. As an action you can stop belly sliding and stand up. You also have a swim speed of 50 feet.
\subsection*{Penguin Strike}
If you have advantage on the attack roll and you are hidden from the target, you can make an unarmed strike stunning the target until the end of its next turn. The target must make a Constitution saving throw against your attack roll DC (the DC can be set at 10 + your Strength or Dexterity modifier, depending on your character build) or is stunned for 1 addtional round. If the target's hit points are equal to or lower than half of their maximum hit points and they fail the save, they are knocked unconscious for 1 minute or until they take damage. You can use this attack equal to half your player level rounded down per long rest.

\section*{Elemental Adept}
\textbf{Fire, Thunder}\\
Spells you cast ignore resistance to damage of the chosen type. In addition, when you roll damage for a spell you cast that deals damage of that type, you can treat any 1 on a damage die as a 2.

\section*{Artificer Traits}
Masters of invention, artificers use ingenuity and magic to unlock extraordinary capabilities in objects. They see magic as a complex system waiting to be decoded and then harnessed in their spells and inventions. You can find everything you need to play one of these inventors in the next few sections.

Artificers use a variety of tools to channel their arcane power. To cast a spell, an artificer might use alchemist's supplies to create a potent elixir, calligrapher's supplies to inscribe a sigil of power, or tinker's tools to craft a temporary charm. The magic of artificers is tied to their tools and their talents, and few other characters can produce the right tool for a job as well as an artificer.
\subsection*{Magical Tinkering}
At 1st level, you've learned how to invest a spark of magic into mundane objects. To use this ability, you must have thieves' tools or artisan's tools in hand. You then touch a Tiny nonmagical object as an action and give it one of the following magical properties of your choice:
\begin{itemize}
	\item The object sheds bright light in a 5-foot radius and dim light for an additional 5 feet.
	\item Whenever tapped by a creature, the object emits a recorded message that can be heard up to 10 feet away. You utter the message when you bestow this property on the object, and the recording can be no more than 6 seconds long.
	\item The object continuously emits your choice of an odor or a nonverbal sound (wind, waves, chirping, or the like). The chosen phenomenon is perceivable up to 10 feet away.
	\item A static visual effect appears on one of the object's surfaces. This effect can be a picture, up to 25 words of text, lines and shapes, or a mixture of these elements, as you like.
\end{itemize}
The chosen property lasts indefinitely. As an action, you can touch the object and end the property early.

You can bestow magic on multiple objects, touching one object each time you use this feature, though a single object can only bear one property at a time. The maximum number of objects you can affect with this feature at one time is equal to your Intelligence modifier (minimum of one object). If you try to exceed your maximum, the oldest property immediately ends, and then the new property applies.

(\textbf{Usages: \intcalcAdd{0}{\calculateModifier{\IntelligenceScoreValue}}})
\subsection*{Infuse Item}
\textbf{6 Known}\\
\textbf{3 Infused Items at a time}\\
At 2nd level, you've gained the ability to imbue mundane items with certain magical infusions, turning those objects into magic items.

Whenever you gain a level in this class, you can replace one of the artificer infusions you learned with a new one.
\subsubsection*{Infusing an Item}
Whenever you finish a long rest, you can touch a nonmagical object and imbue it with one of your artificer infusions, turning it into a magic item. An infusion works on only certain kinds of objects, as specified in the infusion's description. If the item requires attunement, you can attune yourself to it the instant you infuse the item. If you decide to attune to the item later, you must do so using the normal process for attunement (see the attunement rules in the Dungeon Master's Guide).

Your infusion remains in an item indefinitely, but when you die, the infusion vanishes after a number of days equal to your Intelligence modifier (minimum of 1 day). The infusion also vanishes if you replace your knowledge of the infusion.

You can infuse more than one nonmagical object at the end of a long rest; the maximum number of objects appears in the Infused Items column of the Artificer table. You must touch each of the objects, and each of your infusions can be in only one object at a time. Moreover, no object can bear more than one of your infusions at a time. If you try to exceed your maximum number of infusions, the oldest infusion ends, and then the new infusion applies.

If an infusion ends on an item that contains other things, like a bag of holding, its contents harmlessly appear in and around its space.
\subsubsection*{Known Infusions}
\paragraph*{Enhanced Arcane Focus}\hfill\\
\textbf{Item: A rod, staff or wand (requires attunement)}\\
While holding this item, a creature gains +1 bonus to spell attack rolls. In addition, the creature ignores half cover when making a spell attack.

The bonus increases to +2 when you reach 10th level in this class.\\
\paragraph*{Homunculus Servant}\hfill\\
\textbf{Item: A gem or crystal worth at least 100 gp}\\
You learn intricate methods for magically creating a special homunculus that serves you. The item you infuse serves as the creature's heart, around which the creature's body instantly forms.

You determine the homunculus's appearance. Some artificers prefer mechanical-looking birds, whereas some like winged vials or miniature, animate cauldrons.

The homunculus is friendly to you and your companions, and it obeys your commands. See this creature's game statistics in the Homunculus Servant stat block, which uses your proficiency bonus (PB) in several places.

In combat, the homunculus shares your initiative count, but it takes its turn immediately after yours. It can move and use its reaction on its own, but the only action it takes on its turn is the Dodge action, unless you take a bonus action on your turn to command it to take another action. That action can be one in its stat block or some other action. If you are incapacitated, the homunculus can take any action of its choice, not just Dodge.

The homunculus regains 2d6 hit points if the mending spell is cast on it. If you or the homunculus dies, it vanishes, leaving its heart in its space.
\begin{DndMonster}[width=0.5\textwidth]{Homunculus Servant}
    \DndMonsterType{Tiny Construct}

    % If you want to use commas in the key values, enclose the values in braces.
    \DndMonsterBasics[
        armor-class = {13 (Natural Armor)},
        hit-points  = {\intcalcAdd{1}{\intcalcAdd{\calculateModifier{\IntelligenceScoreValue}}{\LevelValue}} (\LevelValue d4)},
        speed       = {20 ft., fly 30 ft.},
    ]
    
	\renewcommand{\AbilityScoreSpacer}{~}
    \DndMonsterAbilityScores[
		str = 4,
		dex = 15,
		con = 12,
		int = 10,
		wis = 10,
		cha = 7,
    ]

    \DndMonsterDetails[
        saving-throws = {Dex +0 + PB},
        skills = {Perception +0 + PB x 2, Stealth +2 + PB},
        %damage-vulnerabilities = {cold},
        %damage-resistances = {bludgeoning, piercing, and slashing from nonmagical attacks},
        damage-immunities = {poison},
        senses = {Darkvision 60 ft., Passive Perception 10 + (PB x 2)},
        condition-immunities = {poisoned},
        languages = {understands the languages you speak},
        challenge = 1,
    ]
    
    \DndMonsterAction{Evasion}
    If the homunculus is subjected to an effect that allows it to make a Dexterity saving throw to take only half damage, it instead takes no damage if it succeeds on the saving throw, and only half damage if it fails. It can't use this trait if it's incapacitated.
	
	\DndMonsterSection{Actions}	
	\DndMonsterAttack[
      name=Force Strike,
      distance=ranged, % valid options are in the set {both,melee,ranged},
      %type=weapon, %valid options are in the set {weapon,spell}
      mod=\calculateSpellAttack{\calculateModifier{\IntelligenceScoreValue}},
      %reach=5,
      range=30,
      targets=one target you can see,
      dmg=\DndDice{1d4} + PB,
      dmg-type=force,
      %plus-dmg=,
      %plus-dmg-type=,
      %or-dmg=,
      %or-dmg-when=,
      %extra=,
    ]
    
    \DndMonsterSection{Reactions}
	\DndMonsterAction{Channel Magic}
	The homunculus delivers a spell you cast that has a range of touch. The homunculus must be within 120 feet of you.	
\end{DndMonster}
\paragraph*{Resistant Armor}\hfill\\
\textbf{Prerequisite: 6th-level artificer}\\
\textbf{Item: A suit of armor (requires attunement)}\\
While wearing this armor, a creature has resistance to one of the following damage types, which you choose when you infuse the item: acid, cold, fire, force, lightning, necrotic, poison, psychic, radiant, or thunder.\\
\paragraph*{Replicate Magic Item (3)}\hfill\\
Using this infusion, you replicate a particular magic item. You can learn this infusion multiple times; each time you do so, choose a magic item that you can make with it, picking from the Replicable Items tables. A table's title tells you the level you must be in the class to choose an item from the table. Alternatively, you can choose the magic item from among the common magic items in the game, not including potions or scrolls.

In the tables, an item's entry tells you whether the item requires attunement. See the item's description in the Dungeon Master's Guide for more information about it, including the type of object required for its making.\\
\textbf{Chosen Magic Items:}
\begin{itemize}
	\item Smoke Grenade
	\item Bag of Holding
	\item Cloak of Elvenkind
\end{itemize}
\textbf{Kowalski's Flaw:} The newly recreated magical items have a 25 percent chance to explode within the next hour but also have a 5 percent chance to gain a beneficial attribute. The explosion range is 20 feet and all creatures within range must make a DC 15 Dexterity Saving Throw or take \DndDice{2d6} force damage. The beneficial attributes are at the DM's discretion.\\
\subparagraph*{Smoke Grenade}
As an action, a character can throw a grenade at a point up to 60 feet away. With a grenade launcher, the character can propel the grenade up to 120 feet away.

At the end of the turn after a smoke grenade lands, it emits a cloud of smoke that creates a heavily obscured area in a 20-foot radius. A moderate wind (at least 10 miles per hour) disperses the smoke in 4 rounds; a strong wind (20 or more miles per hour) disperses it in 1 round.\\
\subparagraph*{Bag of Holding}
This bag has an interior space considerably larger than its outside dimensions, roughly 2 feet in diameter at the mouth and 4 feet deep. The bag can hold up to 500 pounds, not exceeding a volume of 64 cubic feet. The bag weighs 15 pounds, regardless of its contents. Retrieving an item from the bag requires an action.

If the bag is overloaded, pierced, or torn, it ruptures and is destroyed, and its contents are scattered in the Astral Plane. If the bag is turned inside out, its contents spill forth, unharmed, but the bag must be put right before it can be used again. Breathing creatures inside the bag can survive up to a number of minutes equal to 10 divided by the number of creatures (minimum 1 minute), after which time they begin to suffocate.

Placing a bag of holding inside an extradimensional space created by a Heward's Handy Haversack, Portable Hole, or similar item instantly destroys both items and opens a gate to the Astral Plane. The gate originates where the one item was placed inside the other. Any creature within 10 feet of the gate is sucked through it to a random location on the Astral Plane. The gate then closes. The gate is one-way only and can't be reopened.\\
\subparagraph*{Cloak of Elvenkind}
While you wear this cloak with its hood up, Wisdom (Perception) checks made to see you have disadvantage, and you have advantage on Dexterity (Stealth) checks made to hide, as the cloak's color shifts to camouflage you. Pulling the hood up or down requires an action.
\subsection*{Artillerist}
An Artillerist specializes in using magic to hurl energy, projectiles, and explosions on a battlefield. This destructive power is valued by armies in the wars on many different worlds. And when war passes, some members of this specialization seek to build a more peaceful world by using their powers to fight the resurgence of strife. The world-hopping gnome artificer Vi has been especially vocal about making things right: "It's about time we fixed things instead of blowing them all to hell."
\subsubsection*{Artillerist Spells}
Starting at 3rd level, you always have certain spells prepared after you reach particular levels in this class, as shown in the Artillerist Spells table. These spells count as artificer spells for you, but they don’t count against the number of artificer spells you prepare.
\begin{DndTable}[header=Artillerist Spells]{llX}
			& \textbf{Artificer Level}  	&\textbf{Spells}				\\
$\bullet$	& 3rd						&Shield, Thunderwave			\\
$\bullet$	& 5th						&Scorching Ray, Shatter			\\
			& 9th						&Fireball, Wind Wall			\\
			& 13th						&Ice Storm, Wall of Fire		\\
			& 17th						&Cone of Cold, Wall of Force	\\
\end{DndTable}
\subsubsection*{Edlritch Cannon}
Also at 3rd level, you've learned how to create a magical cannon. Using woodcarver's tools or smith's tools, you can take an action to magically create a Small or Tiny eldritch cannon in an unoccupied space on a horizontal surface within 5 feet of you. A Small eldritch cannon occupies its space, and a Tiny one can be held in one hand. Once you create a cannon, you can't do so again until you finish a long rest or until you expend a spell slot to create one. You can have only one cannon at a time and can't create one while your cannon is present.

The cannon is a magical object. Regardless of size, the cannon has an AC of 18 and a number of hit points equal to five times your artificer level. It is immune to poison damage and psychic damage. If it is forced to make an ability check or a saving throw, treat all its ability scores as 10 (+0). If the mending spell is cast on it, it regains 2d6 hit points. It disappears if it is reduced to 0 hit points or after 1 hour. You can dismiss it early as an action.

When you create the cannon, you determine its appearance and whether it has legs. You also decide which type it is, choosing from the options on the Eldritch Cannons table. On each of your turns, you can take a bonus action to cause the cannon to activate if you are within 60 feet of it. As part of the same bonus action, you can direct the cannon to walk or climb up to 15 feet to an unoccupied space, provided it has legs.
\begin{DndTable}[header=Eldritch Cannon]{lX}
\textbf{Cannon}  	&\textbf{Activation}				\\
Flamethrower		&The cannon exhales fire in an adjacent 15-foot cone that you designate. Each creature in that area must make a Dexterity saving throw against your spell save DC, taking 2d8 fire damage on a failed save or half as much damage on a successful one. The fire ignites any flammable objects in the area that aren't being worn or carried.\\
Force Ballista		&Make a ranged spell attack, originating from the cannon, at one creature or object within 120 feet of it. On a hit, the target takes 2d8 force damage, and if the target is a creature, it is pushed up to 5 feet away from the cannon.\\
Protector			&The cannon emits a burst of positive energy that grants itself and each creature of your choice within 10 feet of it a number of temporary hit points equal to 1d8 + your Intelligence modifier (minimum of +1).\\
\end{DndTable}
\paragraph*{Kowalski's Flaw}
The Eldritch Cannon has a  20 percent chance to explode when using its action. Each creature within 20 feet must make a DC 15 Dexterity Saving Throw or take \DndDice{2d6} force damage. It has also a 10 percent chance to get the following  benefits (determined when the cannon is created):
\begin{itemize}
	\item \textbf{Flamethrower} The fire attack has a range of 25 feet.
	\item \textbf{Force Ballista} The Force Ballista's attack is considered a crit on a 19 or 20 roll.
	\item \textbf{Protector} The positive energy heals 2d8 + your Intelligence modifier (minimum of +1) of health points.
\end{itemize}
\subsubsection*{Arcane Firearm}
At 5th level, You know how to turn a wand, staff, or rod into an arcane firearm, a conduit for your destructive spells. When you finish a long rest, you can use woodcarver's tools to carve special sigils into a wand, staff, or rod and thereby turn it into your arcane firearm. The sigils disappear from the object if you later carve them on a different item. The sigils otherwise last indefinitely.

You can use your arcane firearm as a spellcasting focus for your artificer spells. When you cast an artificer spell through the firearm, roll a d8, and you gain a bonus to one of the spell's damage rolls equal to the number rolled.
\subsection*{The Right Tool for the Job}
At 3rd level, you've learned how to produce exactly the tool you need: with thieves' tools or artisan's tools in hand, you can magically create one set of artisan's tools in an unoccupied space within 5 feet of you. This creation requires 1 hour of uninterrupted work, which can coincide with a short or long rest. Though the product of magic, the tools are nonmagical, and they vanish when you use this feature again.
\subsection*{Tool Expertise}
At 6th level, your proficiency bonus is now doubled for any ability check you make that uses your proficiency with a tool.
\subsection*{Flash of Genius}
At 7th level, you've gained the ability to come up with solutions under pressure. When you or another creature you can see within 30 feet of you makes an ability check or a saving throw, you can use your reaction to add your Intelligence modifier to the roll.

You can use this feature a number of times equal to your Intelligence modifier (minimum of once). You regain all expended uses when you finish a long rest.\\
(\textbf{Usages: \intcalcAdd{0}{\calculateModifier{\IntelligenceScoreValue}}})

\section*{Spells}
\subsection*{Cantrips}

\DndSpellHeader
	{Fire Bolt}
	{Evocation cantrip}
	{1 Action}
	{120 feet}
	{V, S}
	{Instantaneous}

You hurl a mote of fire at a creature or object within range. Make a ranged spell attack against the target. On a hit, the target takes 1d10 fire damage. A flammable object hit by this spell ignites if it isn’t being worn or carried.

\subparagraph*{At Higher Levels} This spell’s damage increases by 1d10 when you reach 5th level (2d10), 11th level (3d10), and 17th level (4d10).

\DndSpellHeader
  {Mending}
  {Transmutation cantrip}
  {1 Minute}
  {Touch}
  {V, S, M (two lodestones)}
  {Instantaneous}

This spell repairs a single break or tear in an object you touch, such as a broken chain link, two halves of a broken key, a torn cloak, or a leaking wineskin. As long as the break or tear is no larger than 1 foot in any dimension, you mend it, leaving no trace of the former damage.

This spell can physically repair a magic item or construct, but the spell can’t restore magic to such an object.

\subparagraph*{Kowalski's Flaw:} There is a 10 percent change that the mended object explodes within the next hour after the spell was cast. Each creature within 15 feet must make a DC 15 Dexterity Saving Throw or take \DndDice{2d6} force damage.

\subsection*{Level 1}

\DndSpellHeader
  {Shield}
  {1st-Level Abjuration}
  {1 Reaction, which you take when you are hit by an attack or targeted by the magic missile spell}
  {Self}
  {V, S}
  {1 Round}

An invisible barrier of magical force appears and protects you. Until the start of your next turn, you have a +5 bonus to AC, including against the triggering attack, and you take no damage from magic missile.

\DndSpellHeader
  {Thunderwave}
  {1st-Level Evocation}
  {1 Action}
  {Self (15-foot cube)}
  {V, S}
  {Instantaneous}

A wave of thunderous force sweeps out from you. Each creature in a 15-foot cube originating from you must make a Constitution saving throw. On a failed save, a creature takes 2d8 thunder damage and is pushed 10 feet away from you. On a successful save, the creature takes half as much damage and isn’t pushed.

In addition, unsecured objects that are completely within the area of effect are automatically pushed 10 feet away from you by the spell’s effect, and the spell emits a thunderous boom audible out to 300 feet.

\subparagraph*{At Higher Levels} When you cast this spell using a spell slot of 2nd level or higher, the damage increases by 1d8 for each slot level above 1st.

\DndSpellHeader
  {Detect Magic}
  {1st-Level Divination (Ritual)}
  {1 Action}
  {Self}
  {V, S}
  {Concentration, up to 10 Minutes}

For the duration, you sense the presence of magic within 30 feet of you. If you sense magic in this way, you can use your action to see a faint aura around any visible creature or object in the area that bears magic, and you learn its school of magic, if any.

The spell can penetrate most barriers, but is blocked by 1 foot of stone, 1 inch of common metal, a thin sheet of lead, or 3 feet of wood or dirt.

\DndSpellHeader
  {Expeditious Retreat}
  {1st-Level Transmutation}
  {1 Bonus Action}
  {Self}
  {V, S}
  {Concentration, up to 10 Minutes}

This spell allows you to move at an incredible pace. When you cast this spell, and then as a bonus action on each of your turns until the spell ends, you can take the Dash action.

\DndSpellHeader
  {Identify}
  {1st-Level Divination (Ritual)}
  {1 Minute}
  {Touch}
  {V, S, M (a pearl worth at least 100 gp and an owl feather)}
  {Instantaneous}

You choose one object that you must touch throughout the casting of the spell. If it is a magic item or some other magic-imbued object, you learn its properties and how to use them, whether it requires attunement to use, and how many charges it has, if any. You learn whether any spells are affecting the item and what they are. If the item was created by a spell, you learn which spell created it.

If you instead touch a creature throughout the casting, you learn what spells, if any, are currently affecting it.

\subsection*{Level 2}

\DndSpellHeader
  {Scorching Ray}
  {2nd-Level Evocation}
  {1 Action}
  {120 feet}
  {V, S}
  {Instantaneous}

You create three rays of fire and hurl them at targets within range. You can hurl them at one target or several. Make a ranged spell attack for each ray. On a hit, the target takes 2d6 fire damage.

\subparagraph*{At Higher Levels} When you cast this spell using a spell slot of 3rd level or higher, you create one additional ray for each slot level above 2nd.

\DndSpellHeader
  {Shatter}
  {2nd-Level Evocation}
  {1 Action}
  {60 feet}
  {V, S, M (a chip of mica)}
  {Instantaneous}

A sudden loud ringing noise, painfully intense, erupts from a point of your choice within range. Each creature in a 10-foot-radius sphere centered on that point must make a Constitution saving throw. A creature takes 3d8 thunder damage on a failed save, or half as much damage on a successful one. A creature made of inorganic material such as stone, crystal, or metal has disadvantage on this saving throw.

A nonmagical object that isn’t being worn or carried also takes the damage if it’s in the spell’s area.

\subparagraph*{At Higher Levels} When you cast this spell using a spell slot of 3rd level or higher, the damage increases by 1d8 for each slot level above 2nd.

\DndSpellHeader
  {Continual Flame}
  {2nd-Level Evocation}
  {1 Action}
  {Touch}
  {V, S, M (ruby dust worth 50 gp, which the spell consumes)}
  {Until dispelled}

A flame, equivalent in brightness to a torch, springs forth from an object that you touch. The effect looks like a regular flame, but it creates no heat and doesn’t use oxygen. A continual flame can be covered or hidden but not smothered or quenched.

\DndSpellHeader
  {Darkvision}
  {2nd-Level Transmutation}
  {1 Action}
  {Touch}
  {V, S, M (either a pinch of dried carrot or an agate)}
  {8 hours}

You touch a willing creature to grant it the ability to see in the dark. For the duration, that creature has darkvision out to a range of 60 feet.

\DndSpellHeader
  {Pyrotechnics}
  {2nd-Level Transmutation}
  {60 feet}
  {Target}
  {V, S}
  {Instantaneous}

Choose an area of flame that you can see and that can fit within a 5-foot cube within range. You can extinguish the fire in that area, and you create either fireworks or smoke.

\subparagraph*{Fireworks} The target explodes with a dazzling display of colors. Each creature within 10 feet of the target must succeed on a Constitution saving throw or become blinded until the end of your next turn.

\subparagraph*{Smoke} Thick black smoke spreads out from the target in a 20-foot radius, moving around corners. The area of the smoke is heavily obscured. The smoke persists for 1 minute or until a strong wind disperses it.
}