% Headline
\CharacterName{Maurice}

\Class{Druid}
\Level{5}
\Background{Hermit}
\PlayerName{}
\Race{Lemur o. M.}
\Alignment{Lawful Good}
\XP{}

% Ability scores (correct scores, no modifiers are automatically applied)
% Modifiers, Saving Throws and Skills are calculated automatically
\StrengthRolledScore{8}
\DexterityRolledScore{13}
\ConstitutionRolledScore{12}
\IntelligenceRolledScore{14}
\WisdomRolledScore{16}
\CharismaRolledScore{10}

\StrengthScoreBonus{0}
\DexterityScoreBonus{1} % Lemur-Race +1
\ConstitutionScoreBonus{0}
\IntelligenceScoreBonus{0}
\WisdomScoreBonus{0}
\CharismaScoreBonus{1} % Lemur-Race +1

\calculateAbilityScores{}

% Proficiencies (Proficient = 'P', Expertise = 'E', otherwise = '')
\StrengthProficiency{}
\DexterityProficiency{}
\ConstitutionProficiency{}
\IntelligenceProficiency{P}
\WisdomProficiency{P}
\CharismaProficiency{}

\AcrobaticsProficiency{P}
\AnimalHandlingProficiency{}
\ArcanaProficiency{}
\AthleticsProficiency{}
\DeceptionProficiency{}
\HistoryProficiency{}
\InsightProficiency{P}
\IntimidationProficiency{}
\InvestigationProficiency{}
\MedicineProficiency{P}
\NatureProficiency{}
\PerceptionProficiency{P}
\PerformanceProficiency{P}
\PersuasionProficiency{}
\ReligionProficiency{P}
\SleightOfHandProficiency{}
\StealthProficiency{}
\SurvivalProficiency{}

\StrengthModifierBonus{0}
\DexterityModifierBonus{0}
\ConstitutionModifierBonus{0}
\IntelligenceModifierBonus{0}
\WisdomModifierBonus{0}
\CharismaModifierBonus{0}

\Inspiration{}
\Proficiency{+3}

% Armor Class is not automatically calculated
\ArmorClass{14}
\InitiativeModifier{0}
\Speed{30}
\MaxHitPointsRolled{30} % Without Constitution Bonus, is added automatically
\CurrentHitPoints{}
\TemporaryHitPoints{}
\HitDice{d8}
\HitDiceSpent{0}

\addWeaponStatistic{Quarterstaff}{Maurice}{STR}{P}{0}{1d6 b}
\addWeaponStatistic{Quarterstaff}{MauriceV}{STR}{P}{0}{1d8 b (v)}
\addWeaponStatistic{Unarmed Strike}{BlankCharacter}{STR}{P}{0}{\intcalcAdd{1}{\calculateModifier{\StrengthScoreValue}} b}

\AttacksAdditional{
	Quarterstaff\\
	Hide Armor\\
	\textbf{Insignia of Claws}
}

\OtherProficienciesLanguages{
\textbf{Languages:}\\ Common, Chameleon Color Language, Druidic \\
\textbf{Armor:} Light Armor, Medium Armor, Shields (won't wear armor or use shields made of metal)\\
\textbf{Weapons:} Clubs, Daggers, Darts, Javelins, Maces, Quarterstaff, Scimitars, Sickles, Slings, Spears \\
\textbf{Tools:}\\ Herbalism Kit
}

\PersonalityTraits{
	Maurice is often portrayed as the voice of reason and wisdom among the lemurs, offering thoughtful advice and guidance. He is known for his patience, often trying to calm the impulsive nature of King Julien and the other lemurs.
}

\Ideals{
	Maurice values stability and order, striving to maintain a sense of balance and calm in the chaotic world of the lemurs.
}

\Bonds{
	Maurice has a strong bond with King Julien, serving as his right-hand lemur and offering him guidance and support.
}

\Flaws{
	Maurice's cautious nature can lead to indecision and reluctance to take risks, potentially hindering progress.
}

\FeaturesTraits{
	\textbf{Lemur of Madagascar Traits}
	\begin{itemize}
		\item Like to MOVE IT!
		\item Stealth Sense
		\item Arboreal Movement
	\end{itemize}
	\textbf{Hermit}\\
	\textbf{Metamagic Adept}\\
	\textbf{Druid Traits}
	\begin{itemize}
		\item Wild Shape
		\item Druid Circle (Spores)
		\begin{itemize}
			\item Circle Spells
			\item Halo of Spores
			\item Symbiotic Entity
		\end{itemize}
	\end{itemize}
}

\CharacterAppearancePicture{\PATH one_shots/Penguins_of_Madagascar/characters/images/Maurice.png}

% Magic

\SpellcastingClass{Druid}
\SpellcastingAbility{WIS} % STR, DEX, CON, INT, WIS, CHA
\SpellSaveDCModifier{0} % any modifier that isn't contained in "8 + Ability Modifier + Proficiency Bonus"
\SpellAttackModifier{0} % any modifier that isn't contained in "Ability Modifier + Proficiency Bonus"

\CantripSlotA{Chill Touch (V, S)}
\CantripSlotB{Guidance (V, S)}
\CantripSlotC{Produce Flame (V, S)}
\CantripSlotD{Shape Water (S)}

\FirstLevelSpellSlotsTotal{4}
\FirstLevelSpellSlotA{Detect Magic (V, S)}
\FirstLevelSpellSlotB{Entangle (V, S)}
\FirstLevelSpellSlotC{Healing Word (V)}

\SecondLevelSpellSlotsTotal{3}
\SecondLevelSpellSlotA{Blindness/Deafness (V)}
\SecondLevelSpellSlotB{Gentle Repose (V, S, M)}
\SecondLevelSpellSlotC{Augury (V, S, M)}
\SecondLevelSpellSlotD{Darkvision (V, S, M)}
\SecondLevelSpellSlotE{Lesser Restoration (V, S)}

\ThirdLevelSpellSlotsTotal{2}
\ThirdLevelSpellSlotA{Animate Dead (V, S, M)}
\ThirdLevelSpellSlotB{Gaseous Form (V, S, M)}
\ThirdLevelSpellSlotC{Conjure Animals (V, S)}
\ThirdLevelSpellSlotD{Dispel Magic (V, S)}

\renewcommand{\FeaturesAndSpells}{
\chapter*{Features, Magic Items and Spells}

\section*{Lemur of Madagascar Traits}
\subsection*{Like to MOVE IT!}
You gain proficiency in the Performance and Acrobatics skills. If you already have proficiency in those skills or gain these proficiency, you will gain expertise in those skills instead. You also gain advantage for Performance Skill rolls if performing in a group of size 3 or larger.
\subsection*{Stealth Sense}
When well-rested, you are able to sense that someone or something is in stealth but you are unable to pinpoint its' location if you are within 50 feet of it.
\subsection*{Arboreal Movement}
You have a climbing speed of 35 feet and roll with advantage on climbing/jumping tasks.

\section*{Metmagic Adept}
You've learned how to exert your will on your spells to alter how they function:
\begin{itemize}
	\item You learn two Metamagic options of your choice from the sorcerer class. You can use only one Metamagic option on a spell when you cast it, unless the option says otherwise. Whenever you reach a level that grants the Ability Score Improvement feature, you can replace one of these Metamagic options with another one from the sorcerer class.
	\item You gain 2 sorcery points to spend on Metamagic (these points are added to any sorcery points you have from another source but can be used only on Metamagic). You regain all spent sorcery points when you finish a long rest.
\end{itemize}
\subsection*{Matemagic Options}
\paragraph*{Subtle Spell} When you cast a spell, you can spend 1 sorcery point to cast it without any somatic or verbal components.
\paragraph*{Transmuted Spell} When you cast a spell that deals a type of damage from the following list, you can spend 1 sorcery point to change that damage type to one of the other listed types: acid, cold, fire, lightning, poison, thunder.

\section*{Druid Traits}
Whether calling on the elemental forces of nature or emulating the creatures of the animal world, druids are an embodiment of nature's resilience, cunning, and fury. They claim no mastery over nature, but see themselves as extensions of nature's indomitable will.
\subsection*{Wild Shape}
Starting at 2nd level, you can use your action to magically assume the shape of a beast that you have seen before. You can use this feature twice. You regain expended uses when you finish a short or long rest.

Your druid level determines the beasts you can transform into, as shown in the Beast Shapes table. At 2nd level, for example, you can transform into any beast that has a challenge rating of 1/4 or lower that doesn't have a flying or swimming speed.
\begin{DndTable}[header=Beast Shapes]{llXl}
\textbf{Level}	& \textbf{Max. CR}  	&\textbf{Limitations}			&\textbf{Example}	\\
2nd 			& 1/4					&No flying or swimming speed	&Wolf				\\
4th 			& 1/2					&No flying speed				&Crocodile			\\
8th 			& 1						&								&Giant Eagle		\\
\end{DndTable}
You can stay in a beast shape for a number of hours equal to half your druid level (rounded down). You then revert to your normal form unless you expend another use of this feature. You can revert to your normal form earlier by using a bonus action on your turn. You automatically revert if you fall unconscious, drop to 0 hit points, or die.

While you are transformed, the following rules apply:
\begin{itemize}
	\item Your game statistics are replaced by the statistics of the beast, but you retain your alignment, personality, and Intelligence, Wisdom, and Charisma scores. You also retain all of your skill and saving throw proficiencies, in addition to gaining those of the creature. If the creature has the same proficiency as you and the bonus in its stat block is higher than yours, use the creature's bonus instead of yours. If the creature has any legendary or lair actions, you can't use them.
	\item When you transform, you assume the beast's hit points and Hit Dice. When you revert to your normal form, you return to the number of hit points you had before you transformed. However, if you revert as a result of dropping to 0 hit points, any excess damage carries over to your normal form, For example, if you take 10 damage in animal form and have only 1 hit point left, you revert and take 9 damage. As long as the excess damage doesn't reduce your normal form to 0 hit points, you aren't knocked unconscious.
	\item You can't cast spells, and your ability to speak or take any action that requires hands is limited to the capabilities of your beast form. Transforming doesn't break your concentration on a spell you've already cast, however, or prevent you from taking actions that are part of a spell, such as Call Lightning, that you've already cast.
	\item You retain the benefit of any features from your class, race, or other source and can use them if the new form is physically capable of doing so. However, you can't use any of your special senses, such as darkvision, unless your new form also has that sense.
	\item You choose whether your equipment falls to the ground in your space, merges into your new form, or is worn by it. Worn equipment functions as normal, but the DM decides whether it is practical for the new form to wear a piece of equipment, based on the creature's shape and size. Your equipment doesn't change size or shape to match the new form, and any equipment that the new form can't wear must either fall to the ground or merge with it. Equipment that merges with the form has no effect until you leave the form.
\end{itemize}
\subsection*{Druid Circle}
At 2nd level, you choose to identify with a circle of druids. Your choice grants you features at 2nd level and again at 6th, 10th, and 14th level.
\subsection*{Druid of Spores}
Druids of the Circle of Spores find beauty in decay. They see within mold and other fungi the ability to transform lifeless material into abundant, albeit somewhat strange, life. These druids believe that life and death are parts of a grand cycle, with one leading to the other and then back again. Death isn't the end of life, but instead a change of state that sees life shift into a new form.

Druids of this circle have a complex relationship with the undead. They see nothing inherently wrong with undeath, which they consider to be a companion to life and death. But these druids believe that the natural cycle is healthiest when each segment of it is vibrant and changing. Undead that seek to replace all life with undeath, or that try to avoid passing to a final rest, violate the cycle and must be thwarted.
\subsubsection*{Circle Spells}
Your symbiotic link to fungi and your ability to tap into the cycle of life and death grants you access to certain spells. At 2nd level, you learn the Chill Touch cantrip.

At 3rd, 5th, 7th, and 9th level you gain access to the spells listed for that level in the Circle of Spores Spells table. Once you gain access to one of these spells, you always have it prepared, and it doesn't count against the number of spells you can prepare each day. If you gain access to a spell that doesn't appear on the druid spell list, the spell is nonetheless a druid spell for you.
\begin{DndTable}[header=Circle of Spores Spells]{llX}
			& \textbf{Druid Level}  	&\textbf{Circle Spells}				\\
$\bullet$	& 2nd						&Chill Touch						\\
$\bullet$	& 3rd						&Blindness/Deafness, Gentle Repose	\\
$\bullet$	& 5th						&Animate Dead, Gaseous Form			\\
			& 7th						&Blight, Confusion					\\
			& 9th						&Cloudkill, Contagion				\\
\end{DndTable}
\subsubsection*{Halo of Spores}
Starting at 2nd level, you are surrounded by invisible, necrotic spores that are harmless until you unleash them on a creature nearby. When a creature you can see moves into a space within 10 feet of you or starts its turn there, you can use your reaction to deal 1d4 necrotic damage to that creature unless it succeeds on a Constitution saving throw against your spell save DC. The necrotic damage increases to 1d6 at 6th level, 1d8 at 10th level, and 1d10 at 14th level.
\subsubsection*{Symbiotic Entity}
Also at 2nd level, you gain the ability to channel magic into your spores. As an action, you can expend a use of your Wild Shape feature to awaken those spores, rather than transforming into a beast form, and you gain 4 temporary hit points for each level you have in this class. While this feature is active, you gain the following benefits:
\begin{itemize}
	\item When you deal your Halo of Spores damage, roll the damage die a second time and add it to the total.
	\item Your melee weapon attacks deal an extra 1d6 necrotic damage to any target they hit.
\end{itemize}
These benefits last for 10 minutes, until you lose all these temporary hit points or until you use your Wild Shape again.

\section*{Spells}
\subsection*{Cantrip}

\DndSpellHeader
  {Chill Touch}
  {Necromancy Cantrip}
  {1 Action}
  {120 foot}
  {V, S}
  {1 Round}

You create a ghostly, skeletal hand in the space of a creature within range. Make a ranged spell attack against the creature to assail it with the chill of the grave. On a hit, the target takes 1d8 necrotic damage, and it can’t regain hit points until the start of your next turn. Until then, the hand clings to the target. If you hit an undead target, it also has disadvantage on attack rolls against you until the end of your next turn.

\subparagraph*{At Higher Levels} This spell's damage increases by 1d8 when you reach 5th level (2d8), 11th level (3d8), and 17th level (4d8).

\DndSpellHeader
  {Guidance}
  {Divination Cantrip}
  {1 Action}
  {Touch}
  {V, S}
  {Concentration, up to 1 Minute}

You touch one willing creature. Once before the spell ends, the target can roll a d4 and add the number rolled to one ability check of its choice. It can roll the die before or after making the ability check. The spell then ends.

\DndSpellHeader
  {Produce Flame}
  {Conjuration Cantrip}
  {1 Action}
  {Self}
  {V, S}
  {10 Minutes}

A flickering flame appears in your hand. The flame remains there for the duration and harms neither you nor your equipment. The flame sheds bright light in a 10-foot radius and dim light for an additional 10 feet. The spell ends if you dismiss it as an action or if you cast it again.

You can also attack with the flame, although doing so ends the spell. When you cast this spell, or as an action on a later turn, you can hurl the flame at a creature within 30 feet of you. Make a ranged spell attack. On a hit, the target takes 1d8 fire damage.

\subparagraph*{At Higher Levels} This spell's damage increases by 1d8 when you reach 5th level (2d8), 11th level (3d8), and 17th level (4d8).

\DndSpellHeader
  {Shape Water}
  {Transmutation Cantrip}
  {1 Action}
  {30 feet}
  {S}
  {Instantaneous or 1 Hour}

You choose an area of water that you can see within range and that fits within a 5-foot cube. You manipulate it in one of the following ways:

\begin{itemize}
	\item You instantaneously move or otherwise change the flow of the water as you direct, up to 5 feet in any direction. This movement doesn’t have enough force to cause damage.
	\item You cause the water to form into simple shapes and animate at your direction. This change lasts for 1 hour.
	\item You change the water’s color or opacity. The water must be changed in the same way throughout. This change lasts for 1 hour.
	\item You freeze the water, provided that there are no creatures in it. The water unfreezes in 1 hour.
\end{itemize}
If you cast this spell multiple times, you can have no more than two of its non-instantaneous effects active at a time, and you can dismiss such an effect as an action.

\subsection*{Level 1}

\DndSpellHeader
  {Detect Magic}
  {1st-Level Divination (Ritual)}
  {1 Action}
  {Self}
  {V, S}
  {Concentration, up to 10 Minutes}

For the duration, you sense the presence of magic within 30 feet of you. If you sense magic in this way, you can use your action to see a faint aura around any visible creature or object in the area that bears magic, and you learn its school of magic, if any.

The spell can penetrate most barriers, but is blocked by 1 foot of stone, 1 inch of common metal, a thin sheet of lead, or 3 feet of wood or dirt.

\DndSpellHeader
  {Entangle}
  {1st-Level Conjuration}
  {1 Action}
  {90 feet}
  {V, S}
  {Concentration, up to 1 Minute}

Grasping weeds and vines sprout from the ground in a 20-foot square starting from a point within range. For the duration, these plants turn the ground in the area into difficult terrain.

A creature in the area when you cast the spell must succeed on a Strength saving throw or be restrained by the entangling plants until the spell ends. A creature restrained by the plants can use its action to make a Strength check against your spell save DC. On a success, it frees itself.

When the spell ends, the conjured plants wilt away.

\DndSpellHeader
  {Healing Word}
  {1st-Level Evocation}
  {1 Bonus Action}
  {60 feet}
  {V}
  {Instantaneous}

A creature of your choice that you can see within range regains hit points equal to 1d4 + your spellcasting ability modifier. This spell has no effect on undead or constructs.

\subparagraph*{At Higher Levels} When you cast this spell using a spell slot of 2nd level or higher, the healing increases by 1d4 for each slot level above 1st.

\subsection*{Level 2}

\DndSpellHeader
  {Blindness/Deafness}
  {2nd-Level Necromancy}
  {1 Action}
  {30 feet}
  {V}
  {1 Minute}

You can blind or deafen a foe. Choose one creature that you can see within range to make a Constitution saving throw. If it fails, the target is either blinded or deafened (your choice) for the duration. At the end of each of its turns, the target can make a Constitution saving throw. On a success, the spell ends.

\subparagraph*{At Higher Levels} When you cast this spell using a spell slot of 3rd level or higher, you can target one additional creature for each slot level above 2nd.

\DndSpellHeader
  {Gentle Repose}
  {2nd-Level Necromancy (Ritual)}
  {1 Action}
  {Touch}
  {V, S, M (a pinch of salt and one copper piece placed on each of the corpse’s eyes, which must remain there for the duration)}
  {10 Days}

You touch a corpse or other remains. For the duration, the target is protected from decay and can’t become undead.

The spell also effectively extends the time limit on raising the target from the dead, since days spent under the influence of this spell don’t count against the time limit of spells such as raise dead.

\DndSpellHeader
  {Augury}
  {2nd-Level Divination (Ritual)}
  {1 Minute}
  {Self}
  {V, S, M (specially marked sticks, bones, or similar tokens worth at least 25 gp)}
  {Instantaneous}

By casting gem-inlaid sticks, rolling dragon bones, laying out ornate cards, or employing some other divining tool, you receive an omen from an otherworldly entity about the results of a specific course of action that you plan to take within the next 30 minutes. The DM chooses from the following possible omens:
\begin{itemize}
	\item Weal, for good results
	\item Woe, for bad results
	\item Weal and woe, for both good and bad results
	\item Nothing, for results that aren’t especially good or bad
\end{itemize}
The spell doesn’t take into account any possible circumstances that might change the outcome, such as the casting of additional spells or the loss or gain of a companion. If you cast the spell two or more times before completing your next long rest, there is a cumulative 25 percent chance for each casting after the first that you get a random reading. The DM makes this roll in secret.

\DndSpellHeader
  {Darkvision}
  {2nd-Level Transmutation}
  {1 Action}
  {Touch}
  {V, S, M (either a pinch of dried carrot or an agate)}
  {8 Hours}

You touch a willing creature to grant it the ability to see in the dark. For the duration, that creature has darkvision out to a range of 60 feet.

\DndSpellHeader
  {Lesser Restoration}
  {2nd-Level Abjuration}
  {1 Action}
  {Touch}
  {V, S}
  {Instantaneous}

You touch a creature and can end either one disease or one condition afflicting it. The condition can be blinded, deafened, paralyzed, or poisoned.

\subsection*{Level 3}

\DndSpellHeader
  {Animate Dead}
  {3rd-Level Necromancy}
  {1 Minute}
  {10 feet}
  {V, S, M (a drop of blood, a piece of flesh, and a pinch of bone dust)}
  {Instantaneous}

This spell creates an undead servant. Choose a pile of bones or a corpse of a Medium or Small humanoid within range. Your spell imbues the target with a foul mimicry of life, raising it as an undead creature. The target becomes a skeleton if you chose bones or a zombie if you chose a corpse (the DM has the creature's game statistics).

On each of your turns, you can use a bonus action to mentally command any creature you made with this spell if the creature is within 60 feet of you (if you control multiple creatures, you can command any or all of them at the same time, issuing the same command to each one). You decide what action the creature will take and where it will move during its next turn, or you can issue a general command, such as to guard a particular chamber or corridor. If you issue no commands, the creature only defends itself against hostile creatures. Once given an order, the creature continues to follow it until its task is complete.

The creature is under your control for 24 hours, after which it stops obeying any command you've given it. To maintain the control of the creature for another 24 hours, you must cast this spell on the creature again before the current 24-hour period ends. This use of the spell reasserts your control over up to four creatures you have animated with this spell, rather than animating a new one.

\subparagraph*{At Higher Levels} When you cast this spell using a spell slot of 4th level or higher, you animate or reassert control over two additional undead creatures for each slot level above 3rd. Each of the creatures must come from a different corpse or pile of bones.

\DndSpellHeader
  {Gaseous Form}
  {3rd-Level Transmutation}
  {1 Action}
  {Touch}
  {V, S, M (a bit of gauze and a wisp of smoke)}
  {Concentration, Up to 1 Hour}

You transform a willing creature you touch, along with everything it’s wearing and carrying, into a misty cloud for the duration. The spell ends if the creature drops to 0 hit points. An incorporeal creature isn’t affected.

While in this form, the target’s only method of movement is a flying speed of 10 feet. The target can enter and occupy the space of another creature. The target has resistance to nonmagical damage, and it has advantage on Strength, Dexterity, and Constitution saving throws. The target can pass through small holes, narrow openings, and even mere cracks, though it treats liquids as though they were solid surfaces. The target can’t fall and remains hovering in the air even when stunned or otherwise incapacitated.

While in the form of a misty cloud, the target can’t talk or manipulate objects, and any objects it was carrying or holding can’t be dropped, used, or otherwise interacted with. The target can’t attack or cast spells.

\DndSpellHeader
  {Conjure Animals}
  {3rd-Level Conjuration}
  {1 Action}
  {60 feet}
  {V, S}
  {Concentration, Up to 1 Hour}

You summon fey spirits that take the form of beasts and appear in unoccupied spaces that you can see within range.

Choose one of the following options for what appears:

\begin{itemize}
	\item One beast of challenge rating 2 or lower
	\item Two beasts of challenge rating 1 or lower
	\item Four beasts of challenge rating 1/2 or lower
	\item Eight beasts of challenge rating 1/4 or lower
\end{itemize}

Each beast is also considered fey, and it disappears when it drops to 0 hit points or when the spell ends.

The summoned creatures are friendly to you and your companions. Roll initiative for the summoned creatures as a group, which has its own turns. They obey any verbal commands that you issue to them (no action required by you). If you don't issue any commands to them, they defend themselves from hostile creatures, but otherwise take no actions. The DM has the creatures' statistics.

\subparagraph*{At Higher Levels} When you cast this spell using certain higher-level spell slots, you choose one of the summoning options above, and more creatures appear: twice as many with a 5th-level slot, three times as many with a 7th-level slot, and four times as many with a 9th-level slot.

\DndSpellHeader
  {Dispel Magic}
  {3rd-Level Abjuration}
  {1 Action}
  {120 feet}
  {V, S}
  {Instantaneous}

Choose any creature, object, or magical effect within range. Any spell of 3rd level or lower on the target ends. For each spell of 4th level or higher on the target, make an ability check using your spellcasting ability. The DC equals 10 + the spell's level. On a successful check, the spell ends.

\subparagraph*{At Higher Levels} When you cast this spell using a spell slot of 4th level or higher, you automatically end the effects of a spell on the target if the spell's level is equal to or less than the level of the spell slot you used.

\section*{Magic Items}
\subsection*{Insignia of Claws}
While wearing the insignia, you gain a +1 bonus to the attack rolls and the damage rolls you make with unarmed strikes and natural weapons. Such attacks are considered to be magical.

\section*{Miscellaneous}
\subsection*{Attack and Damage Rolls}
\subsubsection*{Melee Weapons}
\paragraph*{Attack Roll}\hfill\\
\underline{\textit{Quarterstaff (Versatile):}}\\
1d20 + STR-Modifier + Proficiency Modifier\\
\indent Current Max: \intcalcAdd{20}{\intcalcAdd{\calculateModifier{\StrengthScoreValue}}{\ProficiencyValue}}
\paragraph*{Damage Roll}\hfill\\
\underline{\textit{Quarterstaff (Versatile):}}\\
1d6 (1d8) + STR-Modifier\\
\indent Current Max (one-handed): \intcalcAdd{6}{\calculateModifier{\StrengthScoreValue}}\\
\indent Current Max (two-handed): \intcalcAdd{8}{\calculateModifier{\StrengthScoreValue}}
\subsubsection*{Special Attacks}
\paragraph*{Attack Roll}\hfill\\
\underline{\textit{Unarmed Strike:}}\\
1d20 + STR-Modifier + Proficiency Modifier\\
\indent Current Max: \intcalcAdd{20}{\intcalcAdd{\calculateModifier{\StrengthScoreValue}}{\ProficiencyValue}}
\paragraph*{Damage Roll}\hfill\\
\underline{\textit{Unarmed Strike:}}\\
1 + STR-Modifier\\
\indent Current Max: \intcalcAdd{1}{\calculateModifier{\StrengthScoreValue}}
}