% Headline
\CharacterName{Maurice}

\Class{Druid}
\Level{7}
\Background{Hermit}
\PlayerName{}
\Race{Lemur o. M.}
\Alignment{Lawful Good}
\XP{}

% Ability scores (correct scores, no modifiers are automatically applied)
% Modifiers, Saving Throws and Skills are calculated automatically
\StrengthScore{8}
\DexterityScore{14}
\ConstitutionScore{12}
\IntelligenceScore{14}
\WisdomScore{16}
\CharismaScore{11}

% Proficiencies (Proficient = 'P', Expertise = 'E', otherwise = '')
\StrengthProficiency{}
\DexterityProficiency{}
\ConstitutionProficiency{}
\IntelligenceProficiency{P}
\WisdomProficiency{P}
\CharismaProficiency{}

\AcrobaticsProficiency{P}
\AnimalHandlingProficiency{}
\ArcanaProficiency{}
\AthleticsProficiency{}
\DeceptionProficiency{}
\HistoryProficiency{}
\InsightProficiency{P}
\IntimidationProficiency{}
\InvestigationProficiency{}
\MedicineProficiency{P}
\NatureProficiency{}
\PerceptionProficiency{P}
\PerformanceProficiency{P}
\PersuasionProficiency{}
\ReligionProficiency{P}
\SleightOfHandProficiency{}
\StealthProficiency{}
\SurvivalProficiency{}

\Inspiration{}
\Proficiency{+3}

% Armor Class is not automatically calculated
\ArmorClass{13}
\InitiativeModifier{0}
\Speed{30}
\MaxHitPointsRolled{40} % Without Constitution Bonus, is added automatically
\CurrentHitPoints{}
\TemporaryHitPoints{}
\HitDice{d8}
\HitDiceSpent{0}


\OtherProficienciesLanguages{
\textbf{Languages:}\\ Common, Druidic, Elvish \\
\textbf{Armor:} Light Armor, Medium Armor, Shields (won't wear armor or use shields made of metal)\\
\textbf{Weapons:} Clubs, Daggers, Darts, Javelins, Maces, Quarterstaff, Scimitars, Sickles, Slings, Spears \\
\textbf{Tools:}\\ Herbalism Kit
}

\PersonalityTraits{
	Maurice is often portrayed as the voice of reason and wisdom among the lemurs, offering thoughtful advice and guidance. He is known for his patience, often trying to calm the impulsive nature of King Julien and the other lemurs.
}

\Ideals{
	Maurice values stability and order, striving to maintain a sense of balance and calm in the chaotic world of the lemurs.
}

\Bonds{
	Maurice has a strong bond with King Julien, serving as his right-hand lemur and offering him guidance and support.
}

\Flaws{
	Maurice's cautious nature can lead to indecision and reluctance to take risks, potentially hindering progress.
}

\FeaturesTraits{
	\textbf{Lemur of Madagascar Traits}
	\begin{itemize}
		\item Like to MOVE IT!
		\item Stealth Sense
		\item Arboreal Movement
	\end{itemize}
	\textbf{Hermit}\\
	\textbf{Metamagic Adept}\\
	\textbf{Druid Traits}
	\begin{itemize}
		\item Wild Shape
		\item Druid Circle
		\begin{itemize}
			\item Circle of Spores
		\end{itemize}
	\end{itemize}
}

\CharacterAppearancePicture{\PATH one_shots/Penguins_of_Madagascar/characters/Maurice/images/Maurice.png}

% Magic

\SpellcastingClass{Druid}
\SpellcastingAbility{WIS} % STR, DEX, CON, INT, WIS, CHA
\SpellSaveDCModifier{0} % any modifier that isn't contained in "8 + Ability Modifier + Proficiency Bonus"

\CantripSlotA{Chill Touch (V, S)}
\CantripSlotB{Guidance (V, S)}
\CantripSlotC{Shape Water (S)}
\CantripSlotD{Thunderclap (S)}

\FirstLevelSpellSlotsTotal{4}
\FirstLevelSpellSlotA{Absorb Elements (S)}
\FirstLevelSpellSlotB{Detect Magic (V, S)}
\FirstLevelSpellSlotC{Entangle (V, S)}
\FirstLevelSpellSlotD{Faerie Fire (V)}
\FirstLevelSpellSlotE{Goodberry (V, S, M)}
\FirstLevelSpellSlotF{Healing Word (V)}

\FirstLevelSpellSlotBPrepared{True}
\FirstLevelSpellSlotCPrepared{True}
\FirstLevelSpellSlotFPrepared{True}

\SecondLevelSpellSlotsTotal{3}
\SecondLevelSpellSlotA{Blindness/Deafness (V)}
\SecondLevelSpellSlotB{Gentle Repose (V, S, M)}
\SecondLevelSpellSlotC{Augury (V, S, M)}
\SecondLevelSpellSlotD{Continual Flame (V, S, M)}
\SecondLevelSpellSlotE{Darkvision (V, S, M)}
\SecondLevelSpellSlotF{Enhance Ability (V, S, M)}
\SecondLevelSpellSlotG{Lesser Restoration (V, S)}
\SecondLevelSpellSlotH{Pass without Trace (V, S, M)}
\SecondLevelSpellSlotI{Spike Growth (V, S, M)}
\SecondLevelSpellSlotJ{Summon Beast (V, S, M)}

\SecondLevelSpellSlotAPrepared{True}
\SecondLevelSpellSlotBPrepared{True}
\SecondLevelSpellSlotCPrepared{True}
\SecondLevelSpellSlotGPrepared{True}
\SecondLevelSpellSlotJPrepared{True}

\ThirdLevelSpellSlotsTotal{3}
\ThirdLevelSpellSlotA{Animate Dead (V, S, M)}
\ThirdLevelSpellSlotB{Gaseous Form (V, S, M)}
\ThirdLevelSpellSlotC{Conjure Animals (V, S)}
\ThirdLevelSpellSlotD{Dispel Magic (V, S)}
\ThirdLevelSpellSlotE{Elemental Weapon (V, S)}
\ThirdLevelSpellSlotF{Revivify (V, S, M)}

\ThirdLevelSpellSlotAPrepared{True}
\ThirdLevelSpellSlotBPrepared{True}
\ThirdLevelSpellSlotDPrepared{True}
\ThirdLevelSpellSlotFPrepared{True}

\FourthLevelSpellSlotsTotal{1}
\FourthLevelSpellSlotA{Blight (V, S)}
\FourthLevelSpellSlotB{Confusion (V, S, M)}
\FourthLevelSpellSlotC{Dominate Beast (V, S)}
\FourthLevelSpellSlotD{Fire Shield (V, S, M)}
\FourthLevelSpellSlotE{Giant Insect (V, S)}
\FourthLevelSpellSlotF{Polymorph (V, S, M)}
\FourthLevelSpellSlotG{Stone Shape (V, S, M)}
\FourthLevelSpellSlotH{Wall of Fire (V, S, M)}

\FourthLevelSpellSlotAPrepared{True}
\FourthLevelSpellSlotBPrepared{True}
\FourthLevelSpellSlotDPrepared{True}
\FourthLevelSpellSlotHPrepared{True}

\renewcommand{\FeaturesAndSpells}{
\chapter*{Features, Magic Items and Spells}

\section*{Lemur of Madagascar Traits}
\subsection*{Like to MOVE IT!}
You gain proficiency in the Performance and Acrobatics skills. If you already have proficiency in those skills or gain these proficiency, you will gain expertise in those skills instead. You also gain advantage for Performance Skill rolls if performing in a group of size 3 or larger.
\subsection*{Stealth Sense}
When well-rested, you are able to sense that someone or something is in stealth but you are unable to pinpoint its' location if you are within 50 feet of it.
\subsection*{Arboreal Movement}
You have a climbing speed of 35 feet and roll with advantage on climbing/jumping tasks.

\section*{Metamagic Adept}
You've learned how to exert your will on your spells to alter how they function:
\begin{itemize}
	\item You learn two Metamagic options of your choice from the sorcerer class. You can use only one Metamagic option on a spell when you cast it, unless the option says otherwise. Whenever you reach a level that grants the Ability Score Improvement feature, you can replace one of these Metamagic options with another one from the sorcerer class.
	\item You gain 2 sorcery points to spend on Metamagic (these points are added to any sorcery points you have from another source but can be used only on Metamagic). You regain all spent sorcery points when you finish a long rest.
\end{itemize}
\subsection*{Quickened Spell}
When you cast a spell that has a casting time of 1 action, you can spend 2 sorcery points to change the casting time to 1 bonus action for this casting.
\subsection*{Subtle Spell}
When you cast a spell, you can spend 1 sorcery point to cast it without any somatic or verbal components.

\section*{Druid Traits}
Whether calling on the elemental forces of nature or emulating the creatures of the animal world, druids are an embodiment of nature's resilience, cunning, and fury. They claim no mastery over nature, but see themselves as extensions of nature's indomitable will.
\subsection*{Wild Shape}
Starting at 2nd level, you can use your action to magically assume the shape of a beast that you have seen before. You can use this feature twice. You regain expended uses when you finish a short or long rest.

Your druid level determines the beasts you can transform into, as shown in the Beast Shapes table. At 2nd level, for example, you can transform into any beast that has a challenge rating of 1/4 or lower that doesn't have a flying or swimming speed.
\begin{DndTable}[header=Beast Shapes]{llXl}
\textbf{Level}	& \textbf{Max. CR}  	&\textbf{Limitations}			&\textbf{Example}	\\
2nd 			& 1/4					&No flying or swimming speed	&Wolf				\\
4th 			& 1/2					&No flying speed				&Crocodile			\\
8th 			& 1						&								&Giant Eagle		\\
\end{DndTable}
You can stay in a beast shape for a number of hours equal to half your druid level (rounded down). You then revert to your normal form unless you expend another use of this feature. You can revert to your normal form earlier by using a bonus action on your turn. You automatically revert if you fall unconscious, drop to 0 hit points, or die.

While you are transformed, the following rules apply:
\begin{itemize}
	\item Your game statistics are replaced by the statistics of the beast, but you retain your alignment, personality, and Intelligence, Wisdom, and Charisma scores. You also retain all of your skill and saving throw proficiencies, in addition to gaining those of the creature. If the creature has the same proficiency as you and the bonus in its stat block is higher than yours, use the creature's bonus instead of yours. If the creature has any legendary or lair actions, you can't use them.
	\item When you transform, you assume the beast's hit points and Hit Dice. When you revert to your normal form, you return to the number of hit points you had before you transformed. However, if you revert as a result of dropping to 0 hit points, any excess damage carries over to your normal form, For example, if you take 10 damage in animal form and have only 1 hit point left, you revert and take 9 damage. As long as the excess damage doesn't reduce your normal form to 0 hit points, you aren't knocked unconscious.
	\item You can't cast spells, and your ability to speak or take any action that requires hands is limited to the capabilities of your beast form. Transforming doesn't break your concentration on a spell you've already cast, however, or prevent you from taking actions that are part of a spell, such as Call Lightning, that you've already cast.
	\item You retain the benefit of any features from your class, race, or other source and can use them if the new form is physically capable of doing so. However, you can't use any of your special senses, such as darkvision, unless your new form also has that sense.
	\item You choose whether your equipment falls to the ground in your space, merges into your new form, or is worn by it. Worn equipment functions as normal, but the DM decides whether it is practical for the new form to wear a piece of equipment, based on the creature's shape and size. Your equipment doesn't change size or shape to match the new form, and any equipment that the new form can't wear must either fall to the ground or merge with it. Equipment that merges with the form has no effect until you leave the form.
\end{itemize}
\subsection*{Druid Circle}
At 2nd level, you choose to identify with a circle of druids. Your choice grants you features at 2nd level and again at 6th, 10th, and 14th level.
\subsection*{Druid of Spores}
Druids of the Circle of Spores find beauty in decay. They see within mold and other fungi the ability to transform lifeless material into abundant, albeit somewhat strange, life. These druids believe that life and death are parts of a grand cycle, with one leading to the other and then back again. Death isn't the end of life, but instead a change of state that sees life shift into a new form.

Druids of this circle have a complex relationship with the undead. They see nothing inherently wrong with undeath, which they consider to be a companion to life and death. But these druids believe that the natural cycle is healthiest when each segment of it is vibrant and changing. Undead that seek to replace all life with undeath, or that try to avoid passing to a final rest, violate the cycle and must be thwarted.
\subsubsection*{Circle Spells}
Your symbiotic link to fungi and your ability to tap into the cycle of life and death grants you access to certain spells. At 2nd level, you learn the Chill Touch cantrip.

At 3rd, 5th, 7th, and 9th level you gain access to the spells listed for that level in the Circle of Spores Spells table. Once you gain access to one of these spells, you always have it prepared, and it doesn't count against the number of spells you can prepare each day. If you gain access to a spell that doesn't appear on the druid spell list, the spell is nonetheless a druid spell for you.
\begin{DndTable}[header=Circle of Spores Spells]{llX}
			& \textbf{Druid Level}  	&\textbf{Circle Spells}				\\
$\bullet$	& 2nd						&Chill Touch						\\
$\bullet$	& 3rd						&Blindness/Deafness, Gentle Repose	\\
$\bullet$	& 5th						&Animate Dead, Gaseous Form			\\
$\bullet$	& 7th						&Blight, Confusion					\\
			& 9th						&Cloudkill, Contagion				\\
\end{DndTable}
\subsubsection*{Halo of Spores}
Starting at 2nd level, you are surrounded by invisible, necrotic spores that are harmless until you unleash them on a creature nearby. When a creature you can see moves into a space within 10 feet of you or starts its turn there, you can use your reaction to deal 1d4 necrotic damage to that creature unless it succeeds on a Constitution saving throw against your spell save DC. The necrotic damage increases to 1d6 at 6th level, 1d8 at 10th level, and 1d10 at 14th level.
\subsubsection*{Symbiotic Entity}
Also at 2nd level, you gain the ability to channel magic into your spores. As an action, you can expend a use of your Wild Shape feature to awaken those spores, rather than transforming into a beast form, and you gain 4 temporary hit points for each level you have in this class. While this feature is active, you gain the following benefits:
\begin{itemize}
	\item When you deal your Halo of Spores damage, roll the damage die a second time and add it to the total.
	\item Your melee weapon attacks deal an extra 1d6 necrotic damage to any target they hit.
\end{itemize}
These benefits last for 10 minutes, until you lose all these temporary hit points or until you use your Wild Shape again.
\subsubsection*{Fungal Infestation}
At 6th level, your spores gain the ability to infest a corpse and animate it. If a beast or a humanoid that is Small or Medium dies within 10 feet of you, you can use your reaction to animate it, causing it to stand up immediately with 1 hit point. The creature uses the Zombie stat block in the Monster Manual. It remains animate for 1 hour, after which time it collapses and dies.

In combat, the zombie's turn comes immediately after yours. It obeys your mental commands, and the only action it can take is the Attack action, making one melee attack.

You can use this feature a number of times equal to your Wisdom modifier (minimum of once), and you regain all expended uses of it when you finish a long rest.

\section*{Spells}
\subsection*{Cantrip}

\DndSpellHeader
  {Chill Touch}
  {Necromancy Cantrip}
  {1 Action}
  {120 foot}
  {V, S}
  {1 Round}

You create a ghostly, skeletal hand in the space of a creature within range. Make a ranged spell attack against the creature to assail it with the chill of the grave. On a hit, the target takes 1d8 necrotic damage, and it can’t regain hit points until the start of your next turn. Until then, the hand clings to the target. If you hit an undead target, it also has disadvantage on attack rolls against you until the end of your next turn.

\subparagraph*{At Higher Levels} This spell’s damage increases by 1d8 when you reach 5th level (2d8), 11th level (3d8), and 17th level (4d8).

\DndSpellHeader
  {Guidance}
  {Divination Cantrip}
  {1 Action}
  {Touch}
  {V, S}
  {Concentration, up to 1 Minute}

You touch one willing creature. Once before the spell ends, the target can roll a d4 and add the number rolled to one ability check of its choice. It can roll the die before or after making the ability check. The spell then ends.

\DndSpellHeader
  {Shape Water}
  {Transmutation Cantrip}
  {1 Action}
  {30 feet}
  {S}
  {Instantaneous or 1 Hour}

You choose an area of water that you can see within range and that fits within a 5-foot cube. You manipulate it in one of the following ways:

\begin{itemize}
	\item You instantaneously move or otherwise change the flow of the water as you direct, up to 5 feet in any direction. This movement doesn’t have enough force to cause damage.
	\item You cause the water to form into simple shapes and animate at your direction. This change lasts for 1 hour.
	\item You change the water’s color or opacity. The water must be changed in the same way throughout. This change lasts for 1 hour.
	\item You freeze the water, provided that there are no creatures in it. The water unfreezes in 1 hour.
\end{itemize}
If you cast this spell multiple times, you can have no more than two of its non-instantaneous effects active at a time, and you can dismiss such an effect as an action.

\DndSpellHeader
  {Thunderclap}
  {Evocation Cantrip}
  {1 Action}
  {Self (5-foot Radius)}
  {S}
  {Instantaneous}

You create a burst of thunderous sound, which can be heard 100 feet away. Each creature other than you within 5 feet of you must make a Constitution saving throw. On a failed save, the creature takes 1d6 thunder damage.

\subparagraph*{At Higher Levels} The spell’s damage increases by 1d6 when you reach 5th level (2d6), 11th level (3d6), and 17th level (4d6).

\subsection*{Level 1}

\DndSpellHeader
  {Absorb Elements}
  {1st-Level Abjuration}
  {1 Reaction, which you take when you take acid, cold, fire, lightning, or thunder damage}
  {Self}
  {S}
  {1 Round}

The spell captures some of the incoming energy, lessening its effect on you and storing it for your next melee attack. You have resistance to the triggering damage type until the start of your next turn. Also, the first time you hit with a melee attack on your next turn, the target takes an extra 1d6 damage of the triggering type, and the spell ends.

\subparagraph*{At Higher Levels} When you cast this spell using a spell slot of 2nd level or higher, the extra damage increases by 1d6 for each slot level above 1st.

\DndSpellHeader
  {Detect Magic}
  {1st-Level Divination (Ritual)}
  {1 Action}
  {Self}
  {V, S}
  {Concentration, up to 10 Minutes}

For the duration, you sense the presence of magic within 30 feet of you. If you sense magic in this way, you can use your action to see a faint aura around any visible creature or object in the area that bears magic, and you learn its school of magic, if any.

The spell can penetrate most barriers, but is blocked by 1 foot of stone, 1 inch of common metal, a thin sheet of lead, or 3 feet of wood or dirt.

\DndSpellHeader
  {Entangle}
  {1st-Level Conjuration}
  {1 Action}
  {90 feet}
  {V, S}
  {Concentration, up to 1 Minute}

Grasping weeds and vines sprout from the ground in a 20-foot square starting from a point within range. For the duration, these plants turn the ground in the area into difficult terrain.

A creature in the area when you cast the spell must succeed on a Strength saving throw or be restrained by the entangling plants until the spell ends. A creature restrained by the plants can use its action to make a Strength check against your spell save DC. On a success, it frees itself.

When the spell ends, the conjured plants wilt away.

\DndSpellHeader
  {Fairie Fire}
  {1st-Level Evocation}
  {1 Action}
  {60 feet}
  {V}
  {Concentration, up to 1 Minute}

Each object in a 20-foot cube within range is outlined in blue, green, or violet light (your choice).

Any creature in the area when the spell is cast is also outlined in light if it fails a Dexterity saving throw. For the duration, objects and affected creatures shed dim light in a 10-foot radius.

Any attack roll against an affected creature or object has advantage if the attacker can see it, and the affected creature or object can’t benefit from being invisible.

\DndSpellHeader
  {Goodberry}
  {1st-Level Transmutation}
  {1 Action}
  {Touch}
  {V, S, M (a sprig of mistletoe)}
  {Instantaneous}

Up to ten berries appear in your hand and are infused with magic for the duration. A creature can use its action to eat one berry. Eating a berry restores 1 hit point, and the berry provides enough nourishment to sustain a creature for one day.

The berries lose their potency if they have not been consumed within 24 hours of the casting of this spell.

\DndSpellHeader
  {Healing Word}
  {1st-Level Evocation}
  {1 Bonus Action}
  {60 feet}
  {V}
  {Instantaneous}

A creature of your choice that you can see within range regains hit points equal to 1d4 + your spellcasting ability modifier. This spell has no effect on undead or constructs.

\subparagraph*{At Higher Levels} When you cast this spell using a spell slot of 2nd level or higher, the healing increases by 1d4 for each slot level above 1st.

\subsection*{Level 2}

\DndSpellHeader
  {Blindness/Deafness}
  {2nd-Level Necromancy}
  {1 Action}
  {30 feet}
  {V}
  {1 Minute}

You can blind or deafen a foe. Choose one creature that you can see within range to make a Constitution saving throw. If it fails, the target is either blinded or deafened (your choice) for the duration. At the end of each of its turns, the target can make a Constitution saving throw. On a success, the spell ends.

\subparagraph*{At Higher Levels} When you cast this spell using a spell slot of 3rd level or higher, you can target one additional creature for each slot level above 2nd.

\DndSpellHeader
  {Gentle Repose}
  {2nd-Level Necromancy (Ritual)}
  {1 Action}
  {Touch}
  {V, S, M (a pinch of salt and one copper piece placed on each of the corpse’s eyes, which must remain there for the duration)}
  {10 Days}

You touch a corpse or other remains. For the duration, the target is protected from decay and can’t become undead.

The spell also effectively extends the time limit on raising the target from the dead, since days spent under the influence of this spell don’t count against the time limit of spells such as raise dead.

\DndSpellHeader
  {Augury}
  {2nd-Level Divination (Ritual)}
  {1 Minute}
  {Self}
  {V, S, M (specially marked sticks, bones, or similar tokens worth at least 25 gp)}
  {Instantaneous}

By casting gem-inlaid sticks, rolling dragon bones, laying out ornate cards, or employing some other divining tool, you receive an omen from an otherworldly entity about the results of a specific course of action that you plan to take within the next 30 minutes. The DM chooses from the following possible omens:
\begin{itemize}
	\item Weal, for good results
	\item Woe, for bad results
	\item Weal and woe, for both good and bad results
	\item Nothing, for results that aren’t especially good or bad
\end{itemize}
The spell doesn’t take into account any possible circumstances that might change the outcome, such as the casting of additional spells or the loss or gain of a companion. If you cast the spell two or more times before completing your next long rest, there is a cumulative 25 percent chance for each casting after the first that you get a random reading. The DM makes this roll in secret.

\DndSpellHeader
  {Continual Flame}
  {2nd-Level Evocation}
  {1 Action}
  {Touch}
  {V, S, M (ruby dust worth 50 gp, which the spell consumes)}
  {Until dispelled}

A flame, equivalent in brightness to a torch, springs forth from an object that you touch. The effect looks like a regular flame, but it creates no heat and doesn’t use oxygen. A continual flame can be covered or hidden but not smothered or quenched.

\DndSpellHeader
  {Darkvision}
  {2nd-Level Transmutation}
  {1 Action}
  {Touch}
  {V, S, M (either a pinch of dried carrot or an agate)}
  {8 Hours}

You touch a willing creature to grant it the ability to see in the dark. For the duration, that creature has darkvision out to a range of 60 feet.

\DndSpellHeader
  {Enhance Ability}
  {2nd-Level Transmutation}
  {1 Action}
  {Touch}
  {V, S, M (fur or a feather from a beast)}
  {Concentration, up to 1 Hour}

You touch a creature and bestow upon it a magical enhancement. Choose one of the following effects; the target gains the effect until the spell ends.
\begin{itemize}
	\item \textbf{Bear’s Endurance.} The target has advantage on Constitution checks. It also gains 2d6 temporary hit points, which are lost when the spell ends.
	\item \textbf{Bull’s Strength.} The target has advantage on Strength checks, and their carrying capacity doubles.
	\item \textbf{Cat’s Grace.} The target has advantage on Dexterity checks. It also doesn’t take damage from falling 20 feet or less if it isn’t incapacitated.
	\item \textbf{Eagle’s Splendor.} The target has advantage on Charisma checks.
	\item \textbf{Fox’s Cunning.} The target has advantage on Intelligence checks.
	\item \textbf{Owl’s Wisdom.} The target has advantage on Wisdom checks.
\end{itemize}

\subparagraph*{At Higher Levels} When you cast this spell using a spell slot of 3rd level or higher, you can target one additional creature for each slot level above 2nd.

\DndSpellHeader
  {Lesser Restoration}
  {2nd-Level Abjuration}
  {1 Action}
  {Touch}
  {V, S}
  {Instantaneous}

You touch a creature and can end either one disease or one condition afflicting it. The condition can be blinded, deafened, paralyzed, or poisoned.

\DndSpellHeader
  {Pass Without Trace}
  {2nd-Level Abjuration}
  {1 Action}
  {Self}
  {V, S, M (ashes from a burned leaf of mistletoe and a sprig of spruce)}
  {Concentration, up to 1 Hour}

A veil of shadows and silence radiates from you, masking you and your companions from detection. For the duration, each creature you choose within 30 feet of you (including you) has a +10 bonus to Dexterity (Stealth) checks and can’t be tracked except by magical means. A creature that receives this bonus leaves behind no tracks or other traces of its passage.

\DndSpellHeader
  {Spike Growth}
  {2nd-Level Transmutation}
  {1 Action}
  {150 feet}
  {V, S, M (seven sharp thorns or seven small twigs, each sharpened to a point)}
  {Concentration, up to 10 Minutes}

The ground in a 20-foot radius centered on a point within range twists and sprouts hard spikes and thorns. The area becomes difficult terrain for the duration. When a creature moves into or within the area, it takes 2d4 piercing damage for every 5 feet it travels.

The transformation of the ground is camouflaged to look natural. Any creature that can’t see the area at the time the spell is cast must make a Wisdom (Perception) check against your spell save DC to recognize the terrain as hazardous before entering it.

\DndSpellHeader
  {Summon Beast}
  {2nd-Level Conjuration}
  {1 Action}
  {90 feet}
  {V, S, M (a feather, tuft of fur, and fish tail inside a gilded acorn worth at least 200 gp)}
  {Concentration, up to 1 Hour}

You call forth a bestial spirit. It manifests in an unoccupied space that you can see within range. This corporeal form uses the Bestial Spirit stat block. When you cast the spell, choose an environment: Air, Land, or Water. The creature resembles an animal of your choice that is native to the chosen environment, which determines certain traits in its stat block. The creature disappears when it drops to 0 hit points or when the spell ends.

The creature is an ally to you and your companions. In combat, the creature shares your initiative count, but it takes its turn immediately after yours. It obeys your verbal commands (no action required by you). If you don’t issue any, it takes the Dodge action and uses its move to avoid danger.

\subparagraph*{At Higher Levels} When you cast this spell using a spell slot of 3rd level or higher, use the higher level where the spell’s level appears in the stat block.

\begin{DndMonster}[width=0.5\textwidth]{Bestial Spirit}
    \DndMonsterType{Small Beast}

    % If you want to use commas in the key values, enclose the values in braces.
    \DndMonsterBasics[
        armor-class = {11 + the level of the spell (natural armor)},
        hit-points  = {20 (Air only) or 30 (Land and Water only) + 5 for each spell level above 2nd},
        speed       = {30 ft., climb 30 ft. (Land only), fly 60 ft. (Air only), swim 30 ft. (Water only)},
    ]
    
	\renewcommand{\AbilityScoreSpacer}{~}
    \DndMonsterAbilityScores[
		str = 18,
		dex = 11,
		con = 16,
		int = 4,
		wis = 14,
		cha = 5,
    ]

    \DndMonsterDetails[
        %saving-throws = {STR +3 + PB, CON +3 + PB},
        %skills = {Athletics +3 + PB, Intimidation +1 + (2 x PB)},
        %damage-vulnerabilities = {cold},
        %damage-resistances = {bludgeoning, piercing, and slashing from nonmagical attacks},
        %damage-immunities = {poison},
        senses = {Darkvision 60 ft., Passive Perception 12},
        %condition-immunities = {prone},
        languages = {understands the languages you speak},
        challenge = 1,
    ]
    
    \DndMonsterAction{Flyby (Air Only)}
    The beast doesn’t provoke opportunity attacks when it flies out of an enemy’s reach.
    
    \DndMonsterAction{Pack Tactics (Land and Water Only)}
    The beast has advantage on an attack roll against a creature if at least one of the beast’s allies is within 5 feet of the creature and the ally isn’t incapacitated.
    
    \DndMonsterAction{Water Breathing (Water Only)}
    The beast can breathe only underwater.
	
	\DndMonsterSection{Actions}
	\DndMonsterAction{Multiattack}
	The beast makes a number of attacks equal to half this spell’s level (rounded down).
	
	\DndMonsterAttack[
      name=Maul,
      distance=melee, % valid options are in the set {both,melee,ranged},
      %type=weapon, %valid options are in the set {weapon,spell}
      mod=your spell attack modifier,
      reach=5,
      %range=30,
      targets=one target,
      dmg=\DndDice{1d8} + 4 + the spell's level,
      dmg-type=piercing,
      %plus-dmg=,
      %plus-dmg-type=,
      %or-dmg=,
      %or-dmg-when=,
      %extra=,
    ]
\end{DndMonster}

\subsection*{Level 3}

\DndSpellHeader
  {Animate Dead}
  {3rd-Level Necromancy}
  {1 Minute}
  {10 feet}
  {V, S, M (a drop of blood, a piece of flesh, and a pinch of bone dust)}
  {Instantaneous}

This spell creates an undead servant. Choose a pile of bones or a corpse of a Medium or Small humanoid within range. Your spell imbues the target with a foul mimicry of life, raising it as an undead creature. The target becomes a skeleton if you chose bones or a zombie if you chose a corpse (the DM has the creature’s game statistics).

On each of your turns, you can use a bonus action to mentally command any creature you made with this spell if the creature is within 60 feet of you (if you control multiple creatures, you can command any or all of them at the same time, issuing the same command to each one). You decide what action the creature will take and where it will move during its next turn, or you can issue a general command, such as to guard a particular chamber or corridor. If you issue no commands, the creature only defends itself against hostile creatures. Once given an order, the creature continues to follow it until its task is complete.

The creature is under your control for 24 hours, after which it stops obeying any command you’ve given it. To maintain the control of the creature for another 24 hours, you must cast this spell on the creature again before the current 24-hour period ends. This use of the spell reasserts your control over up to four creatures you have animated with this spell, rather than animating a new one.

\subparagraph*{At Higher Levels} When you cast this spell using a spell slot of 4th level or higher, you animate or reassert control over two additional undead creatures for each slot level above 3rd. Each of the creatures must come from a different corpse or pile of bones.

\DndSpellHeader
  {Gaseous Form}
  {3rd-Level Transmutation}
  {1 Action}
  {Touch}
  {V, S, M (a bit of gauze and a wisp of smoke)}
  {Concentration, up to 1 Hour}

You transform a willing creature you touch, along with everything it’s wearing and carrying, into a misty cloud for the duration. The spell ends if the creature drops to 0 hit points. An incorporeal creature isn’t affected.

While in this form, the target’s only method of movement is a flying speed of 10 feet. The target can enter and occupy the space of another creature. The target has resistance to nonmagical damage, and it has advantage on Strength, Dexterity, and Constitution saving throws. The target can pass through small holes, narrow openings, and even mere cracks, though it treats liquids as though they were solid surfaces. The target can’t fall and remains hovering in the air even when stunned or otherwise incapacitated.

While in the form of a misty cloud, the target can’t talk or manipulate objects, and any objects it was carrying or holding can’t be dropped, used, or otherwise interacted with. The target can’t attack or cast spells.

\DndSpellHeader
  {Conjure Animals}
  {3rd-Level Conjuration}
  {1 Action}
  {60 feet}
  {V, S}
  {Concentration, up to 1 Hour}

You summon fey spirits that take the form of beasts and appear in unoccupied spaces that you can see within range.

Choose one of the following options for what appears:
\begin{itemize}
	\item One beast of challenge rating 2 or lower
	\item Two beasts of challenge rating 1 or lower
	\item Four beasts of challenge rating 1/2 or lower
	\item Eight beasts of challenge rating 1/4 or lower
\end{itemize}
Each beast is also considered fey, and it disappears when it drops to 0 hit points or when the spell ends.

The summoned creatures are friendly to you and your companions. Roll initiative for the summoned creatures as a group, which has its own turns. They obey any verbal commands that you issue to them (no action required by you). If you don’t issue any commands to them, they defend themselves from hostile creatures, but otherwise take no actions. The DM has the creatures’ statistics.

\subparagraph*{At Higher Levels} When you cast this spell using certain higher-level spell slots, you choose one of the summoning options above, and more creatures appear: twice as many with a 5th-level slot, three times as many with a 7th-level slot, and four times as many with a 9th-level slot.

\DndSpellHeader
  {Dispel Magic}
  {3rd-Level Abjuration}
  {1 Action}
  {120 feet}
  {V, S}
  {Instantaneous}

Choose any creature, object, or magical effect within range. Any spell of 3rd level or lower on the target ends. For each spell of 4th level or higher on the target, make an ability check using your spellcasting ability. The DC equals 10 + the spell's level. On a successful check, the spell ends.

\subparagraph*{At Higher Levels} When you cast this spell using a spell slot of 4th level or higher, you automatically end the effects of a spell on the target if the spell's level is equal to or less than the level of the spell slot you used.

\DndSpellHeader
  {Elemental Weapon}
  {3rd-Level Transmutation}
  {1 Action}
  {Touch}
  {V, S}
  {Concentration, up to 1 Hour}

A nonmagical weapon you touch becomes a magic weapon. Choose one of the following damage types: acid, cold, fire, lightning, or thunder. For the duration, the weapon has a +1 bonus to attack rolls and deals an extra 1d4 damage of the chosen type when it hits.

\subparagraph*{At Higher Levels} When you cast this spell using a spell slot of 5th or 6th level, the bonus to attack rolls increases to +2 and the extra damage increases to 2d4. When you use a spell slot of 7th level or higher, the bonus increases to +3 and the extra damage increases to 3d4.

\DndSpellHeader
  {Revivify}
  {3rd-Level Necromancy}
  {1 Action}
  {Touch}
  {V, S, M (diamonds worth 300 gp, which the spell consumes)}
  {Instantaneous}

You touch a creature that has died within the last minute. That creature returns to life with 1 hit point. This spell can’t return to life a creature that has died of old age, nor can it restore any missing body parts.

\subsection*{Level 4}

\DndSpellHeader
  {Blight}
  {4th-Level Necromancy}
  {1 Action}
  {30 feet}
  {V, S}
  {Instaneous}

Necromantic energy washes over a creature of your choice that you can see within range, draining moisture and vitality from it. The target must make a Constitution saving throw. The target takes 8d8 necrotic damage on a failed save, or half as much damage on a successful one. This spell has no effect on undead or constructs.

If you target a plant creature or a magical plant, it makes the saving throw with disadvantage, and the spell deals maximum damage to it. If you target a nonmagical plant that isn’t a creature, such as a tree or shrub, it doesn’t make a saving throw; it simply withers and dies.

\subparagraph*{At Higher Levels} When you cast this spell using a spell slot of 5th level or higher, the damage increases by 1d8 for each slot level above 4th.

\DndSpellHeader
  {Confusion}
  {4th-Level Enchantment}
  {1 Action}
  {90 feet}
  {V, S, M (three nut shells)}
  {Concentration, up to 1 Minute}

This spell assaults and twists creatures’ minds, spawning delusions and provoking uncontrolled actions. Each creature in a 10-foot-radius sphere centered on a point you choose within range must succeed on a Wisdom saving throw when you cast this spell or be affected by it.

An affected target can’t take reactions and must roll a d10 at the start of each of its turns to determine its behavior for that turn.
\begin{DndTable}[]{lX}
\textbf{d10}  	&\textbf{Behavior}\\
1				&The creature uses all its movement to move in a random direction. To determine the direction, roll a d8 and assign a direction to each die face. The creature doesn’t take an action this turn.\\
2-6				&The creature doesn’t move or take actions this turn.\\
7-8				&The creature uses its action to make a melee attack against a randomly determined creature within its reach. If there is no creature within its reach, the creature does nothing this turn.\\
9-10			&The creature can act and move normally.\\
\end{DndTable}
At the end of its turns, an affected target can make a Wisdom saving throw. If it succeeds, this effect ends for that target.

\subparagraph*{At Higher Levels} When you cast this spell using a spell slot of 5th level or higher, the radius of the sphere increases by 5 feet for each slot level above 4th.

\DndSpellHeader
  {Dominate Beast}
  {4th-Level Enchantment}
  {1 Action}
  {60 feet}
  {V, S}
  {Concentration, up to 1 Minute}

You attempt to beguile a beast that you can see within range. It must succeed on a Wisdom saving throw or be charmed by you for the duration. If you or creatures that are friendly to you are fighting it, it has advantage on the saving throw.

While the beast is charmed, you have a telepathic link with it as long as the two of you are on the same plane of existence. You can use this telepathic link to issue commands to the creature while you are conscious (no action required), which it does its best to obey. You can specify a simple and general course of action, such as “Attack that creature,” “Run over there,” or “Fetch that object.” If the creature completes the order and doesn’t receive further direction from you, it defends and preserves itself to the best of its ability.

You can use your action to take total and precise control of the target. Until the end of your next turn, the creature takes only the actions you choose, and doesn’t do anything that you don’t allow it to do. During this time, you can also cause the creature to use a reaction, but this requires you to use your own reaction as well.

Each time the target takes damage, it makes a new Wisdom saving throw against the spell. If the saving throw succeeds, the spell ends.

\subparagraph*{At Higher Levels} When you cast this spell with a 5th-level spell slot, the duration is concentration, up to 10 minutes. When you use a 6th-level spell slot, the duration is concentration, up to 1 hour. When you use a spell slot of 7th level or higher, the duration is concentration, up to 8 hours.

\DndSpellHeader
  {Fire Shield}
  {4th-Level Evocation}
  {1 Action}
  {Self}
  {V, S, M (a bit of phosphorus or a firefly)}
  {10 Minutes}

Thin and wispy flames wreathe your body for the duration, shedding bright light in a 10-foot radius and dim light for an additional 10 feet. You can end the spell early by using an action to dismiss it.

The flames provide you with a warm shield or a chill shield, as you choose. The warm shield grants you resistance to cold damage, and the chill shield grants you resistance to fire damage.

In addition, whenever a creature within 5 feet of you hits you with a melee attack, the shield erupts with flame. The attacker takes 2d8 fire damage from a warm shield, or 2d8 cold damage from a cold shield.

\DndSpellHeader
  {Giant Insect}
  {4th-Level Transmutation}
  {1 Action}
  {30 feet}
  {V, S}
  {Concentration, up to 10 Minutes}

You transform up to ten centipedes, three spiders, five wasps, or one scorpion within range into giant versions of their natural forms for the duration. A centipede becomes a giant centipede, a spider becaomes a giant spider, a wasp becomes a giant wasp, and a scorpion becomes a giant scorpion.

Each creature obeys your verbal commands, and in combat, they act on your turn each round. The DM has the statistics for these creatures and resolves their actions and movement.

A creature remains in its giant size for the duration, until it drops to 0 hit points, or until you use an action to dismiss the effect on it.

The DM might allow you to choose different targets. For example, if you transform a bee, its giant version might have the same statistics as a giant wasp.

\DndSpellHeader
  {Polymorph}
  {4th-Level Transmutation}
  {1 Action}
  {60 feet}
  {V, S, M (a caterpillar cocoon)}
  {Concentration, up to 1 Hour}

This spell transforms a creature that you can see within range into a new form. An unwilling creature must make a Wisdom saving throw to avoid the effect. A shapechanger automatically succeeds on this saving throw.

The transformation lasts for the duration, or until the target drops to 0 hit points or dies. The new form can be any beast whose challenge rating is equal to or less than the target’s (or the target’s level, if it doesn’t have a challenge rating). The target’s game statistics, including mental ability scores, are replaced by the statistics of the chosen beast. It retains its alignment and personality.

The target assumes the hit points of its new form. When it reverts to its normal form, the creature returns to the number of hit points it had before it transformed. If it reverts as a result of dropping to 0 hit points, any excess damage carries over to its normal form. As long as the excess damage doesn’t reduce the creature’s normal form to 0 hit points, it isn’t knocked unconscious.

The creature is limited in the actions it can perform by the nature of its new form, and it can’t speak, cast spells, or take any other action that requires hands or speech.

The target’s gear melds into the new form. The creature can’t activate, use, wield, or otherwise benefit from any of its equipment. This spell can’t affect a target that has 0 hit points.

\DndSpellHeader
  {Stone Shape}
  {4th-Level Transmutation}
  {1 Action}
  {Touch}
  {V, S, M (soft clay, which must be worked into roughly the desired shape of the stone object)}
  {Instantaneous}

You touch a stone object of Medium size or smaller or a section of stone no more than 5 feet in any dimension and form it into any shape that suits your purpose. So, for example, you could shape a large rock into a weapon, idol, or coffer, or make a small passage through a wall, as long as the wall is less than 5 feet thick. You could also shape a stone door or its frame to seal the door shut. The object you create can have up to two hinges and a latch, but finer mechanical detail isn’t possible.

\DndSpellHeader
  {Wall of Fire}
  {4th-Level Evocation}
  {1 Action}
  {120 feet}
  {V, S, M (a small piece of phosphorus)}
  {Concentration, up to 1 Minute}

You create a wall of fire on a solid surface within range. You can make the wall up to 60 feet long, 20 feet high, and 1 foot thick, or a ringed wall up to 20 feet in diameter, 20 feet high, and 1 foot thick. The wall is opaque and lasts for the duration.

When the wall appears, each creature within its area must make a Dexterity saving throw. On a failed save, a creature takes 5d8 fire damage, or half as much damage on a successful save.

One side of the wall, selected by you when you cast this spell, deals 5d8 fire damage to each creature that ends its turn within 10 feet of that side or inside the wall. A creature takes the same damage when it enters the wall for the first time on a turn or ends its turn there. The other side of the wall deals no damage.

\subparagraph*{At Higher Levels} When you cast this spell using a spell slot of 5th level or higher, the damage increases by 1d8 for each slot level above 4th.
}