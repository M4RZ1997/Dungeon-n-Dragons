% Headline
\CharacterName{King Julien}

\Class{Ranger}
\Level{7}
\Background{Entertainer}
\PlayerName{}
\Race{Lemur o. M.}
\Alignment{Chaotic Neutral}
\XP{}

% Ability scores (correct scores, no modifiers are automatically applied)
% Modifiers, Saving Throws and Skills are calculated automatically
\StrengthScore{7}
\DexterityScore{17}
\ConstitutionScore{13}
\IntelligenceScore{9}
\WisdomScore{14}
\CharismaScore{14}

% Proficiencies (Proficient = 'P', Expertise = 'E', otherwise = '')
\StrengthProficiency{P}
\DexterityProficiency{P}
\ConstitutionProficiency{}
\IntelligenceProficiency{}
\WisdomProficiency{}
\CharismaProficiency{}

\AcrobaticsProficiency{E}
\AnimalHandlingProficiency{P}
\ArcanaProficiency{}
\AthleticsProficiency{}
\DeceptionProficiency{}
\HistoryProficiency{}
\InsightProficiency{}
\IntimidationProficiency{}
\InvestigationProficiency{}
\MedicineProficiency{}
\NatureProficiency{}
\PerceptionProficiency{P}
\PerformanceProficiency{E}
\PersuasionProficiency{}
\ReligionProficiency{}
\SleightOfHandProficiency{}
\StealthProficiency{}
\SurvivalProficiency{P}

\Inspiration{}
\Proficiency{+3}

% Armor Class is not automatically calculated
\ArmorClass{13}
\InitiativeModifier{0}
\Speed{30}
\MaxHitPointsRolled{41} % Without Constitution Bonus, is added automatically
\CurrentHitPoints{}
\TemporaryHitPoints{}
\HitDice{d10}
\HitDiceSpent{0}


\OtherProficienciesLanguages{
\textbf{Languages:}\\ Common \\
\textbf{Armor:}\\ Light Armor, Medium Armor, Shields\\
\textbf{Weapons:}\\ Simple Weapons, Martial Weapons \\
\textbf{Tools:}\\  Lute
}

\PersonalityTraits{
	King Julien is known for his eccentric and flamboyant personality, making him stand out in any crowd. He often prioritizes his own needs and desires, believing that he should be the center of attention.
}

\Ideals{
	King Julien values pleasure, indulgence, and living life to the fullest. He often seeks enjoyment and avoids anything that might be unpleasant.
}

\Bonds{
	He is constantly seeking the adoration and admiration of others, and has a similar strong desire for recognition.
}

\Flaws{
	Julien's self-centered nature can lead to narcissism, making him overly concerned with his own appearance and desires.
}

\FeaturesTraits{
	\textbf{Lemur of Madagascar Traits}
	\begin{itemize}
		\item Like to MOVE IT!
		\item Stealth Sense
		\item Arboreal Movement
	\end{itemize}
	\textbf{Entertainer}\\
	\textbf{Crossbow Expert}\\
	\textbf{Sharpshooter}\\
	\textbf{Ranger Traits}
	\begin{itemize}
		\item Favored Enemy
		\item Natural Explorer
		\item Fighting Style
		\begin{itemize}
			\item Archery
		\end{itemize}
		\item Beast Master
		\begin{itemize}
			\item Primal Companion (Mort)
			\item Exceptional Training
		\end{itemize}
		\item Primal Awareness
		\item Extra Attack
	\end{itemize}
}

\CharacterAppearancePicture{\PATH one_shots/Penguins_of_Madagascar/operation_Ring_of_Fire/characters/King_Julien/images/King_Julien.png}

% Magic

\SpellcastingClass{Ranger}
\SpellcastingAbility{WIS} % STR, DEX, CON, INT, WIS, CHA
\SpellSaveDCModifier{0} % any modifier that isn't contained in "8 + Ability Modifier + Proficiency Bonus"

\FirstLevelSpellSlotsTotal{4}
\FirstLevelSpellSlotA{Speak with Animals (V, S)}
\FirstLevelSpellSlotB{Absorb Elements (S)}
\FirstLevelSpellSlotC{Entangle (V, S)}
\FirstLevelSpellSlotD{Hunter's Mark (V)}

\SecondLevelSpellSlotsTotal{3}
\SecondLevelSpellSlotA{Beast Sense (S)}
\SecondLevelSpellSlotB{Pass Without Trace (V, S, M)}
\SecondLevelSpellSlotC{Spike Growth (V, S, M)}

\renewcommand{\FeaturesAndSpells}{
\chapter*{Features, Magic Items and Spells}

\section*{Lemur of Madagascar Traits}
\subsection*{Like to MOVE IT!}
You gain proficiency in the Performance and Acrobatics skills. If you already have proficiency in those skills or gain these proficiency, you will gain expertise in those skills instead. You also gain advantage for Performance Skill rolls if performing in a group of size 3 or larger.
\subsection*{Stealth Sense}
When well-rested, you are able to sense that someone or something is in stealth but you are unable to pinpoint its' location if you are within 50 feet of it.
\subsection*{Arboreal Movement}
You have a climbing speed of 35 feet and roll with advantage on climbing/jumping tasks.

\section*{Entertainer}
\textbf{Dancer}\\
You thrive in front of an audience. You know how to entrance them, entertain them, and even inspire them. Your poetics can stir the hearts of those who hear you, awakening grief or joy, laughter or anger. Your music raises their spirits or captures their sorrow. Your dance steps captivate, your humor cuts to the quick. Whatever techniques you use, your art is your life.
\subsection*{By Popular Demand}
You can always find a place to perform, usually in an inn or tavern but possibly with a circus, at a theater, or even in a noble's court. At such a place, you receive free lodging and food of a modest or comfortable standard (depending on the quality of the establishment), as long as you perform each night. In addition, your performance makes you something of a local figure. When strangers recognize you in a town where you have performed, they typically take a liking to you.

\section*{Crossbow Expert}
Thanks to extensive practice with the crossbow, you gain the following benefits:
\begin{itemize}
	\item You ignore the loading quality of crossbows with which you are proficient.
	\item Being within 5 feet of a hostile creature doesn't impose disadvantage on your ranged attack rolls.
	\item When you use the Attack action and attack with a one handed weapon, you can use a bonus action to attack with a hand crossbow you are holding.
\end{itemize}

\section*{Sharpshooter}
You have mastered ranged weapons and can make shots that others find impossible. You gain the following benefits:
\begin{itemize}
	\item Attacking at long range doesn't impose disadvantage on your ranged weapon attack rolls.
	\item Your ranged weapon attacks ignore half and three-quarters cover.
	\item Before you make an attack with a ranged weapon that you are proficient with, you can choose to take a -5 penalty to the attack roll. If that attack hits, you add +10 to the attack's damage.
\end{itemize}

\section*{Ranger Traits}
\subsection*{Favored Enemy}
\textbf{(Plants, Monstrosities)}\\
Beginning at 1st level, you have significant experience studying, tracking, hunting, and even talking to a certain type of enemy.

Choose a type of favored enemy: aberrations, beasts, celestials, constructs, dragons, elementals, fey, fiends, giants, monstrosities, oozes, plants, or undead. Alternatively, you can select two races of humanoid (such as gnolls and orcs) as favored enemies.

You have advantage on Wisdom (Survival) checks to track your favored enemies, as well as on Intelligence checks to recall information about them.

When you gain this feature, you also learn one language of your choice that is spoken by your favored enemies, if they speak one at all.

You choose one additional favored enemy, as well as an associated language, at 6th and 14th level. As you gain levels, your choices should reflect the types of monsters you have encountered on your adventures.
\subsection*{Natural Explorer}
\textbf{(Forest, Grassland)}\\
Also at 1st level, you are particularly familiar with one type of natural environment and are adept at traveling and surviving in such regions. Choose one type of favored terrain: arctic, coast, desert, forest, grassland, mountain, swamp, or the Underdark. When you make an Intelligence or Wisdom check related to your favored terrain, your proficiency bonus is doubled if you are using a skill that you’re proficient in.

While traveling for an hour or more in your favored terrain, you gain the following benefits:
\begin{itemize}
	\item Difficult terrain doesn’t slow your group’s travel.
	\item Your group can’t become lost except by magical means.
	\item Even when you are engaged in another activity while traveling (such as foraging, navigating, or tracking), you remain alert to danger.
	\item If you are traveling alone, you can move stealthily at a normal pace.
	\item When you forage, you find twice as much food as you normally would.
	\item While tracking other creatures, you also learn their exact number, their sizes, and how long ago they passed through the area.
\end{itemize}
You choose additional favored terrain types at 6th and 10th level.
\subsection*{Fighting Style}
At 2nd level, you adopt a particular style of fighting as your specialty. Choose one of the following options. You can't take a Fighting Style option more than once, even if you later get to choose again.
\subsubsection*{Archery}
You gain a +2 bonus to attack rolls you make with ranged weapons.
\subsection*{Beast Master}
The Beast Master archetype embodies a friendship between the civilized races and the beasts of the world. United in focus, beast and ranger work as one to fight the monstrous foes that threaten civilization and the wilderness alike. Emulating the Beast Master archetype means committing yourself to this ideal, working in partnership with an animal as its companion and friend.
\subsubsection*{Primal Companion}
You magically summon a primal beast, which draws strength from your bond with nature. The beast is friendly to you and your companions and obeys your commands. Choose its stat block-Beast of the Land, Beast of the Sea, or Beast of the Sky-which uses your proficiency bonus (PB) in several places. You also determine the kind of animal the beast is, choosing a kind appropriate for the stat block. Whatever kind you choose, the beast bears primal markings, indicating its mystical origin.

In combat, the beast acts during your turn. It can move and use its reaction on its own, but the only action it takes is the Dodge action, unless you take a bonus action on your turn to command it to take another action. That action can be one in its stat block or some other action. You can also sacrifice one of your attacks when you take the Attack action to command the beast to take the Attack action. If
you are incapacitated, the beast can take any action of its choice, not just Dodge.

If the beast has died within the last hour, you can use your action to touch it and expend a spell slot of 1st level or higher. The beast returns to life after 1 minute with all its hit points restored. When you finish a long rest, you can summon a different primal beast. The new beast appears in an unoccupied space within 5 feet of you, and you choose its stat block and appearance. If you already have a beast from this feature, it vanishes when the new beast appears. The beast also vanishes if you die.
\begin{DndMonster}[width=0.5\textwidth]{Mort}
    \DndMonsterType{Tiny Beast}

    % If you want to use commas in the key values, enclose the values in braces.
    \DndMonsterBasics[
        armor-class = {13 + PB (Natural Armor)},
        hit-points  = {\intcalcMul{5}{\LevelValue} (\LevelValue d8)},
        speed       = {25 ft., climb 35 ft.},
    ]
    
	\renewcommand{\AbilityScoreSpacer}{~}
    \DndMonsterAbilityScores[
		str = 6,
		dex = 16,
		con = 10,
		int = 8,
		wis = 8,
		cha = 16,
    ]

    \DndMonsterDetails[
        saving-throws = {Dex +3 + PB},
        skills = {Stealth +3 + (2 x PB), Deception +1 + PB},
        %damage-vulnerabilities = {cold},
        %damage-resistances = {bludgeoning, piercing, and slashing from nonmagical attacks},
        %damage-immunities = {poison},
        senses = {Darkvision 60 ft., Passive Perception 12},
        %condition-immunities = {poisoned},
        languages = {understands the languages you speak},
        challenge = 1,
    ]
    
    \DndMonsterAction{Buff-Up}
    Mort becomes the Buffed-Up version of himself for 1 minute. See the Buffed-Up Mort statblock. Mort can use this Action once per long rest.
    
    \DndMonsterAction{Primal Bond}
    You can add your proficiency bonus to any ability check or saving throw that the beast makes.
	
	\DndMonsterSection{Actions}	
	\DndMonsterAttack[
      name=Pounce,
      distance=melee, % valid options are in the set {both,melee,ranged},
      %type=weapon, %valid options are in the set {weapon,spell}
      mod=\calculateSpellAttack{\calculateModifier{\WisdomScoreValue}},
      reach=5,
      %range=30,
      targets=one target,
      dmg=\DndDice{1d4} - 2 + PB,
      dmg-type=bludgeoning,
      %plus-dmg=,
      %plus-dmg-type=,
      %or-dmg=,
      %or-dmg-when=,
      %extra=,
    ]	
\end{DndMonster}
\begin{tikzpicture}[remember picture, overlay]
	\node[opacity=1,inner sep=0pt, xshift=-2.9cm, yshift=-2.1cm] at (current page.north east){\includegraphics[width=0.25\paperwidth, keepaspectratio]{\PATH one_shots/Penguins_of_Madagascar/operation_Ring_of_Fire/characters/King_Julien/images/Mort.png}};
	\node[opacity=1,inner sep=0pt, xshift=-2.9cm, yshift=-7cm] at (current page.north east){\includegraphics[width=0.25\paperwidth, keepaspectratio]{\PATH one_shots/Penguins_of_Madagascar/operation_Ring_of_Fire/characters/King_Julien/images/Buffed_Mort.png}};
\end{tikzpicture}
\begin{DndMonster}[width=0.5\textwidth]{Buffed-Up Mort}
    \DndMonsterType{Medium Beast}

    % If you want to use commas in the key values, enclose the values in braces.
    \DndMonsterBasics[
        armor-class = {14 + PB (Natural Armor)},
        hit-points  = {\intcalcMul{5}{\LevelValue} (\LevelValue d8) + \intcalcMul{4}{\LevelValue} Temporary},
        speed       = {30 ft.},
    ]
    
	\renewcommand{\AbilityScoreSpacer}{~}
    \DndMonsterAbilityScores[
		str = 16,
		dex = 8,
		con = 16,
		int = 6,
		wis = 8,
		cha = 12,
    ]

    \DndMonsterDetails[
        saving-throws = {STR +3 + PB, CON +3 + PB},
        skills = {Athletics +3 + PB, Intimidation +1 + (2 x PB)},
        %damage-vulnerabilities = {cold},
        damage-resistances = {bludgeoning, piercing, and slashing from nonmagical attacks},
        %damage-immunities = {poison},
        senses = {Darkvision 60 ft., Passive Perception 12},
        condition-immunities = {prone},
        languages = {understands the languages you speak},
        challenge = 1,
    ]
    
    \DndMonsterAction{Charge}
    If Mort moves at least 20 feet straight toward a target and then hits it with a Slam attack on the same turn, the target takes an extra ld8 bludgeoning damage. If the target is a creature, it must succeed on a Strength saving throw against your spell save DC or be knocked prone.
    
    \DndMonsterAction{Primal Bond}
    You can add your proficiency bonus to any ability check or saving throw that the beast makes.
	
	\DndMonsterSection{Actions}	
	\DndMonsterAttack[
      name=Slam,
      distance=melee, % valid options are in the set {both,melee,ranged},
      %type=weapon, %valid options are in the set {weapon,spell}
      mod=\calculateSpellAttack{\calculateModifier{\WisdomScoreValue}},
      reach=5,
      %range=30,
      targets=one target,
      dmg=\DndDice{1d10} + 3 + PB,
      dmg-type=bludgeoning,
      %plus-dmg=,
      %plus-dmg-type=,
      %or-dmg=,
      %or-dmg-when=,
      %extra=,
    ]
    
    \DndMonsterAttack[
      name=Pounce,
      distance=melee, % valid options are in the set {both,melee,ranged},
      %type=weapon, %valid options are in the set {weapon,spell}
      mod=\calculateSpellAttack{\calculateModifier{\WisdomScoreValue}},
      reach=5,
      %range=30,
      targets=one target,
      dmg=\DndDice{2d8} + 3 + PB,
      dmg-type=bludgeoning,
      %plus-dmg=,
      %plus-dmg-type=,
      %or-dmg=,
      %or-dmg-when=,
      %extra=,
    ]
    
    \DndMonsterSection{Reactions}
    \DndMonsterAction{Flash Grapple}
    If another creature provokes an opportunity attack of Mort, he can instead try to Grapple that creature and has advantage on the attack throw. Mort can not use this reaction if he already grapples another creature.
\end{DndMonster}
\subsubsection*{Exceptional Training}
Beginning at 7th level, on any of your turns when your beast companion doesn’t attack, you can use a bonus action to command the beast to take the Dash, Disengage, or Help action on its turn. In addition, the beast’s attacks now count as magical for the purpose of overcoming resistance and immunity to nonmagical attacks and damage.
\subsection*{Primal Awareness}
This 3rd-level feature replaces the Primeval Awareness feature. You gain no benefit from the replaced feature and don't qualify for anything in the game that requires it.

You can focus your awareness through the interconnections of nature: you learn additional spells when you reach certain levels in this class if you don't already know them, as shown in the Primal Awareness Spells table. These spells don't count against the number of ranger spells you know.
\begin{DndTable}[header=Primal Awareness Spells]{llX}
			& \textbf{Ranger Level}  	&\textbf{Spells}		\\
$\bullet$	& 3rd						&Speak with Animals		\\
$\bullet$	& 5th						&Beast Sense			\\
			& 9th						&Speak with Plants		\\
			& 13th						&Locate Creature		\\
			& 17th						&Commune with Nature	\\
\end{DndTable}
You can cast each of these spells once without expending a spell slot. Once you cast a spell in this way, you can't do so again until you finish a long rest.
\subsection*{Extra Attack}
Beginning at 5th level, you can attack twice, instead of once, whenever you take the Attack action on your turn.

\section*{Spells}
\subsection*{Level 1}

\DndSpellHeader
  {Speak with Animals}
  {1st-Level Divination (Ritual)}
  {1 Action}
  {Self}
  {V, S}
  {10 Minutes}

You gain the ability to comprehend and verbally communicate with beasts for the duration. The knowledge and awareness of many beasts is limited by their intelligence, but at minimum, beasts can give you information about nearby locations and monsters, including whatever they can perceive or have perceived within the past day. You might be able to persuade a beast to perform a small favor for you, at the DM’s discretion.

\DndSpellHeader
  {Absorb Elements}
  {1st-Level Abjuration}
  {1 Reaction, which you take when you take acid, cold, fire, lightning, or thunder damage}
  {Self}
  {S}
  {1 Round}

The spell captures some of the incoming energy, lessening its effect on you and storing it for your next melee attack. You have resistance to the triggering damage type until the start of your next turn. Also, the first time you hit with a melee attack on your next turn, the target takes an extra 1d6 damage of the triggering type, and the spell ends.

\subparagraph*{At Higher Levels} When you cast this spell using a spell slot of 2nd level or higher, the extra damage increases by 1d6 for each slot level above 1st.

\DndSpellHeader
  {Entangle}
  {1st-Level Conjuration}
  {1 Action}
  {90 feet}
  {V, S}
  {Concentration, up to 1 Minute}

Grasping weeds and vines sprout from the ground in a 20-foot square starting from a point within range. For the duration, these plants turn the ground in the area into difficult terrain.

A creature in the area when you cast the spell must succeed on a Strength saving throw or be restrained by the entangling plants until the spell ends. A creature restrained by the plants can use its action to make a Strength check against your spell save DC. On a success, it frees itself.

When the spell ends, the conjured plants wilt away.

\DndSpellHeader
  {Hunter's Mark}
  {1st-Level Divination}
  {1 Bonus Action}
  {90 feet}
  {V}
  {Concentration, up to 1 Hour}

You choose a creature you can see within range and mystically mark it as your quarry. Until the spell ends, you deal an extra 1d6 damage to the target whenever you hit it with a weapon attack, and you have advantage on any Wisdom (Perception) or Wisdom (Survival) check you make to find it.

If the target drops to 0 hit points before this spell ends, you can use a bonus action on a subsequent turn of yours to mark a new creature.

\subparagraph*{At Higher Levels} When you cast this spell using a spell slot of 3rd or 4th level, you can maintain your concentration on the spell for up to 8 hours. When you use a spell slot of 5th level or higher, you can maintain your concentration on the spell for up to 24 hours.

\subsection*{Level 2}

\DndSpellHeader
  {Beast Sense}
  {2nd-Level Divination (Ritual)}
  {1 Action}
  {Touch}
  {S}
  {Concentration, up to 1 Hour}

You touch a willing beast. For the duration of the spell, you can use your action to see through the beast’s eyes and hear what it hears, and continue to do so until you use your action to return to your normal senses.

\subparagraph*{At Higher Levels} When you cast this spell using a spell slot of 3rd level or higher, you create one additional ray for each slot level above 2nd.

\DndSpellHeader
  {Pass Without Trace}
  {2nd-Level Abjuration}
  {1 Action}
  {Self}
  {V, S, M (ashes from a burned leaf of mistletoe and a sprig of spruce)}
  {Concentration, up to 1 Hour}

A veil of shadows and silence radiates from you, masking you and your companions from detection. For the duration, each creature you choose within 30 feet of you (including you) has a +10 bonus to Dexterity (Stealth) checks and can’t be tracked except by magical means. A creature that receives this bonus leaves behind no tracks or other traces of its passage.

\DndSpellHeader
  {Spike Growth}
  {2nd-Level Transmutation}
  {1 Action}
  {150 feet}
  {V, S, M (seven sharp thorns or seven small twigs, each sharpened to a point)}
  {Concentration, up to 10 Minutes}

The ground in a 20-foot radius centered on a point within range twists and sprouts hard spikes and thorns. The area becomes difficult terrain for the duration. When a creature moves into or within the area, it takes 2d4 piercing damage for every 5 feet it travels.

The transformation of the ground is camouflaged to look natural. Any creature that can’t see the area at the time the spell is cast must make a Wisdom (Perception) check against your spell save DC to recognize the terrain as hazardous before entering it.
}