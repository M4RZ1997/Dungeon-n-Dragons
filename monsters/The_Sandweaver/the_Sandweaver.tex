\documentclass[a4paper,openany,twoside,twocolumn]{book}
%\documentclass[letterpaper,openany,oneside,twocolumn]{book}

\newcommand{\PATH}{../../}

\usepackage[justified]{\PATH dndtemplate/dnd}
\usepackage{\PATH monsters/stylesheets/monster_stylesheet}

\usepackage[english]{babel}
\usepackage[utf8]{inputenc}

%\newcommand{\entryfont}{\DndFontStatBlockBody}
\newfontfamily\entryfont{Kalam}[Path=\PATH template/fonts/,Extension=.ttf,UprightFont=Kalam-Regular,BoldFont=Kalam-Bold]

\geometry{a4paper}

\begin{document}

%\layout

\MonsterSheetGeometry

\mainmatter%

\twocolumn[\section*{\entryfont Giant Sandweaver}]

\entryfont \noindent \DndDropCapLine{N}ative to sandy deserts and shrugged canyons, the Giant Sandweaver is a horrendous sight to encounter. Different to common spiders these monsters do not spin large intricate webs, but instead cover large areas of their territory with a thick weave of silk to act as an alarm system for intruders. Furthermore, Giant Sandweaver live in large families, similar to hives, which form around a matriarch - a so-called Sandwidow.\\
\textbf{Appearance.} Giant Sandweavers are large serpents that can reach lengths of up to 12 feet and have 6-8 spider-like legs. Individual specimens, especially Sandwidows, are said to have been up to 15 feet in length. The body itself is covered by hard plates, giving a very good resistance to any attack. Closer to the head these bodyplates form massive spikes. While the Giant Sandweaver is buried in the sand these spikes might be mistaken for stones or obelisk by unaware travellers.\\
\textbf{Ritual Abomination.} The first Giant Sandweavers were created in dark rituals by the lizardfolk tribe Yuan-Shu during the war with the Yuan-Ti. These were released in crucial positions, most commonly canyons, to hinder the advances of the opponent's troops and attack any snakefolk that trespasses the terrain. The Yuan-Shu relished in the idea of turning snakes into spider-like monsters that preyed upon the snake worshipping Yuan-Ti, a metaphor of the lizardfolk's superiority.\\
During combat the Giant Sandweaver tries to catch or incapacitate its opponents within its webbing. Already caught opponents are not the preferred target if other non-incapacitated opponents are around.\\
\textbf{Monstrous Mounts.} Over the centuries, Yuan-Ti were able to tame some Giant Seandweavers just enough to serve as mounts in combat. However, Giant Sandweaver are notoriously treacherous creatures and their alliance usually lies with whoever feeds them.\\
- M4RZ.\\

\MonsterGraphicAndShortInfo{8.1cm}{-1.4cm}{17}{15em}{pictureoverlay}%
{%
	First created by the Yuan-Shu to gain an advantage in combat and war, these spider-like serpents roam through the vast deserts and deep canyons. These monsters are formidable foes and it is ill-advised to mess with their hive.%
}{0.95cm}{375pt}{images/The_Sandweaver.png}%

\vfill\eject

\MonsterVariant{0.5em}{images/The_Sandweaver_variant.png}{Sandwidow}{%
	The matriarch of a Giant Sandweaver family has more similarities to an actual snake and its legs are covered in hair, that can detect pressure differences and air movement in the Sandwidow's surroundings.
	\DndMonsterAction{Hit Points} \DndDice{15d12 + 66}
	\DndMonsterAction{Pit organs}
	The Sandwidow has blindsight in a 50 ft. radius, on creatures whose body temperature is more than half than the temperature of its surroundings.
	\DndMonsterAction{Trichobothria}
	The Sandwidow can detect movement within a 100ft. radius.
	\DndMonsterAction{Spit Venom (Recharge 5-6)}
	The Sandwidow spits venom in a 40 foot line that is 5 feet wide. Each creature in that line must make a DC 18 Constitution saving throw, taking \DndDice{6d8 + 6} poison damage on a failed save, or half as much on a successful one.
}%

% Monster stat block
\begin{DndMonster}[width=0.5\textwidth]{Giant Sandweaver}
    \DndMonsterType{Huge monstrosity, unaligned}

    % If you want to use commas in the key values, enclose the values in braces.
    \DndMonsterBasics[
        armor-class = {18 (natural armor)},
        hit-points  = {\DndDice{8d12 + 42}},
        speed       = {30 ft., climb 25 ft., dig/tunneling 25 ft.},
      ]

    \DndMonsterAbilityScores[
        str = 19,
        dex = 15,
        con = 21,
        int = 6,
        wis = 9,
        cha = 5,
      ]

    \DndMonsterDetails[
        %saving-throws = {Str +0, Dex +0, Con +0, Int +0, Wis +0, Cha +0},
        %skills = {Acrobatics +0, Animal Handling +0, Arcana +0, Athletics +0, Deception +0, History +0, Insight +0, Intimidation +0, Investigation +0, Medicine +0, Nature +0, Perception +0, Performance +0, Persuasion +0, Religion +0, Sleight of Hand +0, Stealth +0, Survival +0},
        %damage-vulnerabilities = {cold},
        damage-resistances = {bludgeoning, piercing, and slashing from nonmagical attacks (not Sandwidow), poison},
        %damage-immunities = {cold},
        senses = {passive Perception 14},
        condition-immunities = {frightened, poisoned, prone},
        languages = {Primordial},
        challenge = 9,
      ]
      
    \DndMonsterAction{Spider Climb}
    The Giant Sandweaver can climb difficult surfaces, including upside down ceilings, without the need of an ability check.
      
    \DndMonsterAction{Web Sense}
    While in contact with a web, the Giant Sandweaver knows the exact location of any other creature in contact with the same web.
      
    \DndMonsterSection{Actions}
	\DndMonsterAction{Multiattack}
	The Giant Sandweaver can attack twice each turn.   
    
    \DndMonsterAttack[
      name=Poisonous Bite,
      distance=melee, % valid options are in the set {both,melee,ranged},
      %type=weapon, %valid options are in the set {weapon,spell}
      mod=+7,
      reach=5,
      %range=20/60,
      targets=one target,
      dmg={\DndDice{2d6 + 1}},
      dmg-type=piercing,
      %plus-dmg=,
      %plus-dmg-type=,
      %or-dmg=,
      %or-dmg-when=,
      extra={. The target must make a DC 17 Constitution saving throw, taking \DndDice{4d6} poison damage on a failed save, or half as much damage on a successful one. On a failure, the target is poisoned},
    ]
    
    \DndMonsterAttack[
      name=Scythe Slashing,
      distance=melee, % valid options are in the set {both,melee,ranged},
      %type=weapon, %valid options are in the set {weapon,spell}
      mod=+6,
      reach=5,
      %range=20/60,
      targets=one target,
      dmg={\DndDice{3d8 + 8}},
      dmg-type=piercing,
      %plus-dmg=,
      %plus-dmg-type=,
      %or-dmg=,
      %or-dmg-when=,
      %extra=,
    ]
      
    \DndMonsterAttack[
      name=Web Shooter (Recharge 5-6),
      distance=ranged, % valid options are in the set {both,melee,ranged},
      %type=weapon, %valid options are in the set {weapon,spell}
      mod=+3,
      reach=20/60,
      %range=20/60,
      targets=5 ft. radius,
      %dmg=,
      %dmg-type=slashing,
      %plus-dmg=,
      %plus-dmg-type=,
      %or-dmg=,
      %or-dmg-when=,
      extra={Any target within the radius is restrained by webbing. As an action, the restrained target can make a DC 16 Strength check, bursting the web on a success. The webbing can also be attacked and destroyed (AC 10; HP 7; vulnerability to fire damage; immunity to bludgeoning, poison, and psychic damage)},
    ]
      
\end{DndMonster}

% --------------------------------------------------------------------------------------------------- %
% ################################################################################################### %
% #-#-#-#-#-#-#-#-#-#-#-#-#-#-#-#-# Monster-Sheet  with full Banner #-#-#-#-#-#-#-#-#-#-#-#-#-#-#-#-# %
% ################################################################################################### %
% --------------------------------------------------------------------------------------------------- %

\MonsterBannerGraphic%
	{Ancient Sandweaver Lair}% name of the monster to be displayed as header
	{180pt}% offset for the section header
	{250pt}% max height of the image
	{images/Desert_Lair.png}% image to be displayed as a banner
	{}% used for keepaspectratio for image ({} or {, keepaspectratio})
	
\entryfont \noindent \DndDropCapLine{B}uild by the snakefolk Yuan-Ti,\\ this monument to the snake deity Sseth\\ is winding across the peaks of a jagged cliff formation in the middle of the [Desert]. These rugged lands are home to Giant Sandweavers, huge abominations of part spider part snake. Occasionally, members of the Yuan-Ti are following the Path of Enlightenment visiting this shrine to bring offerings. Unbeknownst to many an even larger creature is calling this sanctum home - the Ancient Sandweaver. Legends are spreading around this ungodly creature and feared by those who live to tell the tale of its encounter.\\
The desert is filled with stone pinnacles and broken obelisks and monument ruins clutter the area - But be aware, not everything is what it seems to be...
	
% Monster stat block
\begin{DndMonster}[float*=b,width=\textwidth +8pt]{Ancient Sandweaver}
    \vspace*{-17.5pt}\begin{multicols}{2}\DndMonsterType{Gargantuan monstrosity, unaligned}

    % If you want to use commas in the key values, enclose the values in braces.
    \DndMonsterBasics[
        armor-class = {22 (natural armor)},
        hit-points  = {\DndDice{16d20 + 112}},
        speed       = {50 ft., dig/tunneling 50 ft.},
      ]
      
    \renewcommand{\AbilityScoreSpacer}{~}

    \DndMonsterAbilityScores[
        str = 24,
        dex = 10,
        con = 26,
        int = 14,
        wis = 15,
        cha = 5,
      ]

    \DndMonsterDetails[
        %saving-throws = {Str +0, Dex +0, Con +0, Int +0, Wis +0, Cha +0},
        %skills = {Acrobatics +0, Animal Handling +0, Arcana +0, Athletics +0, Deception +0, History +0, Insight +0, Intimidation +0, Investigation +0, Medicine +0, Nature +0, Perception +0, Performance +0, Persuasion +0, Religion +0, Sleight of Hand +0, Stealth +0, Survival +0},
        %damage-vulnerabilities = {cold},
        damage-resistances = {bludgeoning, piercing, and slashing from nonmagical attacks},
        damage-immunities = {poison},
        senses = {Tremorsense 3000ft., passive Perception 19},
        condition-immunities = {frightened, poisoned, prone},
        languages = {Primordial},
        challenge = 15,
      ] 
    \DndMonsterAction{Web Sense}
    While in contact with a web, the Ancient Sandweaver knows the exact location of any other creature in contact with the same web.
      
    \DndMonsterSection{Actions}
	\DndMonsterAction{Multiattack}
	The Ancient Sandweaver can attack twice each turn.   
    
    \DndMonsterAttack[
      name=Poisonous Bite,
      distance=melee, % valid options are in the set {both,melee,ranged},
      %type=weapon, %valid options are in the set {weapon,spell}
      mod=+10,
      reach=25,
      %range=20/60,
      targets=one target,
      dmg={\DndDice{10d6 + 13}},
      dmg-type=piercing,
      %plus-dmg=,
      %plus-dmg-type=,
      %or-dmg=,
      %or-dmg-when=,
      extra={. The target must make a DC 20 Constitution saving throw, taking \DndDice{8d6 + 16} poison damage on a failed save, or half as much damage on a successful one. On a failure, the target is poisoned},
    ]
    
    \DndMonsterAttack[
      name=Bash,
      distance=melee, % valid options are in the set {both,melee,ranged},
      %type=weapon, %valid options are in the set {weapon,spell}
      mod=+12,
      reach=35,
      %range=20/60,
      targets=one target,
      dmg={\DndDice{12d12 + 12}},
      dmg-type=bludgeoning,
      %plus-dmg=,
      %plus-dmg-type=,
      %or-dmg=,
      %or-dmg-when=,
      extra={. If the targets are creatures, they must succeed on a DC 22 Strength or Dexterity saving throw or be knocked prone},
    ]
      
    \DndMonsterAttack[
      name=Web Shooter (Recharge 5-6),
      distance=ranged, % valid options are in the set {both,melee,ranged},
      %type=weapon, %valid options are in the set {weapon,spell}
      mod=+8,
      reach=20/60,
      %range=20/60,
      targets=10 ft. radius,
      %dmg=,
      %dmg-type=slashing,
      %plus-dmg=,
      %plus-dmg-type=,
      %or-dmg=,
      %or-dmg-when=,
      extra={Any target within the radius is restrained by webbing. As an action, the restrained target can make a DC 16 Strength check, bursting the web on a success. The webbing can also be attacked and destroyed (AC 10; HP 7; vulnerability to fire damage; immunity to bludgeoning, poison, and psychic damage)},
    ]
    
    \DndMonsterSection{Reaction}
    \DndMonsterAction{Wrap}
    When a creature is hit by the Sandweaver's poisonous bite attack and becomes paralyzed by its poison, the Ancient Sandweaver can use its silk webbing to restrain the target. If the target is Medium in size or smaller, the Ancient Sandweaver can then attach the target to its back or belly. The Ancient Sandweaver  can have only one creature attached to its body at a time. A creature freed from this webbing is no longer attached to the Ancient Sandweaver's body.
    
    \DndMonsterSection{Legendary Actions}
    The Ancient Sandweaver can take 3 Legendary Actions, choosing from the options below. Only one Legendary Action option can be used at a time and only at the end of another creature's turn. The Ancient Sandweaver regains spent Legendary Actions at the start of its turn.
    \DndMonsterAction{Bite}
    The Ancient Sandweaver makes one Poisonous Bite attack
    \DndMonsterAction{Sandhide (Costs 2 Actions)}
    The Ancient Sandweaver creates a 50ft. cloud of sand. A creature engulfed by the sand must make a DC 14 Constitution saving throw or is blinded until the end of their next turn.
    \DndMonsterAction{Dig and Strike (Costs 3 Actions)}
    The Ancient Sandweaver digs down, being hidden until it resurfaces, and has advantage and +30 ft. reach (cumulative) on its next melee attack.
      
\end{multicols}\end{DndMonster}
\vfill\eject\hfill\vspace*{-8pt}
\subsection*{Lair Actions}
\DndMonsterAction{Passive Effect: Legendary Malevolence}
In a radius of 10.000ft. around the lair no creature is benefitting from a long rest.\\
On initiative count 20, the Ancient Sandweaver takes a lair action to cause one of the following effect. It can't use the same effect two rounds in a row:
\begin{itemize}
	\item A 30 foot square area of ground within 120 feet of the Ancient Sandweaver becomes quicksand.
	\item A 5 foot rock the Sandweaver can see is crashing to the ground. Any creature within that area must succeed a DC 14 Dexterity saving throw or suffers \DndDice{2d10} bludgeoning damage and their movement rate is set to 0 until the end of their next turn.
	\item A 10 foot wide, 20 foot deep hole within 120 feet of the Sandweaver is opening up.
\end{itemize}

\end{document}
