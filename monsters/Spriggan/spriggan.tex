\documentclass[letterpaper,openany,oneside,twocolumn]{book}

\newcommand{\PATH}{../../}

\usepackage{fontspec}
\usepackage[justified]{\PATH dndtemplate/dnd}
\usepackage{\PATH monsters/stylesheets/monster_stylesheet}
\usepackage[english]{babel}

\usepackage[utf8]{inputenc}

%\newcommand{\entryfont}{\DndFontStatBlockBody} & uses the default font provided by the LaTeX DnD-Template
\newfontfamily\entryfont{Kalam}[Path=\PATH template/fonts/,Extension=.ttf,UprightFont=Kalam-Regular,BoldFont=Kalam-Bold] % requires XeLaTeX or LuaTeX

\begin{document}

%\layout

\mainmatter%

% --------------------------------------------------------------------------------------------------- %
% ################################################################################################### %
% #-#-#-#-#-#-#-#-#-#-#-#-#-#-# Monster-Sheet with two Smaller Pictures #-#-#-#-#-#-#-#-#-#-#-#-#-#-# %
% ################################################################################################### %
% --------------------------------------------------------------------------------------------------- %

\twocolumn[\section*{\entryfont Blank Monster}]

\entryfont \noindent \DndDropCapLine{T}his document is a LaTeX-template for easily creating a DnD-Monster-Sheet. It provides many different environments and macros to build up many different blocks similar to the ones seen in the DnD books.\\\\
This document uses the LaTeX dnd\textunderscore template provided in the GitHub repository:\\\\ https://github.com/rpgtex/DND-5e-LaTeX-Template\\\\ and requires of XeLaTeX or LuaTeX as the fontspec package is not part of the normal PDFLaTeX. By altering the \entryfont in the preamble this requirement can be excluded, however, the font has to be reset to the default one. \\\\
... enjoy!\\\\
- M4RZ.\\

\MonsterGraphicAndShortInfo%
	{9.4cm}{-0.45cm}% Position of Mini-Info-Box relativ to bottom left text-box corner
	{25}% rotation of Mini-Info-Box
	{9em}% maximum width of Mini-Info-Box
	{infoboxoverlay} % What should be above? Picture or InfoBox {{pictureoverlay}, {infoboxoverlay}} - if something else is written in there no InfoBox is displayed
	{This is the Mini-Info-Box:\\The Blank Monster has never been seen and there are no pictures of it as of now.}%
	{0cm}% offset of picture to bottom of page
	{375pt}% maximum height of picture
	{images/spriggan.png}% picture

\vfill\eject

\MonsterVariant%
	{1em}% offset to top of page for image
	{example-image-a}% image above Info-Box
	{The Mysterious One}% Name of the variant
	{%
		The Mysterious One is even more mysterious than the blank one.
		\DndMonsterAction{Condition Immunities}
		detection
		\DndMonsterAction{Surprise}
		Whenever someone sees the mysterious one it is charmed and thinks it didn't see anything
	}% Monster Variant Info

% Monster stat block
\begin{DndMonster}[width=0.5\textwidth]{Blank Monster}
    \DndMonsterType{Unidentified Monster, unaligned}

    % If you want to use commas in the key values, enclose the values in braces.
    \DndMonsterBasics[
        armor-class = {10},
        hit-points  = {\DndDice{1d12 + 6}},
        speed       = {30 ft., climb 25 ft.},
    ]

    \DndMonsterAbilityScores[
        %str = 19,
        %dex = 15,
        %con = 13,
        %int = 4,
        %wis = 9,
        %cha = 5,
    ]

    \DndMonsterDetails[
        %saving-throws = {Str +0, Dex +0, Con +0, Int +0, Wis +0, Cha +0},
        %skills = {Acrobatics +0, Animal Handling +0, Arcana +0, Athletics +0, Deception +0, History +0, Insight +0, Intimidation +0, Investigation +0, Medicine +0, Nature +0, Perception +0, Performance +0, Persuasion +0, Religion +0, Sleight of Hand +0, Stealth +0, Survival +0},
        %damage-vulnerabilities = {cold},
        %damage-resistances = {bludgeoning, piercing, and slashing from nonmagical attacks},
        %damage-immunities = {cold},
        %senses = {passive Perception 14},
        %condition-immunities = {frightened, poisoned},
        %languages = {-},
        challenge = 1,
    ]
    
    \DndMonsterAction{One of Many}
    The Blank Monster has advantage against being detected and hiding in large groups.
	
	\DndMonsterSection{Actions}
	\DndMonsterAction{Multiattack}
	The Blank Monster makes two attacks with its dagger.
	
	\DndMonsterAttack[
      name=Dagger,
      distance=melee, % valid options are in the set {both,melee,ranged},
      %type=weapon, %valid options are in the set {weapon,spell}
      mod=+3,
      reach=10,
      %range=20/60,
      targets=one target,
      dmg=\DndDice{1d8 + 2},
      dmg-type=slashing,
      %plus-dmg=,
      %plus-dmg-type=,
      %or-dmg=,
      %or-dmg-when=,
      %extra=,
    ]
      
\end{DndMonster}

\end{document}
