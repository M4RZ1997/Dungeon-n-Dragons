\documentclass[letterpaper,openany,oneside,twocolumn]{book}

\newcommand{\PATH}{../../}

\usepackage{\PATH templates/utilities/m4rz-fonts}
\usepackage{\PATH templates/utilities/m4rz-colors}

\usepackage[justified]{\PATH templates/template_dnd/dnd}

\usepackage[edition=pre5e24]{\PATH templates/template_character-sheet/character-sheet-stylesheet}
\usepackage{\PATH templates/template_magic-item/magic-items_commands}

\setlength\oddsidemargin{\dimexpr(\paperwidth-\textwidth)/2 - 1in\relax}
\setlength\evensidemargin{\oddsidemargin}

% Headline
\CharacterName{Laucian Ilphelkiir}

\Class{Ranger}
\Level{14}
\Background{Folk Hero}
\PlayerName{M4RZ}
\Race{Wood Elf}
\Alignment{Chaotic Good}
\XP{}

% Ability scores
\StrengthRolledScore{13}
\DexterityRolledScore{15}
\ConstitutionRolledScore{12}
\IntelligenceRolledScore{10}
\WisdomRolledScore{13}
\CharismaRolledScore{10}

\StrengthScoreBonus{2} % ASI + 1
\DexterityScoreBonus{5} % Elf-Race +2, ASI +2
\ConstitutionScoreBonus{0}
\IntelligenceScoreBonus{0}
\WisdomScoreBonus{1} % ASI + 1
\CharismaScoreBonus{0}

\calculateAbilityScores{}

% Proficiencies (Proficient = 'P', Expertise = 'E', otherwise = '')
\StrengthProficiency{P}
\DexterityProficiency{P}
\ConstitutionProficiency{}
\IntelligenceProficiency{}
\WisdomProficiency{}
\CharismaProficiency{}

\AcrobaticsProficiency{P}
\AnimalHandlingProficiency{P}
\ArcanaProficiency{}
\AthleticsProficiency{P}
\DeceptionProficiency{}
\HistoryProficiency{}
\InsightProficiency{}
\IntimidationProficiency{}
\InvestigationProficiency{}
\MedicineProficiency{}
\NatureProficiency{P}
\PerceptionProficiency{P}
\PerformanceProficiency{}
\PersuasionProficiency{}
\ReligionProficiency{}
\SleightOfHandProficiency{}
\StealthProficiency{P}
\SurvivalProficiency{P}

\StrengthModifierBonus{0}
\DexterityModifierBonus{0}
\ConstitutionModifierBonus{0}
\IntelligenceModifierBonus{0}
\WisdomModifierBonus{0}
\CharismaModifierBonus{0}

\Inspiration{}
\Proficiency{+5}
\PassivePerceptionModifier{0}

\ArmorClass{17}
\InitiativeModifier{0}
\Speed{35ft}
\MaxHitPointsRolled{102}
\CurrentHitPoints{}
\TemporaryHitPoints{}
\HitDice{d10}
\HitDiceSpent{0}

\CP{}
\SP{}
\EP{}
\GP{7883}
\PP{}

\addWeaponStatistic{Shortsword}{LaucianIlphelkiir}{STR}{P}{0}{1d6 p}
\addWeaponStatistic{Oathbow}{LaucianIlphelkiir}{DEX}{P}{6}{1d8 p}
\addWeaponStatistic{Unarmed Strike}{LaucianIlphelkiir}{STR}{P}{0}{\intcalcAdd{1}{\calculateModifier{\StrengthScoreValue}} b}

\AttacksAdditional{
	\textbf{Oathbow (150/600)}\\
	2 Shortswords\\
	60 Arrows (1 with Oil)\\
	\textbf{Boomerang Arrow}\\
	\textbf{50 Fire Arrows}\\
	\textbf{36 Drow Arrows}\\
	\textbf{10 Ethereal Arrows}\\
	\textbf{4 Ice Arrows}\\
	\textbf{Arrow of Slaying (Dragon)}\\
}

\OtherProficienciesLanguages{
	\textbf{Languages:}\\Common, Elvish, Sylvan, Draconic, Primordial\\
	\textbf{Armor:}\\Light, Medium Armor, Shields\\
	\textbf{Weapons:}\\Shortsword, Longsword, Shortbow, Longbow, Simple, Martial Weapons\\
	\textbf{Tools:}\\Woodcarver's Tools
}

\Equipment{
	Chain Shirt +2, Leather Armor, Magical Shield (?)\\
	\textbf{Quiver of Ehlonna, Quiver of the Storm's Fury, Wand of Secrets, Ring of Free Action}\\
	a Tranquilizer Vial, 3 Potion of Greater Healing, Potion of Supreme Healing, Ruby (500 gold), Potion of Giant Size
}
\Clutter{
	Woodcarver's Tools, a Shovel, an Iron Pot, Set of Common Clothes, a Backpack, a Bedrool, a Mess Kit, a Tinderbox, 10 Torches, 12 Days of Rations, a Waterskin, 50 ft Hempen Rope, Firewine, Wedgestone
}

\PersonalityTraits{
	\textbf{Compassionate:} Laucian possesses a deep empathy for others and feels a strong sense of responsibility to help those in need, especially those affected by the destruction of the forest.
}

\Ideals{
	\textbf{Nature's Guardian:} He feels a strong bond with the natural world and considers himself a protector and advocate for its well-being.
}

\Bonds{
	\textbf{Justice:} He upholds the ideals of justice and fairness, seeking to right the wrongs done to his tribe and others affected by similar tragedies.
}

\Flaws{
	\textbf{Pyrophobia:} The traumatic event of witnessing his tribe and home being engulfed in flames has left a lasting impact on him.
}

\FeaturesTraits{
\textbf{Elf Traits}
\begin{itemize}
	\item Darkvision
	\item Fey Ancestry
	\item Trance
	\item Elf Weapon Training
	\item Mask of the Wild
\end{itemize}
\textbf{Favored Enemy}
\begin{itemize}
	\item Elementals, Dragon, Monstrosities
\end{itemize}
\textbf{Favorable Terrain}
\begin{itemize}
	\item Forest, Swamp, Arctic, Mountain
\end{itemize}
\textbf{Fighting Style}
\begin{itemize}
	\item Archer
\end{itemize}
\textbf{Ranger Archetypes}
\begin{itemize}
	\item Horde Breaker
	\item Escape the Horde
	\item Volley
\end{itemize}
\textbf{Feats}
\begin{itemize}
	\item Sharpshooter
	\item Wood Elf Magic
\end{itemize}
}

% Appearance

\Age{240}
\Height{6'2"}
\Weight{110lbs}
\Eyes{Silver}
\Skin{Greenish}
\Hair{Brown}

% background

\CharacterAppearance{images/Laucian_Ilphelkiir.png}{
	\noindent Laucian’s silver-green eyes glisten like his bow. He has short, brown hair. He has long, pointed ears that prick up when Laucian is on the hunt or guard.
}{}{}{}
\AdditionalFeaturesAndTraits{
	Laucian's years spent as a ranger have honed his archery skills to exceptional levels. He is known for his precise and accurate shots, often hitting targets with remarkable speed and efficiency.\\
	Laucian possesses a natural affinity for tracking and hunting. His keen senses and deep knowledge of the wilderness allow him to follow trails, identify animal tracks, and navigate through various terrains with ease.\\
	Having grown up in the forest, Laucian has developed a profound connection with nature. He can communicate with animals, discern signs of upcoming weather changes, and identify useful plants and herbs for survival.\\
	Despite the tragedy he experienced, Laucian's spirit remains unbroken. He possesses great resilience and determination, always striving to protect what remains of the natural world and seeking justice for his fallen tribe.\\
	The destruction caused by Braga, the Red, has instilled a burning hatred for dragons within Laucian. He is driven by a relentless desire for vengeance and is determined to protect others from similar devastation.
}
\Characterbackground{
Laucian was born an elf of the wood - a very reclusive race that dwells in the forests of "Eastern Wallaby Grove". His parents were aristocrats of the tribe whereas his grandfather was the main hunter of the community. Because of his grandfather's teachings, Laucian quickly became an expert marksman with the bow and also in laying traps for wild animals. As Laucian was very capable at a young age he was very respected since he provided the clan with venison and other sorts of foods that were gathered during the seasonal hunt. \\
The most striking event was when Laucian was just 40 years old. On that particular day, the city of "New Forestia" was attacked by a huge dragon that burned down everything that stood in its way. Laucian was the only survivor on that faithful day and he became obsessed with the search for that dragon. Also, he became afraid of the nature of fire as he observed the obstruction it was capable of.
}
\Treasure{
	\textbf{1. Cloak of the Forest Guardian:} A finely woven cloak made from enchanted moss and leaves. When worn, it grants Laucian the ability to blend seamlessly with his surroundings, rendering him nearly invisible in natural environments.\\
	\textbf{2. Memory Stones:} A collection of small, smooth stones that Laucian carries with him. Each stone is engraved with the name of a fallen member of his tribe, serving as a memorial to the lost souls and a way for Laucian to honor their memory.\\
	\textbf{3. Coin and Talisman:} A small pouch with a single silver piece with the old stamping of "New Forestia" and a talisman with the symbol of the tribe. The emblem of "New Forestia" features a beautifully detailed image of a majestic tree, reminiscent of Yggdrasil. The tribe believed the great tree represented unity and protection. It serves as a symbol of honor and heritage for Laucian, a constant reminder of his connection to his lost tribe and the enduring spirit of New Forestia.
}
\AlliesAndOrganizations{
The clan of "New Forestia" was once a very flourishing and wealthy tribe. Although, the main population of the tribe lived very reclusive trade with surrounding tribes prospered and the yield of hunts was a very popular commodity in the nearby villages and cities. \\
On a faithful day, nearly 200 years ago the tribe and the whole forest of "Eastern Wallaby Grove" were burned down by the Dragon "Braga, the Red". Laucian Ilphelkiir, a son of the noblemen of the clan was the only survivor and tries to find allies and friends of "New Forestia" to spread the knowledge of what happened and that the\linebreak
}{
	history of this once prospering tribe is never forgotten.
}
\OrganizationName{New Forestia (Elf Tribe)}
\OrganizationSymbol{images/Tree_of_Forestia.png}

% Magic

\SpellcastingClass{Ranger}
\SpellcastingAbility{WIS} % STR, DEX, CON, INT, WIS, CHA
\SpellSaveDCModifier{0} % any modifier that isn't contained in "8 + Ability Modifier + Proficiency Bonus"
\SpellAttackModifier{0} % any modifier that isn't contained in "Ability Modifier + Proficiency Bonus"

\CantripSlotA{Control Flames (S)}
\CantripSlotB{Longstrider (V, S, M) (once a day)}
\CantripSlotC{Pass Without Trace (V, S, M) (once a day)}

\FirstLevelSpellSlotsTotal{4}
\FirstLevelSpellSlotA{Ensnaring Strike (V)}
\FirstLevelSpellSlotB{Hail of Thorns (V)}

\SecondLevelSpellSlotsTotal{3}
\SecondLevelSpellSlotA{Find Traps (V, S)}
\SecondLevelSpellSlotB{Spike Growth (V, S, M)}
\SecondLevelSpellSlotC{Cordon of Arrows (V, S, M)}

\ThirdLevelSpellSlotsTotal{3}
\ThirdLevelSpellSlotA{Lightning Arrow (V, S)}
\ThirdLevelSpellSlotB{Conjure Barrage (V, S, M)}
\ThirdLevelSpellSlotC{Protection from Energy (V, S)}

\FourthLevelSpellSlotsTotal{1}
\FourthLevelSpellSlotA{Stoneskin (V, S, M)}

\begin{document}

\newgeometry{left=0cm,right=0cm,top=0cm,bottom=0cm}
\onecolumn


% CHARACTER PAGE
\rendercharactersheet

% BACKSTORY PAGE
\renderbackgroundsheet

% SPELLCASTING PAGE
\renderspellsheet



\restoregeometry
\twocolumn

\chapter*{Features, Magic Items and Spells}

\section*{Ranger}
\subsection*{Favored Enemy}
You have significant experience studying, tracking, hunting, and even talking to a certain type of enemy.

Choose a type of favored enemy: aberrations, beasts, celestials, constructs, dragons, elementals, fey, fiends, giants, monstrosities, oozes, plants, or undead. Alternatively, you can select two races of humanoid (such as gnolls and orcs) as favored enemies.

You have advantage on Wisdom (Survival) checks to track your favored enemies, as well as on Intelligence checks to recall information about them.

When you gain this feature, you also learn one language of your choice that is spoken by your favored enemies, if they speak one at all.

You choose one additional favored enemy, as well as an associated language, at 6th and 14th level. As you gain levels, your choices should reflect the types of monsters you have encountered on your adventures.

\subsection*{Natural Explorer}
You are particularly familiar with one type of natural environment and are adept at traveling and surviving in such regions. Choose one type of favored terrain: arctic, coast, desert, forest, grassland, mountain, swamp, or the Underdark. When you make an Intelligence or Wisdom check related to your favored terrain, your proficiency bonus is doubled if you are using a skill that you're proficient in.

While traveling for an hour or more in your favored terrain, you gain the following benefits:
\begin{itemize}
	\item Difficult terrain doesn't slow your group's travel.
	\item Your group can't become lost except by magical means.
	\item Even when you are engaged in another activity while traveling (such as foraging, navigating, or tracking), you remain alert to danger.
	\item If you are traveling alone, you can move stealthily at a normal pace.
	\item When you forage, you find twice as much food as you normally would.
	\item While tracking other creatures, you also learn their exact number, their sizes, and how long ago they passed through the area.
\end{itemize}
You choose additional favored terrain types at 6th and 10th level.

\subsection*{Fighting Style}
You adopt a particular style of fighting as your specialty. Choose one of the following options.

You can't take a Fighting Style option more than once, even if you later get to choose again.

\subsubsection*{Archery}
You gain a +2 bonus to attack rolls you make with ranged weapons.

\subsection*{Ranger Archetypes}
\subsubsection*{Hunter}
Emulating the Hunter archetype means accepting your place as a bulwark between civilization and the terrors of the wilderness. As you walk the Hunter's path, you learn specialized techniques for fighting the threats you face, from rampaging ogres and hordes of orcs to towering giants and terrifying dragons.

\paragraph*{Hunter's Prey}
\subparagraph*{Horde Breaker}
Once on each of your turns when you make a weapon attack, you can make another attack with the same weapon against a different creature that is within 5 feet of the original target and within range of your weapon.
\paragraph*{Defensive Tactics}
\subparagraph*{Escape the Horde}
Opportunity attacks against you are made with disadvantage.
\paragraph*{Multiattack}
\subparagraph*{Volley}
You can use your action to make a ranged attack against any number of creatures within 10 feet of a point you can see within your weapon's range. You must have ammunition for each target, as normal, and you make a separate attack roll for each target.

\subsection*{Primeval Awareness}
You can use your Action and expend one Ranger spell slot to focus your awareness on the region around you. For 1 minute per level of the spell slot you expend, you can sense whether the following types of Creatures are present within 1 mile of you (or within up to 6 miles if you are in your Favored terrain): Aberrations, Celestials, Dragons, Elementals, Fey, Fiends, and Undead. This feature doesn't reveal the creatures' Location or number.

\subsection*{Extra Attack}
You can attack twice, instead of once, whenever you take the Attack Action on your turn.

\subsection*{Land's Stride}
Moving through non-magical Difficult Terrain costs you no extra Movement. You can also pass through non-magical Plants without being slowed by them and without Taking Damage from them if they have thorns, spines, or a similar hazard.

In addition, you have advantage on Saving Throws against Plants that are magically created or manipulated to impede Movement, such those created by the Entangle spell.

\subsection*{Hide in Plain Sight}
You can spend 1 minute creating camouflage for yourself. You must have access to fresh mud, dirt, Plants, soot, and other naturally occurring materials with which to create your camouflage.

Once you are camouflaged in this way, you can try to hide by pressing yourself up against a solid surface, such as a tree or wall, that is at least as tall and wide as you are. You gain a +10 bonus to Dexterity (Stealth) checks as long as you remain there without moving or taking ACTIONS. Once you move or take an Action or a Reaction, you must camouflage yourself again to gain this benefit.

\subsection*{Vanish}
Starting at 14th level, you can use the Hide action as a Bonus Action on your turn, Also, you can't be tracked by nonmagical means, unless you chose to leave a trail.

\section*{Wood Elf Traits}
\subsection*{Darkvision}
Accustomed to twilit forests and the night sky, you have superior vision in dark and dim conditions. You can see in dim light within 60 feet of you as if it were bright light, and in darkness as if it were dim light. You can't discern color in darkness, only shades of gray.

\subsection*{Fey Ancestry}
You have advantage on saving throws against being charmed, and magic can't put you to sleep.

\subsection*{Trance}
Elves don't need to sleep. Instead, they meditate deeply, remaining semiconscious, for 4 hours a day. (The Common word for such meditation is “trance”.) While meditating, you can dream after a fashion; such dreams are actually mental exercises that have become reflexive through years of practice. After resting in this way, you gain the same benefit that a human does from 8 hours of sleep.

\subsection*{Elf Weapon Training}
You have proficiency with the longsword, shortsword, shortbow, and longbow.

\subsection*{Mask of the Wild}
You can attempt to hide even when you are only lightly obscured by foliage, heavy rain, falling snow, mist, and other natural phenomena.

\section*{Background}
\subsection*{Folk Hero}
You come from a humble social rank, but you are destined for so much more. Already the people of your home village regard you as their champion, and your destiny calls you to stand against the tyrants and monsters that threaten the common folk everywhere.

\subparagraph*{Skill Proficiencies} Animal Handling, Survival
\subparagraph*{Tool Proficiencies} One type of artisan's tools, vehicles (land)
\subparagraph*{Equipment} A set of artisan's tools (one of your choice), a shovel, an iron pot, a set of common clothes, and a pouch containing 10 gp

\section*{Feats}
\subsection*{Sharpshooter}
You have mastered ranged weapons and can make shots that others find impossible. You gain the following benefits:
\begin{itemize}
  \item Attacking at long range doesn't impose disadvantage on your ranged weapon attack rolls.
  \item Your ranged weapon attacks ignore half and three-quarters cover.
  \item Before you make an attack with a ranged weapon that you are proficient with, you can choose to take a -5 penalty to the attack roll. If that attack hits, you add +10 to the attack's damage.
\end{itemize}

\subsection*{Wood Elf Magic}
You learn the magic of the primeval woods, which are revered and protected by your people. You learn one Druid cantrip of your choice. You also learn the Longstrider and Pass Without Trace spells, each of which you can cast once without expending a spell slot. You regain the ability to cast these two spells in this way when you finish a long rest. Wisdom is your spellcasting ability for all three spells.

\section*{Spells}

\subsection*{Bonus Spells}

\DndSpellHeader
  {Longstrider}
  {1st-Level Transmutation}
  {1 Action}
  {Touch}
  {V S M (A pinch of dirt)}
  {1 hour}

You touch a creature. The target's speed increases by 10 feet until the spell ends.

\subparagraph*{At Higher Levels} When you cast this spell using a spell slot of 2nd level or higher, you can target one additional creature for each slot level above 1st.

\DndSpellHeader
  {Pass Without Trace}
  {2nd-Level Abjuration}
  {1 Action}
  {Self}
  {V S M (Ashes from a burned leaf of mistletoe and a sprig of spruce)}
  {Concentration, up to 1 hour}

A veil of shadows and silence radiates from you, masking you and your companions from detection. For the duration, each creature you choose within 30 feet of you (including you) has a +10 bonus to Dexterity (Stealth) checks and can't be tracked except by magical means. A creature that receives this bonus leaves behind no tracks or other traces of its passage.

\subsection*{Cantrips}

\DndSpellHeader
  {Control Flames (Lessened)}
  {Druidic Cantrip}
  {1 Action}
  {60 feet (3 ft cube)}
  {S}
  {Instantaneous}

You choose nonmagical flame that you can see within range and that fits within a 3-foot cube. You affect it in one of the following ways:
\begin{itemize}
	\item You instantaneously expand the flame 3 feet in one direction, provided that wood or other fuel is present in the new location.
	\item You instantaneously extinguish the flames within the cube.
	\item You double or halve the area of bright light and dim light cast by the flame, change its color, or both. The change lasts for 1 hour.
	\item You cause simple shapes—such as the vague form of a creature, an inanimate object, or a location—to appear within the flames and animate as you like. The shapes last for 1 hour.
\end{itemize}
If you cast this spell multiple times, you can have up to three of its non-instantaneous effects active at a time, and you can dismiss such an effect as an action.


\subsection*{Level 1}

\DndSpellHeader
  {Ensnaring Strike}
  {1st-Level Conjuration}
  {1 Bonus Action}
  {Self}
  {V}
  {Concentration, up to 1 minute}

The next time you hit a creature with a weapon attack before this spell ends, a writhing mass of thorny vines appears at the point of impact, and the target must succeed on a Strength saving throw or be restrained by the magical vines until the spell ends. A Large or larger creature has advantage on this saving throw. If the target succeeds on the save, the vines shrivel away.

While restrained by this spell, the target takes 1d6 piercing damage at the start of each of its turns. A creature restrained by the vines or one that can touch the creature can use its action to make a Strength check against your spell save DC. On a success, the target is freed.

\subparagraph*{At Higher Levels} If you cast this spell using a spell slot of 2nd level or higher, the damage increases by 1d6 for each slot level above 1st.

\DndSpellHeader
  {Hail of Thorns}
  {1st-Level Conjuration}
  {1 Bonus Action}
  {Self}
  {V}
  {Concentration, up to 1 minute}

The next time you hit a creature with a ranged weapon attack before the spell ends, this spell creates a rain of thorns that sprouts from your ranged weapon or ammunition. In addition to the normal effect of the attack, the target of the attack and each creature within 5 feet of it must make a Dexterity saving throw. A creature takes 1d10 piercing damage on a failed save, or half as much damage on a successful one.

\subparagraph{At Higher Levels} If you cast this spell using a spell slot of 2nd level or higher, the damage increases by 1d10 for each slot level above 1st (to a maximum of 6d10).

\subsection*{Level 2}

\DndSpellHeader
  {Find Traps}
  {2nd-Level Divination}
  {1 Action}
  {120 feet}
  {V S}
  {Instantaneous}

You sense the presence of any trap within range that is within line of sight. A trap, for the purpose of this spell, includes anything that would inflict a sudden or unexpected effect you consider harmful or undesirable, which was specifically intended as such by its creator. Thus, the spell would sense an area affected by the alarm spell, a glyph of warding, or a mechanical pit trap, but it would not reveal a natural weakness in the floor, an unstable ceiling, or a hidden sinkhole.

This spell merely reveals that a trap is present. You don't learn the location of each trap, but you do learn the general nature of the danger posed by a trap you sense.

\DndSpellHeader
  {Spike Growth}
  {2nd-Level Transmutation}
  {1 Action}
  {150 feet}
  {V, S, M (seven sharp thorns or seven small twigs, each sharpened to a point)}
  {Concentration, up to 10 minutes}

The ground in a 20-foot radius centered on a point within range twists and sprouts hard spikes and thorns. The area becomes difficult terrain for the duration. When a creature moves into or within the area, it takes 2d4 piercing damage for every 5 feet it travels.

The transformation of the ground is camouflaged to look natural. Any creature that can't see the area at the time the spell is cast must make a Wisdom (Perception) check against your spell save DC to recognize the terrain as hazardous before entering it.

\DndSpellHeader
  {Cordon of Arrows}
  {2nd-Level Transmutation}
  {1 Action}
  {5 feet}
  {V, S, M (four or more arrows or bolts)}
  {8 hours}

You plant four pieces of nonmagical ammunition – arrows or crossbow bolts – in the ground within range and lay magic upon them to protect an area. Until the spell ends, whenever a creature other than you comes within 30 feet of the ammunition for the first time on a turn or ends its turn there, one piece of ammunition flies up to strike it. The creature must succeed on a Dexterity saving throw or take 1d6 piercing damage. The piece of ammunition is then destroyed. The spell ends when no ammunition remains.

When you cast this spell, you can designate any creatures you choose, and the spell ignores them.

\subparagraph{At Higher Levels} When you cast this spell using a spell slot of 3rd level or higher, the amount of ammunition that can be affected increases by two for each slot level above 2nd.

\subsection*{Level 3}

\DndSpellHeader
  {Lightning Arrow}
  {3rd-Level Transmutation}
  {1 Bonus Action}
  {Self}
  {V, S}
  {Concentration, up to 1 minute}

The next time you make a ranged weapon attack during the spell's duration, the weapon's ammunition, or the weapon itself if it's a thrown weapon, transforms into a bolt of lightning. Make the attack roll as normal. The target takes 4d8 lightning damage on a hit, or half as much damage on a miss, instead of the weapon's normal damage.

Whether you hit or miss, each creature within 10 feet of the target must make a Dexterity saving throw. Each of these creatures takes 2d8 lightning damage on a failed save, or half as much damage on a successful one.

The piece of ammunition or weapon then returns to its normal form. 

\subparagraph*{At Higher Levels} When you cast this spell using a spell slot of 4th level or higher, the damage for both effects of the spell increases by 1d8 for each slot level above 3rd.

\subparagraph*{Miscellaneous} Dexterity and Sharpshooter can be applied to this spells Attack and Damage Roll.

\DndSpellHeader
  {Conjure Barrage}
  {3rd-Level Conjuration}
  {1 Action}
  {Self (60-foot cone)}
  {V, S, M (one piece of ammunition or a thrown weapon)}
  {Instantaneous}

You throw a nonmagical weapon or fire a piece of nonmagical ammunition into the air to create a cone of identical weapons that shoot forward and then disappear. Each creature in a 60-foot cone must succeed on a Dexterity saving throw. A creature takes 3d8 damage on a failed save, or half as much damage on a successful one. The damage type is the same as that of the weapon or ammunition used as a component.

\DndSpellHeader
  {Protection from Energy}
  {3rd-Level Abjuration}
  {1 Action}
  {Touch}
  {V, S}
  {Concentration, up to 1 hour}
  
For the Duration, the willing creature you touch has Resistance to one damage type of your choice - acid, cold, fire, lightning, or thunder.

\subsection*{Level 4}

\DndSpellHeader
  {Stoneskin}
  {4th-Level Abjuration}
  {1 Action}
  {Touch}
  {V, S, M (Diamond dust worth 100gp, which the spell consumes)}
  {Concentration, up to 1 hour}
  
This spell turns the flesh of a willing creature you touch as hard as stone. Until the spell ends, the target has resistance to nonmagical bludgeoning, piercing, and slashing damage.

\vfill\eject
\section*{Miscellaneous}
\subsection*{Attack and Damage Rolls}
\subsubsection*{Melee Weapons}
\paragraph*{Attack Roll}\hfill\\
\underline{\textit{Shortsword (Finesse):}}\\
1d20 + DEX-Modifier + Proficiency Modifier\\
\indent Current Max: \intcalcAdd{20}{\intcalcAdd{\calculateModifier{\DexterityScoreValue}}{\ProficiencyValue}}
\paragraph*{Damage Roll}\hfill\\
\underline{\textit{Shortsword (Finesse):}}\\
1d8 + DEX-Modifier\\
\indent Current Max: \intcalcAdd{8}{\calculateModifier{\DexterityScoreValue}}
\subsubsection*{Ranged Weapons}
\paragraph*{Attack Roll}\hfill\\
\underline{\textit{Oathbow:}}\\
1d20 + DEX-Modifier + Proficiency Modifier + 2 (Fighting Style: Archer) + 2 (Bracers of Trickshot) + 2 (Quiver of the Storm's Fury) (- 5 (Sharpshooter)) \\
\indent Current Max (Normal): \intcalcAdd{20}{\intcalcAdd{\calculateModifier{\DexterityScoreValue}}{\intcalcAdd{\ProficiencyValue}{\intcalcAdd{2}{\intcalcAdd{2}{2}}}}} \\
\indent Current Max (Sharpshooter): \intcalcSub{\intcalcAdd{20}{\intcalcAdd{\calculateModifier{\DexterityScoreValue}}{\intcalcAdd{\ProficiencyValue}{\intcalcAdd{2}{\intcalcAdd{2}{2}}}}}}{5}
\paragraph*{Damage Roll}\hfill\\
\underline{\textit{Oathbow:}}\\
1d8 + 2 (Wedge Stone) + DEX-Modifier + 2 (Bracers of Trickshot) + 2 (Quiver of the Storm's Fury) (+ 3d6 (against Sworn Enemy)) (+ 10 (Sharpshooter))\\
\indent Current Max (Normal): \intcalcAdd{8}{\intcalcAdd{2}{\intcalcAdd{\calculateModifier{\DexterityScoreValue}}{\intcalcAdd{2}{2}}}} \\
\indent Current Max (Sworn Enemy): \intcalcAdd{8}{\intcalcAdd{2}{\intcalcAdd{\intcalcAdd{\calculateModifier{\DexterityScoreValue}}{\intcalcAdd{2}{2}}}{\intcalcMul{3}{6}}}} \\
\indent Current Max (Sharpshooter): \intcalcAdd{8}{\intcalcAdd{2}{\intcalcAdd{\intcalcAdd{\calculateModifier{\DexterityScoreValue}}{\intcalcAdd{2}{2}}}{10}}} \\
\indent Current Max (Sharpshooter, Sworn Enemy): \intcalcAdd{8}{\intcalcAdd{2}{\intcalcAdd{\intcalcAdd{\calculateModifier{\DexterityScoreValue}}{\intcalcAdd{2}{2}}}{\intcalcAdd{\intcalcMul{3}{6}}{10}}}}
\subsubsection*{Special Attacks}
\paragraph*{Attack Roll}\hfill\\
\underline{\textit{Unarmed Strike:}}\\
1d20 + STR-Modifier + Proficiency Modifier\\
\indent Current Max: \intcalcAdd{20}{\intcalcAdd{\calculateModifier{\StrengthScoreValue}}{\ProficiencyValue}}
\paragraph*{Damage Roll}\hfill\\
\underline{\textit{Unarmed Strike:}}\\
1 + STR-Modifier\\
\indent Current Max: \intcalcAdd{1}{\calculateModifier{\StrengthScoreValue}}

\chapter*{Magical Items}
\section*{Oathbow}
\textit{very rare (requires attunement)}

When you nock an arrow on this bow, it whispers in Elvish, "Swift defeat to my Enemies." When you use this weapon to make a Ranged Attack, you can, as a Command phrase, say, "Swift death to you who have wronged me." The target of your Attack becomes your Sworn Enemy until it dies or until dawn seven days later. You can have only one such Sworn Enemy at a time. When your Sworn Enemy dies, you can choose a new one after the next dawn.

When you make a Ranged Attack roll with this weapon against your Sworn Enemy, you have advantage on the roll. In addition, your target gains no benefit from cover, other than total cover, and you suffer no disadvantage due to Long Range. If the Attack hits, your Sworn Enemy takes an extra 3d6 piercing damage.

While your Sworn Enemy lives, you have disadvantage on Attack rolls with all other Weapons.

\section*{Bracers of Trickshot}
\textit{rare (requires attunement)}

While wearing these bracers, you have proficiency with the longbow and shortbow, and you gain a +2 to attack and damage rolls on ranged attacks made with such weapons.

Once per long rest you can make a trickshot with a bow.

\section*{Quiver of Ehlonna}
\textit{wondrous item, uncommon}

Each of the quiver's three compartments connects to an extradimensional space that allows the quiver to hold numerous items while never weighing more than 2 pounds. The shortest compartment can hold up to sixty arrows, bolts, or similar objects. The midsize compartment holds up to eighteen javelins or similar objects. The longest compartment holds up to six long objects, such as bows, quarterstaffs, or spears.

You can draw any item the quiver contains as if doing so from a regular quiver or scabbard.

\vfill\eject

\section*{Quiver of the Storm's Fury}
\textit{wondrous item, very rare}

This quiver is made from the hide of a storm giant's favorite hunting beast and is adorned with lightning motifs. When the ranger draws an arrow from the quiver, it becomes charged with electrical energy, granting a +2 bonus to attack and damage rolls. Additionally, once per short or long rest, the ranger can choose to imbue an arrow with the power of a storm, causing it to create a 20-foot radius sphere of lightning and thunder when it hits a target. Creatures within the sphere must make a DC 17 Dexterity saving throw taking 6d6 lightning damage and 6d6 thunder damage on a failed save, or half as much on a successful one.

\section*{Ring of Free Action}
\textit{rare (requires attunement)}

While you wear this ring, difficult terrain doesn't cost you extra movement. In addition, magic can neither reduce your speed nor cause you to be paralyzed or restrained.

\section*{Wand of Secrets}
\textit{wondrous item, uncommon}

The wand has 3 Charges. While holding it, you can use an Action to expend 1 of its Charges, and if a Secret door or trap is within 30 feet of you, the wand pulses and points at the one nearest to you. The wand regains 1d3 expended Charges daily at dawn.

\section*{Wedge Stone}
\textit{common}

The wedge stone can be used 5 times per long rest. You can take 1 minute to sharpen one arrow increasing its damage by 2.

\vfill\eject
\section*{Arrow-Types}
\subsection*{Arrow of Slaying (Dragon)}
An arrow of slaying is a Magic Weapon meant to slay a particular kind of creature. Some are more focused than others; for example, there are both Arrows of Dragon slaying and Arrows of blue Dragon slaying. If a creature belonging to the type, race, or group associated with an arrow of slaying takes damage from the arrow, the creature must make a DC 17 Constitution saving throw, taking an extra 6d10 piercing damage on a failed save, or half as much extra damage on a successful one.
\subsection*{Boomerang Arrow}
On a miss the Boomerang Arrow returns to the owner's hand.
\subsection*{Drow Arrow}
When indoors or during the night the use of this arrow gives a +1 modifier to hit and damage dice.
\subsection*{Fire Arrow}
This arrow can be lit before using it either by spending an action from a Tinderbox, or spending a free action from an open flame, such as a torch, a bonfire, or a burning tree. When lit, this arrow sets alight any highly flammable materials on impact, and does the normal attack damage from the base weapon plus an extra 1d4 fire damage to the target. It will continue to do 1 fire damage until an action or bonus action is spent to remove it. One of the disadvantages of using the fire arrows is that you lose any advantage when attempting a surprise attacks with it due to its clear visibility.
\subsection*{Ice Arrow}
This magic arrow has a jagged arrowhead formed with a semitransparent pale blue material, reminiscent of ice. Once this arrow is drawn, you can use your bonus action to focus on the arrow, causing it to glow with icy evocation magic. If you make a ranged weapon attack with this glowing arrow before the start of your next turn, the attack deals magical cold damage instead of its normal damage type. If the hit target is a creature, it must succeed on a DC 13 Constitution saving throw or be frozen solid, effectively petrified, until the end of its next turn. If the target has resistance or immunity to cold damage, it automatically succeeds on this saving throw. If the target takes any fire damage while petrified in this manner, the conditions ends.

You can expend 1 magic point when you use the bonus action to cause the attack deal an extra 1d6 cold damage. If you have the Spellcasting feature when you do this, the DC becomes equal to your spell save DC if it would otherwise be lower.

\ItemCategory{Weapons}
\ItemSubCategory{Ammunition}
\ItemFolder{Ethereal_Arrow}

\chapter*{Ethereal Arrow}
\itemDescriptionAndImage{Weapon (Arrow), very rare}{Ethereal_Arrow.png}{10.5cm}\\

\section*{Appearance}

The "Ethereal Arrow" is a striking and intricately designed magical item, a marvel of arcane craftsmanship. The arrow shaft is forged from copper or bronze, giving it a timeless, Atlantean-like appearance that speaks of ancient civilizations and lost knowledge. Delicate, glowing arcane runes are meticulously etched along its length, pulsing softly with an ethereal, magical aura that hints at the arrow's extraordinary teleportation powers.

The arrowhead is the epitome of mystical energy, crafted from a plasma-like arcane, ethereal material that looks like it is made of pure lightning. It emits a dynamic, powerful glow that seems to dance and flicker, casting shimmering reflections in the dim light. This otherworldly tip suggests immense power and precision, capable of piercing through the very fabric of reality.

At the rear, the fletchings are a spectacle of pure energy, glowing in an electric blue. They resemble feathers but are composed entirely of radiant energy, adding to the arrow's otherworldly and futuristic aesthetic. These energy fletchings not only enhance the arrow's appearance but also hint at its enhanced flight capabilities, making it swift and true in its path.

The overall appearance of the "Ethereal Arrow" conveys a blend of ancient magic and advanced technology, making it look both powerful and valuable. It is a testament to the fusion of old-world enchantment and modern innovation, a prized possession for any adventurer or collector.

\section*{History}

The Ethereal Arrow has a storied history steeped in ancient magic and legendary craftsmanship. It was developed by the formidable Storm Giants, known for their mastery over elemental forces, with the assistance of Laucian Ilphelkiir, a renowned wood elf ranger. The creation of this magical ammunition was driven by the need for a weapon effective underwater, where common arrows failed.

Laucian's expertise in archery and arcane energies was crucial in the development process. The shaft, made from sacred copper and bronze, was crafted to channel immense magical energies. Laucian meticulously etched intricate arcane runes, granting the arrow its signature teleportation abilities. The arrowhead, resembling pure lightning, was imbued with the essence of storms, while the fletchings, glowing in electric blue, were composed of radiant energy, enhancing its flight.

The Ethereal Arrow stands as a testament to the extraordinary collaboration between the Storm Giants and Laucian Ilphelkiir, embodying a fusion of ancient magic and advanced craftsmanship. It remains a revered weapon, prized for its unparalleled power and historical significance.

\section*{Magic}

Upon being shot, the Ethereal Arrow immediately blinks out of the Material Plane and into the Ethereal Plane. It travels through this spectral realm, bypassing obstacles and defenses that exist only in the Material Plane. Just moments before hitting its intended target, the arrow re-materializes in the Material Plane, striking with uncanny precision. This effect allows the arrow to ignore cover and barriers, ensuring a direct hit on the target.


\chapter*{Aftermath Story}

\entryfont With Iymrith, the formidable Dragon, vanquished and the Storm Giant's Kingdom restored to its former glory, Laucian Ilphelkiir emerges as a figure of immense renown and reverence. The giants hail him as a legendary hero, their savior in the darkest hour, and an embodiment of the indomitable spirit of the Wood Elf people. Laucian's name becomes etched in the annals of history, whispered with awe and gratitude by the giants and spoken with admiration by his companions and allies.

Having played an instrumental role in the defeat of Iymrith and the restoration of the Storm Giant's Kingdom, Laucian is called upon to guide and advise the giants as they rebuild their realm. His deep knowledge of the land and his intimate connection to nature prove invaluable in the efforts of restoration and healing. With his guidance, the once-devastated territories begin to flourish once again, teeming with vibrant flora and fauna.

Laucian takes on the role of a trusted advisor to the Storm Giant King, counseling him on matters of diplomacy, governance, and the delicate balance between the giants and the surrounding lands. He serves as an emissary, forging alliances with neighboring realms, fostering goodwill, and facilitating cultural exchanges. His diplomatic finesse and natural charm enable him to navigate the complexities of inter-kingdom politics, smoothing the way for peaceful coexistence and mutual understanding.

Amidst his responsibilities to the Storm Giants, Laucian remains committed to his Wood Elf heritage. He establishes a sanctuary within the heart of the re-established Storm Giant's Kingdom, a haven where creatures of the forest find refuge and thrive under his watchful protection. The sanctuary becomes a symbol of harmony and reverence for nature, drawing scholars, druids, and seekers of wisdom from far and wide. It becomes a center for learning, where ancient texts and knowledge are preserved, and where the delicate balance between mortals and nature is cherished and nurtured.

Beyond the realm of giants, Laucian's heroic exploits become the stuff of legends. Bards and storytellers weave tales of his bravery, his unwavering determination, and his connection to the primal forces of the world. His deeds inspire countless souls to stand up against injustice and defend the natural world from encroachment and exploitation. His name becomes a rallying cry for those seeking to restore balance and protect the fragile ecosystems that sustain life.

As his legend spreads, Laucian receives invitations from kingdoms and realms far and wide. Kings and queens seek his wisdom, hoping to tap into his profound understanding of the world and its intricate workings. Laucian becomes known as a sage and seer, his words carrying the weight of ancient wisdom and profound insights. He imparts his knowledge, offering guidance and solutions to the pressing challenges faced by various realms.

Despite the demands of his newfound fame and influence, Laucian remains humble and true to his roots. He maintains close ties with his companions from the Great Giants War: Storr, the irreverent rogue; Jun Wu, the enigmatic monk; Dustin "Floppers" Harper, the quick-witted necromancer rabbit; and Kyuss, the stoic sorcerer. The bonds forged in the fires of battle hold firm, and they continue to embark on new adventures together, their individual skills complementing one another as they face new challenges.

As the years pass, Laucian's influence extends beyond mortal realms. Nature itself responds to his presence, with animals drawn to him as if sensing his kinship with the natural world. Whispered rumors spread of his ability to commune with ancient spirits and harness the primal energies that flow through the land. Some even claim that the spirits of fallen Wood Elves guide his arrows, ensuring unerring accuracy and swift justice.

In the twilight of his life, Laucian retreats to a secluded grove deep within the heart of the Storm Giant's Kingdom. Surrounded by towering trees and the gentle rustling of leaves, he contemplates the magnitude of his journey. He imparts his wisdom to a select group of disciples, passing down the traditions and teachings of the Wood Elves, as well as the hard-earned lessons of his own experiences.

Word of Laucian's secluded grove spreads, and those who seek enlightenment and harmony with nature seek out the legendary hero. His sanctuary becomes a place of pilgrimage, where the weary find solace, the lost find guidance, and the broken find healing. Laucian's days are spent in quiet reflection and contemplation, sharing his wisdom and offering counsel to those who seek it.

As his time draws to a close, Laucian's legacy lives on in the hearts and minds of those he has touched. The Storm Giant's Kingdom thrives under the careful stewardship of the giants, their alliance with neighboring realms ensuring a lasting era of peace and prosperity. The delicate balance between mortals and nature is maintained, as generations inspired by Laucian's teachings continue the work of safeguarding the natural world.

When Laucian finally passes from the mortal realm, the news reverberates across kingdoms and lands. Bards compose epic ballads and eulogies to commemorate his extraordinary life, celebrating his unwavering dedication to justice, his boundless compassion for all living things, and his tireless efforts to preserve the harmony of the natural world. His memory lives on, inspiring generations to come to cherish their connection to the land and strive for a future where mortals and nature coexist in harmony.

\end{document}
